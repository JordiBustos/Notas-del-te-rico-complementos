\section{Repaso de funciones}

Una función $f: A \to B$ es un objeto que consta de tres partes: un conjunto $A$ (dominio), un conjunto $B$ (codominio) y una regla que permite asociar todo elemento de $A$ a un único elemento de $B$. Es decir, $f(x) \in B$, donde $x \in A$.
Además, $f(x) = y$, lo que significa que $f$ asigna a $x$ el valor $f(x)$.

El gráfico de $f: A \to B$ es el subconjunto de $A \times B$ dada por $(x, f(x))$ con $x \in A$ y $f(x) \in B$.\\
Notamos $G(f) = \{ (x, y) \in A \times B : y = f(x) \}$.

\begin{definition}
    $f: A \to B$ es inyectiva cuando para $x, y \in A$, $f(x) = f(y) \Rightarrow x = y$.
\end{definition}

\begin{definition}
    $f: A \to B$ es suryectiva cuando para $(\forall y\in B)(\exists x\in A)(f(x) = y)$
\end{definition}

\begin{definition}
    $f: A \to B$ si es inyectiva y suryectiva.
\end{definition}

\begin{definition}
    Dados $f: A \to B$ y $X \subset A$ se llama imagen de $X$ por $f$ al conjunto $f(X) = \{f(x) : x \in X\}$.
\end{definition}

\subsection{Propiedades}

\begin{prop}
    $f(X) \cup f(Y) = f(X \cup Y)$
\end{prop}

\begin{prop}
    $f(X \cap Y) \subset f(X) \cap f(Y)$. La igualdad vale si y sólo si es inyectiva.
    \begin{proof}
        Sea $a \in f(X \cap Y) \Rightarrow $ $\exists x \in X \cap Y : f(x) = b \Rightarrow$ $ x \in X \Rightarrow f(x) \in f(X) $ y $ y \in Y \Rightarrow f(y) \in f(Y). $ \\ Si $f: A \to B$ no es inyectiva $\Rightarrow \exists x \neq y : f(x) = f(y)$. Si $X = \{x\}$, $Y = \{y\}$ $\Rightarrow X \cap Y = \varnothing$. $f(X) \cap f(Y) = \{f(x)\}$ luego $f(X \cap Y) = \varnothing$. \\ Si $f$ es inyectiva, sea $y \in f(X) \cap f(Y) \Rightarrow \exists a \in X, b \in Y$ tal que $f(a) = f(b) = y$. Como $f$ es inyectiva $a = b \Rightarrow a \in X \cap Y \Rightarrow y = f(a)$, $y \in f(X \cap Y) \Rightarrow f(X) \cap f(Y) \subset f(X \cap Y) \\
            \therefore$ Si $f$ es inyectiva son iguales.
    \end{proof}
\end{prop}


\begin{prop}
    $X \subset Y \Rightarrow f(X) \subset f(Y)$
\end{prop}

\begin{prop}
    $f(\varnothing) = \varnothing$
\end{prop}

\begin{definition}
    Dados $f: A \to B$ y $Y \subset B$ se llama preimagen de $Y$ por $f$ al conjunto $f^{-1}(Y) = \{x \in A : f(x) = y, \forall y \in Y\}$.
\end{definition}

\subsection{Función inversa}

Sea $f: A \to B$:

\begin{prop}
    $f^{-1}(X) \cup f^{-1}(Y) = f^{-1}(X \cup Y)$
\end{prop}

\begin{prop}
    $f^{-1}(X) \cap f^{-1}(Y) = f^{-1}(X \cap Y)$
\end{prop}

\begin{prop}
    $f^{-1}(Y^c) = (f^{-1}(Y))^c$
\end{prop}

\begin{prop}
    $Y \subset Z \Rightarrow f^{-1}(Y) \subset f^{-1}(Z)$
\end{prop}

\begin{prop}
    $f^{-1}(B) = A$
\end{prop}

\begin{prop}
    $f^{-1}(\varnothing) = \varnothing$
\end{prop}

\subsection{Composición de funciones}

Sean $f: A \to B$ y $g: B \to C$, definimos $g \circ f: A \to C$ como $(g \circ f)(x) = g(f(x)) \forall x \in X$, es suficiente que $f(A) \subset B$.

\begin{prop}
    Composición de funciones suryectivas/inyectivas es \\ suryectiva/inyectiva.
\end{prop}

\begin{prop}
    $(g \circ f)^{-1}(Z) = f^{-1}(g^{-1}(Z))$.
\end{prop}

\begin{definition}
    La restricción de $f$ en un subconjunto $X \subset A$ la notamos $f|_X: X \to B$.
\end{definition}

\begin{definition}
    Dada $f: A \to B$ y $g: B \to A$, $g$ es una inversa a izquierda si y sólo si $g \circ f = id_A$. $\exists g$ si y sólo si $f$ es inyectiva. \\
    Análoamente para la inversa a derecha si $f$ es suryectiva.
    Si $f$ es biyectiva $\Rightarrow g$ es inversa a ambos lados y es única.
\end{definition}

\subsection{Familia de funciones}

Sea $L$ un conjunto de elementos que llamamos índices y representamos genéricamente con $\lambda$. Dado un conjunto X, una familia de elementos de X con índices en $L$ es $X: L \to x$.
El valor de $x$ en $\lambda \in L$ lo notamos $x_\lambda$ y la familia $(x_\lambda)_{\lambda \in L}$.

\begin{eg}
    $L = \{ 1, 2 \}$ los valores de $x: \{1, 2\} \to X$ se representan por $x_1, x_2$, es decir que los puedo identificar con pares ordenados $(x_1, x_2)$ de elementos de $X$.
\end{eg}

Una familia con elementos en $\N$ se llama sucesión. $(x_n)_{n \in \N}$ de elementos de $X$ es una función de $x: \N \to X$ donde $x_n = x(n)$.

Las propiedades enunciadas previamente se pueden extender a cualquier familia de conjuntos.

\section{Números naturales}

Partimos de un conjunto $\N$ y una función $S: \N \to \N$ que cumple los siguientes axiomas (de Peano):
\begin{enumerate}
    \item Es inyectiva.
    \item $\N - S(\N)$ tiene un solo elemento y lo llamamos 1.
    \item Principio de inducción, si $X \subset \N$ tal que $1 \in X$ y $\forall m \in X$ vale $S(m) \Rightarrow X = \N$.
\end{enumerate}

El principio de inducción permite definir operaciones

La suma se define como $m+1 = S(m)$, $m+S(n) = S(m+n)$.

\begin{prop}
    Asociatividad: sea $X = \{ p \in \N : m+(n+p) = (m+n) +p, \forall n,m \in \N \}$.
    \begin{proof}
        $1 \in X$, $p \in X \Rightarrow m + (n+S(p)) = m + S(n+p) = S(m +(n+p)) = S((n+m)+p) = (m+n) + S(p)$.
        Por inducción $X = \N$.
    \end{proof}
\end{prop}


\begin{prop}
    Conmutatividad: $n+m = m+n$.
\end{prop}

\begin{prop}
    Ley de cancelación: $m+n = m+p \Rightarrow n=p$.
\end{prop}

\begin{prop}
    Tricotomía: $m,n \in \N$, si $m > n, \exists p:m+p=n$. Si $m < n$,  $\exists p \in \N :n+p=m$.
\end{prop}

\begin{definition}
    La multiplicación se define recursivamente como: $m \times 1 =m$ y $m \times (n+1) = m \times n + m$. \\
    Cumple la asociatividad, conmutatividad, ley de cancelación y monotonía.
\end{definition}

\begin{theorem}
    Principio de buena ordenación \\
    Todo subconjunto no vacío $A \subset \N$ tiene un elemento mínimo.
    \begin{proof}
        Llamemos $\mathbb{I}_m = \{ p \in \N : 1 \leq p \leq n \} = [[ 1, m ]]$ y $X = \{ m \in \N : \mathbb{I}_m \subset \N - A \}$. \\
        Si $1 \in A \Rightarrow 1$ es primer elemento.
        Si $1 \notin A \Rightarrow 1 \in X$ como $X \neq \N$ pues $X \subseteq \N - A$ y $A \neq \varnothing$.
        Por el principio de inducción $\exists m \in X$ tal que $m+1 \notin X$, si no tendríamos que $X = \N$. Luego todos los elementos entre $1$ y $m$ están en $\N - A$ y $m+1 \in A$, se sigue que $a = m+1$ es primer elemento de A.
    \end{proof}
\end{theorem}


\section{Cuerpo}

Un cuerpo es un conjunto dotado de dos operaciones, suma y producto y se denota por $\mathbb{K}$.

\subsection{Axiomas de la suma}

1) Asociatividad: $\forall x,y,z \in \mathbb{K}, (x+y)+z=x+(y+z)$.

2) Conmutatividad: $\forall x,y \in \mathbb{K}, x+y=y+x$.

3) Neutro: $\exists 0 \in \mathbb{K} :x+0 = x, \forall x \in X$.

4) Opuesto: $\forall x \in X, \exists -x \in \mathbb{K} : x +(-x) = 0_\mathbb{K}$.

\subsection{Axiomas del producto}

1) Asociatividad: $\forall x,y,z \in \mathbb{K}, (xy)z=x(yz)$.

2) Conmutatividad: $\forall x,y \in \mathbb{K}, xy=yx$.

3) Neutro: $\exists 1 \in \mathbb{K}-\{0\} :x \cdot 1 = x, \forall x \in X$.

4) Inverso: $\forall x \in X, \exists x^{-1} \in \mathbb{K} : x \cdot x^{-1} = 1_\mathbb{K}$.

Axioma de distributividad: $\forall x,y,z \in \mathbb{K}, x(y+z)=xy+xz$.

\begin{eg}
    $\Q, \Z_2$.
\end{eg}

\subsection{Cuerpos Ordenados}

Un cuerpo ordenado es un cuerpo $\mathbb{K}$ que tiene un subconjunto $P \in \mathbb{K}$ llamado conjunto de elementos positivos de $\mathbb{K}$ que cumplen:
\begin{enumerate}
    \item $x,y \in P \Rightarrow x+y \in P, xy \in P$.
    \item $x \in \mathbb{K} \Rightarrow x \in P$ o $-x \in P$ o $x=0$.
\end{enumerate}

\begin{eg}
    $\Q$ con $P = \{ p/q:p,q\in \N \}$, $\Z_2$ no es ordenado.
\end{eg}

\begin{prop}
    Dado un cuerpo ordenado si $a \neq 0 \Rightarrow a²\in P$.
\end{prop}

En un cuerpo ordenado definimos $x<y$ para significar que $y-x\in P$.

La relación $<$ tiene las siguientes propiedades:

\begin{prop}
    Transitividad: $x<y$ y $y<z \Rightarrow x<z$.
\end{prop}

\begin{prop}
    Tricotomía: $x,y \in \mathbb{K} \Rightarrow x=y$ o $x<y$ o $x>y$.
\end{prop}

\begin{prop}
    Monotonía de la suma $x<y \Rightarrow x+z<y+z$.
\end{prop}

\begin{prop}
    Monotonía del producto $x<y, 0<z \Rightarrow xz <yz$.
\end{prop}

En el cuerpo ordenado $\mathbb{K}$ escribimos $x \leq y$ para significar $x<y$ o $x=y$. O sea $y-x \in P \cup \{0\}$. Con esta relación se cumplen todas las propiedades anteriores y la antisimetría. $x \leq y, y \leq x \Rightarrow x = y$.

Además tenemos la noción de intervalo, dados $a,b \in \mathbb{K}$ definimos $[a,b ] = \{ x \in \mathbb{K} : a \leq x \leq b \}, [a, b), (a, b], (a,b), [a, +\infty), (-\infty, b], (a, +\infty), (-\infty, b), [a, a] = \{a\}$.

Un subconjunto de $X \subset K$ se dice acotado inferiormente, superiormente si tiene cota inferior o cota superior. $\exists b \in \mathbb{K}: x \leq b, \forall x \in X$.
