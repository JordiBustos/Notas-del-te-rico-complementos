\section{Punto de acumulación}

\begin{definition}[Punto de acumulación]
  \(X \subset \R^n\). Un punto \(x \in X\) es de acumulación si pertenece a \(\overline{X - \{ x \}} \).
\end{definition}

\begin{definition}[Conjunto derivado]
  El conjunto derivado de X es el conjunto de todos los puntos de acumulación de \(X\) y lo notamos \(X^\prime \).
\end{definition}

\begin{prop}
  Dados \(X \subset \R^n, x \in X\), son equivalentes:
  \begin{enumerate}
    \item \(x\) es un punto de acumulación.
    \item \(\forall r > 0, B_r(x)\) contiene un punto de \(X\) distinto de \(x\).
    \item Todo entorno abierto de \(x\) contiene un punto de \(X\) distinto de \(x\).
    \item Todo entorno de \(x\) contiene un punto de \(X\) distinto de \(x\).
  \end{enumerate}

  \begin{proof}
    (4) \(\Rightarrow \) (3) \(\Rightarrow \) (2) son inmediatos. \\
    (2) \(\Rightarrow \) (1) Por contrarrecíproco: \(x \in X\) no es punto de acumulación de \(X\) luego \(x \in {(\overline{X - \{x\}})}^c \). Como ese conjunto es abierto \(\exists r > 0, B_r(x) \subset {(\overline{X - \{ x \}})}^c \Rightarrow B_r(x) \cap (X - \{x \}) \subset B_r(x) \cap (\overline{X - \{x\}}) = \varnothing \). \\
    (1) \(\Rightarrow \) (4) Sea \(x \in X^{\prime} \) y \(N\) un entorno de \(x\). Como \(x \in \overline{X - \{x\}} \Rightarrow N \cap (X - \{x\}) \neq \varnothing \).
  \end{proof}
\end{prop}

\begin{prop}
  Si \(X \subset \R^n \Rightarrow \overline{X} = X \cup X^{\prime} \).
  \begin{proof}
    Sea \(X \subset \R^n\) sabemos que \(X \subset \overline{X} \). Por otro lado si \(x \in X^{\prime} \Rightarrow \) \\
    Para cualquier entorno \(N\) de \(X\), \(N \cap X \subset N \cap (\overline{X - \{x\}}) \neq \varnothing \Rightarrow \) \\
    \(x \in \overline{X} \Rightarrow X^{\prime} \subset \overline{X} \Rightarrow X \cup X^{\prime} \subset \overline{X} \). \\

    Sea \(x \in \overline{X} \) y supongamos que \(x \notin X\). Para cada entorno \(N\) de \(X\) tenemos que \(N \cap (X - \{x\}) = N \cap X \neq \varnothing \Rightarrow x \in X^{\prime} \), luego \(\overline{X} \subset X \cup X^{\prime} \). \\
    \(\therefore \overline{X} = X \cup X^{\prime} \).
  \end{proof}
\end{prop}

\clearpage

\begin{prop}
  Sea \(X \subset \R^n\). Un punto \(x\) es punto de acumulación de \(X \iff \forall \) entorno abierto de \(X\) contiene infinitos elementos de \(x\).

  \begin{proof}
    \(\Leftarrow \) Si todo entorno abierto de \(x\) contiene infinitos elementos de \(X\) entonces hay alguno distinto de \(x\) y \(x \in X^{\prime} \). \\
    \(\Rightarrow \) Sea \(x \in X^{\prime} \) y \(N\) un entorno abierto de \(x \Rightarrow \exists r > 0 : B_r(x) \subset N\). \\
    Construyamos una sucesión de puntos \({(x_n)}_{n \in N} \) pertenecientes a \(N \cap X\) y tales que \(d(x_n, x) > d(x_{n+1}, x) > 0, \forall n \in \N \). \\
    \(x_1 \neq x\), si \(n > 0\) y ya elegimos \(x_1, \cdots x_n : d(x_1, x) > \cdots > 0, \forall i \in \{1, \ldots, n-1\} \). Como \(B_{d(x_n, x)}(x)\) es un entorno abierto de \(x\) y \(x \in X^{\prime} \) contiene algún punto distinto de \(x\), lo llamo \(x_{n+1} \). \\
    Tenemos entonces que \(d(x_n, x) > d(x_{n+1}, x) > 0 \Rightarrow \) los términos de la sucesión \({(x_n)}_{n \in N} \) son todos diferentes de \(x\), distintos dos a dos y pertenecen a \(B_r(x)\). \\
    \(\therefore {(x_n)}_{n \in \N} \subset N \cap X\) es infinito.
  \end{proof}
\end{prop}

\begin{prop}
  \(X \subset \R^n, {(X^{\prime})}^{\prime} \subset X^{\prime} \).
  \begin{proof}
    \(x \in {(X^{\prime})}^{\prime} \) y \(N\) un entorno abierto de \(x\). Sabemos que \(N\) contiene un punto \(y \in X^{\prime} \) distinto de \(x\). Luego contiene infinitos puntos de \(X \therefore x \in X^{\prime} \).
  \end{proof}
\end{prop}

\begin{corollary}
  El conjunto derivado de todo subconjunto de \(\R^n\) es cerrado.
  \begin{proof}
    \(\overline{X^{\prime}} = X^{\prime} \cup {(X^{\prime})}^{\prime} = X^{\prime} \therefore \) es cerrado.
  \end{proof}
\end{corollary}

\begin{definition}[Conjunto perfecto]
  \(X \subset \R^n\) es un conjunto perfecto \(\iff X = X^{\prime} \).
\end{definition}

\begin{definition}[Punto aislado]
  \(X \subset \R^n, x \in X\) es aislado en \(X\) si \(\exists N\) un entorno de \(x : N \cap X = \{x\} \).
\end{definition}

\begin{definition}[Conjunto discreto]
  Un conjunto es discreto \(\iff \) todos sus puntos son aislados.
\end{definition}

\begin{eg}
  \(\Z \subset \R \) es discreto.
\end{eg}

\section{Sucesiones en varias dimensiones}

Decimos que una sucesión \({(x_n)}_{n \in \N} \subset \R^n\) converge a un punto \(L \in \R^n\) si \begin{align*}
  (\forall \e > 0)(\exists n_0 \in \N) : (\forall n > n_0) \quad d(x_n, L) < \e.
\end{align*}

\begin{lemma}
  Si una sucesión tiene límite en \(\R^n\) es único.
  \begin{proof}
    Supongamos que converge a \(L\) y a \(L^{\prime} \) con \(L \neq L^{\prime} \Rightarrow \) \begin{align*}
       & (\forall \e > 0)(\exists n_0 \in \N) : (\forall n > n_0), d(x_n, L) < \e                                      \\
       & (\forall \e > 0)(\exists n_1 \in \N) : (\forall n > n_1), d(x_n, L^{\prime}) < \e                             \\
       & d(L, L^{\prime}) \leq d(L, x_n) + d(x_n, L^{\prime}) \leq 2 \cdot \e \quad \forall n > \max(n_0\text{, } n_1)
    \end{align*}
    Sea \(\e = d(L, L^{\prime})/2 \Rightarrow d(L, L^{\prime}) < d(L, L^{\prime})\) Absurdo!
  \end{proof}
\end{lemma}

\begin{prop}
  Sea \({(x_n)}_{n \in \N} \subset \R^n\) y \(L \in \R^n\) son equivalentes:

  \begin{enumerate}
    \item \(L = \lim_{n \to \infty} x_n\)
    \item \(\forall r > 0, \exists n_0 \in \N : \forall n > n_0 \quad x_n \in B_r(L)\)
    \item \(\forall \) entorno abierto \(N\) de \(L, \exists n_0 \in \N : \forall n > n_0 \quad x_n \in N\)
    \item \(\forall \) entorno \(N\) de \(L, \exists n_0 \in \N : \forall n > n_0 \quad x_n \in N\)
  \end{enumerate}
  \begin{proof}
    Sean \({(x_n)}_{n \in \N} \subset \R^n\) y \(L \in \R^n\) \begin{enumerate}
      \item[(4) \(\Rightarrow \) (3) \(\Rightarrow \) (2) \(\Rightarrow \) (1)] Inmediato de la definición de límite.
      \item [(1) \(\Rightarrow \) (4)] Sea \(N\) un entorno de \(L\). Como \(L \in N^{\circ}, \exists r > 0 : B_r(L) \subset N^{\circ} \) y como \(\lim_{n \to \infty} x_n = L \Rightarrow \exists n_0 \in \N : \forall n > n_0, d(L, x_n) < r \Rightarrow \) \\
            \(x_n \in B_r(x), \forall n > n_0 \therefore x_n \in N, \forall n > n_0\).
    \end{enumerate}
  \end{proof}
\end{prop}

\begin{prop}
  Ejercicio: Demostrar las siguientes propiedades en \(\R^n\).
  \begin{enumerate}
    \item Toda sucesión convergente es acotada.
    \item Suma de sucesiones convergentes es convergente y el límite de la suma es la suma de los límites.
    \item Si \({(x_n)}_{n \in \N} \subset \R^n\) y \({(\lambda_n)}_{n \in \N} \subset \R \) convergen \(\Rightarrow {(x_n \cdot \lambda_n)}_{n \in \N} \to \lim_{n \to \infty} x_n \cdot \lim_{n \to \infty} \lambda_n\).
  \end{enumerate}
\end{prop}

\begin{prop}
  Sea \({(x_n)}_{n \in \mathbb{N}} \subset \mathbb{R}^n\), donde cada \(x_n = (x_{n_1}, x_{n_2}, \ldots, x_{n_n}) \in \mathbb{R}^n\). Son equivalentes:
  \begin{enumerate}
    \item La sucesión \((x_n)\) converge en \(\mathbb{R}^n\).
    \item Para todo \(i \in \{1, \ldots, n\} \), la sucesión \({(x_{n_i})}_{n \in \mathbb{N}} \) converge en \(\mathbb{R} \).
  \end{enumerate}
  Además, si se cumplen y:
  \[
    \lim_{n \to \infty} x_n = L \quad \text{y} \quad \lim_{n \to \infty} x_{n_i} = L_i \quad \forall i \in \{1, \ldots, n\},
  \]
  entonces:
  \[
    L = (L_1, \ldots, L_n).
  \]
  \begin{proof}
    Sea \({(x_n)}_{n \in \mathbb{N}} \subset \mathbb{R}^n\) y \(L = (L_1, \ldots, L_n) \in \mathbb{R}^n\).
    \begin{enumerate}
      \item[\((1) \Rightarrow (2)\):] Supongamos que \(\lim_{n \to \infty} x_n = L\). Sea \(\varepsilon > 0\). Entonces, existe \(n_0 \in \mathbb{N} \) tal que para todo \(n > n_0\), se tiene que:
            \[
              \|x_n - L\| < \varepsilon.
            \]
            En particular, esto implica que para cada \(i \in \{1, \ldots, n\} \):
            \[
              |x_{n_i} - L_i| < \varepsilon,
            \]
            lo que demuestra que \(\lim_{n \to \infty} x_{n_i} = L_i\).

      \item[\((2) \Rightarrow (1)\):] Supongamos que para cada \(i \in \{1, \ldots, n\} \), \(\lim_{n \to \infty} x_{n_i} = L_i\). Sea \(\varepsilon > 0\). Como cada coordenada converge, existe \(n_i \in \mathbb{N} \) tal que:
            \[
              |x_{n_i} - L_i| < \frac{\varepsilon}{\sqrt{n}} \quad \text{para todo } n > n_i.
            \]
            Sea \(n_0 = \max \{n_1, \ldots, n_n\} \). Entonces, para todo \(n > n_0\),
            \[
              \|x_n - L\| = {\left( \sum_{i=1}^n |x_{n_i} - L_i|^2 \right)}^{1/2} < {\left( n \cdot \left( \frac{\varepsilon^2}{n} \right) \right)}^{1/2} = \varepsilon.
            \]
            Por lo tanto, \(\lim_{n \to \infty} x_n = L\).
    \end{enumerate}
  \end{proof}
\end{prop}

\clearpage
\section{Conjuntos compactos}

\begin{theorem}
  Un subconjunto \(F \subset \mathbb{R}^n\) es cerrado si y sólo si:
  \(\text{Si } {(x_n)}_{n \in \mathbb{N}} \subset F \text{ y } x_n \to x, \text{ entonces } x \in F\).
  \begin{proof}
    \((\Rightarrow)\) Supongamos que \(F \subset \mathbb{R}^n\) es cerrado.
    Sean \({(x_n)}_{n \in \mathbb{N}} \subset F\) y \(x_n \to x\).
    Supongamos por el absurdo que \(x \notin F\). Entonces \(x \in F^c\).
    Como \(F^c\) es abierto, existe \(r > 0\) tal que \(B_r(x) \subset F^c\).
    Pero como \(x_n \to x\), existe \(n_0 \in \mathbb{N} \) tal que
    \(x_n \in B_r(x) \subset F^c \quad \text{para todo } n > n_0\).
    Esto contradice el hecho de que \(x_n \in F\) para todo \(n\).
    Por lo tanto, \(x \in F\) y se cumple la implicación.

    \medskip

    \((\Leftarrow)\) Supongamos ahora que la propiedad se cumple, y veamos que \(F\) es cerrado.
    Supongamos por el contrario que \(F\) no es cerrado. Entonces existe \(x \in \overline{F} \setminus F\),
    es decir, \(x\) es punto de adherencia de \(F\) pero \(x \notin F\).
    Por ser punto de adherencia, para todo \(n \in \mathbb{N} \) existe \(x_n \in F\) tal que
    \( \|x_n - x\| < \frac{1}{n}\).
    Entonces, \({(x_n)}_{n \in \mathbb{N}} \subset F\) y \(x_n \to x\), pero \(x \notin F\),
    lo que contradice la hipótesis. Por lo tanto, \(F\) es cerrado.
  \end{proof}
\end{theorem}

\begin{eg}
  \( \interval[open left]{0}{1}\) no es cerrado pues \(\dfrac{1}{n} \to 0\) y \(\dfrac{1}{n} \in \interval[open left]{0}{1}, \forall n \in \N \).
\end{eg}

\begin{definition}[Relativamente compacto]
  Un subconjunto \(F \subset \R^n\) se dice relativamente compacto si toda sucesión en \(F\) posee una subsucesión convergente.
\end{definition}

\begin{definition}[Compacto]
  Un conjunto se dice compacto si además de ser relativamente compacto cumple que está acotado.
\end{definition}

\clearpage

\begin{prop}
  Un subconjunto \(F \subset \mathbb{R}^n\) es relativamente compacto si y sólo si es acotado.
  \begin{proof}
    \((\Rightarrow)\) Supongamos que \(F\) no es acotado. Entonces,
    \[
      \forall n \in \mathbb{N},\ \exists y_n \in F : \|y_n\| > n.
    \]
    Si \(f : \mathbb{N} \to \mathbb{N} \) es una función estrictamente creciente, entonces la subsucesión \({(y_{f(n)})}_{n \in \mathbb{N}} \) satisface
    \[
      \|y_{f(n)}\| > f(n) \geq n \quad \Rightarrow \quad (y_{f(n)}) \text{ no es acotada, y por tanto no converge}.
    \]
    Luego, ninguna subsucesión de \((y_n)\) converge, y por tanto \(F\) no es relativamente compacto.

    \medskip

    \((\Leftarrow)\) Supongamos ahora que \(F\) es acotado. Entonces existe \(k > 0\) tal que \(F \subset B_k(0)\).
    Sea \({(x_m)}_{m \in \mathbb{N}} \subset F\) una sucesión cualquiera.
    Escribamos cada \(x_m = (x_{m}^{(1)}, x_{m}^{(2)}, \dots, x_{m}^{(n)}) \in \mathbb{R}^n\).

    Como \( \|x_m\| < k\) para todo \(m\), se sigue que cada sucesión de coordenadas \({(x_m^{(i)})}_{m \in \mathbb{N}} \) está acotada en \(\mathbb{R} \).
    Por el teorema de Bolzano-Weierstrass, toda sucesión acotada en \(\mathbb{R} \) tiene una subsucesión convergente.

    Aplicamos el argumento diagonal:
    Primero, extraemos una subsucesión \({(x_{f_1(m)})}_{m \in \mathbb{N}} \) tal que \((x_{f_1(m)}^{(1)})\) converge.
    Luego, de esta subsucesión extraemos otra \({(x_{f_2(m)})}_{m \in \mathbb{N}} \) tal que \((x_{f_2(m)}^{(2)})\) converge.
    Repetimos el proceso hasta obtener una subsucesión \({(x_{f_n(m)})}_{m \in \mathbb{N}} \) tal que todas las coordenadas \((x_{f_n(m)}^{(i)})\) convergen para \(i = 1, \dots, n\).

    Definimos \(h := f_n \circ \cdots \circ f_1\), que es una función estrictamente creciente, y la sucesión \({(x_{h(m)})}_{m \in \mathbb{N}} \) es una subsucesión de \((x_m)\) cuyas coordenadas convergen.

    Entonces, \((x_{h(m)})\) converge en \(\mathbb{R}^n\).
    Por tanto, toda sucesión en \(F\) tiene una subsucesión convergente en \(\mathbb{R}^n\), es decir, \(F\) es relativamente compacto.
  \end{proof}
\end{prop}

\begin{corollary}
  \(F \subset \R^n\) es compacto \(\iff \) es cerrado y acotado.
\end{corollary}

Esta última parte pertenece a la clase siguiente, pero se coloca aquí para que quede la sección de compactos toda junta.

\begin{definition}[Cubrimiento]
  Sea \(X \subset R^n\). Una familia \(U\) de subconjuntos de \(\R^n\) es un cubrimiento de \(X\) si \(X \subset \bigcup_{u \in U} u\) y es un cubrimiento abierto si todo subconjunto de \(U\) es abierto.
\end{definition}

\begin{definition}[Subcubrimiento]
  Un subcubrimiento de \(U\) es un cubrimiento \(V\) de \(X\) tal que \(V \subset U\).
\end{definition}

\begin{theorem}[Lindelöf]
  Todo cubrimiento abierto de algún subconjunto de \(\R^n\) admite a lo sumo un subcubrimiento a lo sumo numerable.
  \begin{proof}
    Sea \(X \subset \R^n\) y \(U\) un cubrimiento abierto de \(X\).
    \begin{itemize}
      \item Si \(X = \varnothing \Rightarrow \varnothing \) es un subcubrimiento de \(U\) a lo
            sumo numerable.
      \item Si \(X \neq \varnothing \Rightarrow U \neq \varnothing \Rightarrow \) Podemos
            fijar \(U_0 \in U\).
    \end{itemize}
    Para cada \(q \in \Q^n\) y sea \(s \in \Q > 0\). Elegimos un elemento \(U(q, s)\) de \(U\) de la siguiente forma:
    \begin{enumerate}
      \item Si hay elementos de \(U\) que contienen a \(B_s(q)\) elegimos cualquiera de ellos y
            lo llamamos \(u(q, s)\).
      \item Si no hay ninguno ponemos \(U(q, s) = U_0\).
    \end{enumerate}
    Sea \(x \in X\) como \(U\) es un cubrimiento abierto de \(X\), \(\exists u \in U\) abierto tal que \(x \in u \Rightarrow \exists r > 0 : B_r(x) \subset u\). \\
    Si tomamos \(q \in \Q^n : q \in B_{\frac{r}{2}}(x)\) y \(s \in Q : d(x, q) < s < \dfrac{r}{2} \Rightarrow x \in B_s(q)\) y \(B_s(q) \subset B_r(x) \subset u\). \\
    Así que hay elementos de \(U\) que contienen a \(B_s(q)\) y por lo tanto \(x \in B_s(q) \subset U(q, s)\). Esto prueba que \(V = \{ U(q, s) : q \in \Q^n, s \in Q > 0 \} \) es un subcubrimiento de \(U\) y es numerable pues \(\Q^n \times \Q^n\) es numerable y la función \(Q^n \times \Q > 0 \to V\), \((q, s) \mapsto U(q, s)\), es suryectiva \(\therefore V\) es a lo sumo numerable.
  \end{proof}
\end{theorem}

\begin{corollary}
  Un subconjunto \(F \subset \R^n\) es compacto \(\iff \forall \) cubrimiento abierto de \(F\) admite un subcubrimiento finito.
  \begin{proof}
    Para la ida sea \(F \subset \R^n\) compacto, \(U\) un cubrimiento abierto de \(F\). Por Lindelöf hay un subcubrimiento a lo sumo numerable \(V\) de \(U\) y \(\exists f: \N \to V\) suryectiva.
    Por el absurdo supongamos que \(U\) no contiene subcubrimiento finito de \(F\).
    En particular si \(m \in \N, V_m = \{ f(1), \ldots, f(m) \} \in V\) no es subcubrimiento de \(F\), por ser finito
    \begin{align*}
      \Rightarrow \exists x_m \in F - \bigcup_{i = 1}^m f(i)
    \end{align*} 
    De esta forma tenemos una sucesión de elementos de \(F, {(x_m)}_{m \in \N} \). \\
    Como \(F\) es compacto \(\exists h: \N \to \N \) estrictamente creciente tal que \({(x_{h(m)})}_{m \in \N} \to L \in F\). \\
    Como \(V\) es un cubrimiento de \(F\) y \(f\) es sobreyectiva \(\Rightarrow \exists s \in \N : L \in f(s)\).
    Por otro lado como \(\lim_{n \to \infty} x_m = L\) y \(f(s)\) es un entorno de \(L, \exists m_0 \in \N : x_{h(m)} \in f(s), \forall m > m_0\). \\
    Como \(m < h(s + m_0) \Rightarrow x_{h(s+m_0)} \in f(s)\) Absurdo! \\
    \(x_{h(m_0 + s)} \in F - \bigcup_{i=1}^{h(s+m_0)} f(i)\) que es disjunto con \(f(s)\) pues \(s < h(s + m_0)\). \\
    \(\therefore V\) y, por lo tanto, \(U\) contienen un subcubrimiento finito. \\

    Para la vuelta supongamos que todo cubrimiento abierto de \(F\) admite un
    subcubrimiento finito, pero que \(F\) no es compacto. \\ Hay una sucesión
    \({(x_m)}_{m \in \N} \subset F\) que no tiene ninguna subsucesión convergente. Si
    \(x \in F\), no hay ningua subsucesión de \({(x_m)}_{m \in \N} \) que converge a \(x\)
    y, por lo tanto, \(\exists r_x > 0 : S_x = \{ m \in \N : x_m \in B_r(x) \} \) es
    finito. \\ El conjunto \(U = \{ B_{r_x}(x) : x \in F \} \) es un cubrimiento
    abierto de \(F\). \\ Por hipotesis hay un subconjunto finito \( \{ y_1, \ldots,
    y_k\} \) tal que \( \{ B_{r_{y_1}}(y_1), \ldots, B_{r_{y_k}}(y_k) \} \) es
    subcubrimiento de \(F\). \\ En particular \(\forall m \in \N, \exists i \in \{ 1,
    \cdots, k \} : x_m \in B_{r_{y_i}}(y_i) \Rightarrow \) \\ \(m \in S_y \therefore
    \N \subset S_{y_1} \cup \cdots \cup S_{y_k} \) Absurdo! pues los \(S_{y_i} \) son
    finitos.
  \end{proof}
\end{corollary}

\begin{prop}
  Sea \(K \subset \R^n\) compacto. Si \(U\) es un cubrimiento abierto de \(K\) entonces \(\exists \delta > 0 \): si \(x, y \in K, d(x, y) < \delta \Rightarrow \exists u \in U\) abierto tal que \(x, y \in u\). \\
  \(\delta \) se llama número de Lebesgue del cubrimiento \(U\).
  \begin{proof}
    Sea \(U\) un cubrimiento abierto de \(K\). Para cada \(x \in K, \exists U_x \in U : x \in U_x\). Como \(U_x\) es abierto \(\exists r_x > 0 : B_{r_k}(x) \subset U_x\). \\
    El conjunto \(U^{\prime} = \{ B_{\frac{r_x}{2}} : x \in K \} \) es un cubrimiento abierto de \(K\). \(U^{\prime} \) posee un subcubrimiento finito. \\
    \(\exists x_1, \ldots, x_k : K \subset \bigcup_{i = 1}^{k} B_{\frac{r_{x_i}}{2}}(x_i)\). \\
    Veamos que \(\delta = \dfrac{1}{2} \cdot \min(r_{x_1}, \ldots, r_{x_k})\) es un número de Lebesgue: \\
    Sean \(x, y \in K : d(x,y) < \delta \Rightarrow \) \\
    \(\exists i \in \{1, \ldots, m\} : x \in B_{\frac{r_{x_i}}{2}}(x_i) \Rightarrow \|x_i - y\| \leq \|x_i - x\| + \|x+y\| \leq \dfrac{r_{x_i}}{2} + \delta < \dfrac{r_{x_i}}{2} + \dfrac{r_{x_i}}{2} = r_{x_i} \Rightarrow \) \\
    \(x, y \in B_{r_{x_i}}(x_i) \subset U_{x_i} \) que es abierto de \(U\).
  \end{proof}
\end{prop}