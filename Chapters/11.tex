\section{(Continuación) Conjuntos Compactos}

\begin{corollary}
  $F \subset \R^n$ es compacto $\iff$ es cerrado y acotado.
\end{corollary}

\begin{definition}[Cubrimiento]
  Sea $X \subset R^n$. Una familia $U$ de subconjuntos de $\R^n$ es un cubrimiento de $X$ si $X \subset \bigcup_{u \in U} u$ y es un cubrimiento abierto si todo subconjunto de $U$ es abierto.
\end{definition}

\begin{definition}[Subcubrimiento]
  Un subcubrimiento de $U$ es un cubrimiento $V$ de $X$ tal que $V \subset U$.
\end{definition}

\begin{theorem}[Lindelöf]
  Todo cubrimiento abierto de algún subconjunto de $\R^n$ admite a lo sumo un subcubrimiento a lo sumo numerable.
  \begin{proof}
    Sea $X \subset \R^n$ y $U$ un cubrimiento abierto de $X$.
    \begin{itemize}
      \item Si $X = \varnothing \Rightarrow \varnothing$ es un subcubrimiento de $U$ a lo
            sumo numerable.
      \item Si $X \neq \varnothing \Rightarrow U \neq \varnothing \Rightarrow$ Podemos
            fijar $U_0 \in U$.
    \end{itemize}
    Para cada $q \in \Q^n$ y sea $s \in \Q > 0$. Elegimos un elemento $U(q, s)$ de $U$ de la siguiente forma:
    \begin{enumerate}
      \item Si hay elementos de $U$ que contienen a $B_s(q)$ elegimos cualquiera de ellos y
            lo llamamos $u(q, s)$.
      \item Si no hay ninguno ponemos $U(q, s) = U_0$.
    \end{enumerate}
    Sea $x \in X$ como $U$ es un cubrimiento abierto de $X$, $\exists u \in U$ abierto tal que $x \in u \Rightarrow \exists r > 0 : B_r(x) \subset u$. \\
    Si tomamos $q \in \Q^n : q \in B_{\frac{r}{2}}(x)$ y $s \in Q : d(x, q) < s < \dfrac{r}{2} \Rightarrow x \in B_s(q)$ y $B_s(q) \subset B_r(x) \subset u$. \\
    Así que hay elementos de $U$ que contienen a $B_s(q)$ y por lo tanto $x \in B_s(q) \subset U(q, s)$. Esto prueba que $V = \{ U(q, s) : q \in \Q^n, s \in Q > 0 \}$ es un subcubrimiento de $U$ y es numerable pues $\Q^n \times \Q^n$ es numerable y la función $Q^n \times \Q > 0 \to V$, $(q, s) \mapsto U(q, s)$, es suryectiva $\therefore V$ es a lo sumo numerable.
  \end{proof}
\end{theorem}

\begin{corollary}
  Un subconjunto $F \subset \R^n$ es compacto $\iff \forall$ cubrimiento abierto de $F$ admite un subcubrimiento finito.
  \begin{proof}
    Para la ida sea $F \subset \R^n$ compacto, $U$ un cubrimiento abierto de $F$. Por Lindelöf hay un subcubrimiento a lo sumo numerable $V$ de $U$ y $\exists f: \N \to V$ suryectiva.
    Por el absurdo supongamos que $U$ no contiene subcubrimiento finito de $F$.
    En particular si $m \in \N, V_m = \{ f(1), \cdots, f(m) \} \in V$ no es subcubrimiento de $F$, por ser finito
    \begin{equation}
      \Rightarrow \exists x_m \in F - \bigcup_{i = 1}^m f(i)
    \end{equation} De esta forma tenemos una sucesión de elementos de $F, (x_m)_{m \in \N}$. \\
    Como $F$ es compacto $\exists h: \N \to \N$ estrictamente creciente tal que $(x_{h(m)})_{m \in \N} \to L \in F$. \\
    Como $V$ es un cubrimiento de $F$ y $f$ es sobreyectiva $\Rightarrow \exists s \in \N : L \in f(s)$.
    Por otro lado como $lim_{n \to \infty} x_m = L$ y $f(s)$ es un entorno de $L, \exists m_0 \in \N : x_{h(m)} \in f(s), \forall m > m_0$. \\
    Como $m < h(s + m_0) \Rightarrow x_{h(s+m_0)} \in f(s)$ Absurdo! \\
    $x_{h(m_0 + s)} \in F - \bigcup_{i=1}^{h(s+m_0)} f(i)$ que es disjunto con $f(s)$ pues $s < h(s + m_0)$. \\
    $\therefore V$ y, por lo tanto, $U$ contienen un subcubrimiento finito. \\

    Para la vuelta supongamos que todo cubrimiento abierto de $F$ admite un
    subcubrimiento finito, pero que $F$ no es compacto. \\ Hay una sucesión
    $(x_m)_{m \in \N} \subset F$ que no tiene ninguna subsucesión convergente. Si
    $x \in F$, no hay ningua subsucesión de $(x_m)_{m \in \N}$ que converge a $x$
    y, por lo tanto, $\exists r_x > 0 : S_x = \{ m \in \N : x_m \in B_r(x) \}$ es
    finito. \\ El conjunto $U = \{ B_{r_x}(x) : x \in F \}$ es un cubrimiento
    abierto de $F$. \\ Por hipotesis hay un subconjunto finito $\{ y_1, \cdots,
      y_k\}$ tal que $\{ B_{r_{y_1}}(y_1), \cdots, B_{r_{y_k}}(y_k) \}$ es
    subcubrimiento de $F$. \\ En particular $\forall m \in \N, \exists i \in \{ 1,
      \cdots, k \} : x_m \in B_{r_{y_i}}(y_i) \Rightarrow$ \\ $m \in S_y \therefore
      \N \subset S_{y_1} \cup \cdots \cup S_{y_k}$ Absurdo! pues los $S_{y_i}$ son
    finitos.
  \end{proof}
\end{corollary}

\begin{prop}
  Sea $K \subset \R^n$ compacto. Si $U$ es un cubrimiento abierto de $K$ entonces $\exists \delta > 0 :$ si $x, y \in K, d(x, y) < \delta \Rightarrow \exists u \in U$ abierto tal que $x, y \in u$. \\
  $\delta$ se llama número de Lebesgue del cubrimiento $U$.
  \begin{proof}
    Sea $U$ un cubrimiento abierto de $K$. Para cada $x \in K, \exists U_x \in U : x \in U_x$. Como $U_x$ es abierto $\exists r_x > 0 : B_{r_k}(x) \subset U_x$. \\
    El conjunto $U^{\prime} = \{ B_{\frac{r_x}{2}} : x \in K \}$ es un cubrimiento abierto de $K$. $U^{\prime}$ posee un subcubrimiento finito. \\
    $\exists x_1, \cdots, x_k : K \subset \bigcup_{i = 1}^{k} B_{\frac{r_{x_i}}{2}}(x_i)$. \\
    Veamos que $\delta = \dfrac{1}{2} \cdot min(r_{x_1}, \cdots, r_{x_k})$ es un número de Lebesgue: \\
    Sean $x, y \in K : d(x,y) < \delta \Rightarrow$ \\
    $\exists i \in \{1, \cdots, m\} : x \in B_{\frac{r_{x_i}}{2}}(x_i) \Rightarrow \|x_i - y\| \leq \|x_i - x\| + \|x+y\| \leq \dfrac{r_{x_i}}{2} + \delta < \dfrac{r_{x_i}}{2} + \dfrac{r_{x_i}}{2} = r_{x_i} \Rightarrow$ \\
    $x, y \in B_{r_{x_i}}(x_i) \subset U_{x_i}$ que es abierto de $U$.
  \end{proof}
\end{prop}

\section{Propiedades de límites}

\begin{definition}[Límite]
  $A \subset \R^n, x_0$ un punto de acumulación de A y $f: A \to \R^n$ una función, decimos que $y \in R^n$ es límite de $f$ en $x_0$ si: \\
  $(\forall \e > 0)(\exists \delta>0) : (\forall x \in A)$ si $0 < \|x - x_0\| < \delta \Rightarrow \|f(x) - y\| < \e$.
\end{definition}

\begin{lemma}
  Si $A \subset \R^n$ y $x_0 \in A^{\prime}$, $f: A \to \R^k$ tiene como muhco un único límite en $x_0$.
  \begin{proof}
    Sea $y, y^{\prime} \in \R^k : $ son límite de $f$ en $x_0 \in A^{\prime}$. Tomemos $\e = \dfrac{1}{2} \cdot \|y - y^{\prime}\| > 0$. \\
    $\Rightarrow \exists \delta_1, \delta_2 : 0 < \| x - x_0 \| < \delta_1 \Rightarrow \|f(x) - y \| < \e$ \\
    $0 < \|x - x_0\| < \delta_2 \Rightarrow \|f(x) - y^{\prime}\| < \e$. \\
    Si $\delta = min(\delta_1, \delta_2) \Rightarrow$ Si $0 < \|x - x_0\| < \delta \Rightarrow$ \\
    $\|f(x) - y\| + \|f(x) - y^{\prime}\| < 2 \cdot \e$ \\
    $\|y - y^{\prime}\| < 2 \cdot \e = \|y - y^{\prime}$ Absurdo! \\
    $\therefore$ el límite es único.
  \end{proof}
\end{lemma}

\begin{prop}
  Sea $A \subset \R^n, x_0 \in A^{\prime}, f: A \to \R^k, y \in \R^k$ son equivalentes:
  \begin{enumerate}
    \item $lim_{x \to x_0} f(x) = y$
    \item $(\forall \e > 0)(\exists \delta > 0) : f(B_{\delta}(x_0)) \cap (A - \{x_0\}) \subset B_{\e}(y)$
    \item $\forall$ entorno $V$ de $y$, $\exists$ un entorno $U$ de $x_0$ tal que $f(U \cap (A - \{x_0\})) \subset V$.
  \end{enumerate}

  \begin{proof}
    1) $\Rightarrow$ 2) Dado \begin{equation}
      \e > 0, \exists \delta > 0 : \|x - x_0 \| < \delta \Rightarrow \|f(x) - y\| < \e
    \end{equation} Si \begin{equation} x \in B_{\delta}(x_0) \cap (A - \{x_0\}) \Rightarrow x \in A, \|x - x_0\| < \delta \Rightarrow
    \end{equation}
    \begin{equation}
      \|f(x) - y\| < \e \Rightarrow f(x) \in B_{\e}(y) \therefore f(B_{\delta}(x_0)) \cap (A - \{x_0\}) \subset B_{\e}(y)
    \end{equation}

    2) $\Rightarrow$ 3) Si $V$ es un entorno de $y \Rightarrow \exists \e > 0 : B_{\e}(y) \subset V$. Por hipótesis \begin{equation}
      \exists \delta > 0 : f(B_{\delta}(x_0) \cap (A - \{x_0\})) \subset B_{\e}(y)
    \end{equation} Tomo $U = B_{\delta}(x_0)$ que es un entorno de $x$.

    3) $\Rightarrow$ 1) Sea $\e > 0$. Como $B_{\e}(y)$ es un entorno de $y$. Por hipótesis $\exists$ un entorno $U$ de $x_0 : f(U \cap (A - \{x_0\})) \subset B_{\e}(y)$. Como $U$ es entorno de $x_0, \exists \delta > 0 : B_{\delta}(x_0) \subset U$. \\
    Si $x \in A$ es tal que $0 < \|x - x_0\| < \delta \Rightarrow x \in B_{\delta}(x_0) \cap (A - \{x_0\}) \Rightarrow f(x) \in B_{\e}(y) \Rightarrow \|f(x) - y\| < \e$ \\
    $\therefore lim_{x \to x_0} f(x) = y$.
  \end{proof}
\end{prop}

\begin{prop}
  $A \subset \R^n, x_0 \in A^{\prime}, f: A \to \R^k, \lim_{x \to x_0} f(x) = y \iff \forall (x_n)_{n \in \N} \subset A - \{x_0\}, x_n \to x_0, (f(x_n))_{n \in \N}$ converge a $y$.
  \begin{proof}
    $\Rightarrow$) Supongamos que $lim_{x \to x_0} f(x) = y$. Sea $(x_n)_{n \in \N} \subset A - \{x_0\}, x_n \to x_0$. Dado \begin{equation}
      \e > 0, \exists \delta > 0 : d(x, x_0) < \delta \Rightarrow d(f(x), y) < \e
    \end{equation} Además \begin{equation}
      \exists n_0 \in \N : d(x_n, x_0) < \delta, \forall n > n_0
    \end{equation} \begin{equation} \Rightarrow d(f(x_n), y) < \e, \forall n > n_0 \therefore (f(x_n))_{n \in \N} \to y \end{equation}

    $\Leftarrow$) $\forall (x_n)_{n \in \N} \subset A - \{x_0\}, x_n \to x_0, (f(x_n))_{n \in \N} \to y$. \\
    Supongamos que $lim_{x \to x_0} f(x) \neq y$. Luego $\exists \e > 0 : \forall \delta > 0, d(x, x_0) < \delta$, pero $d(f(x), y) \geq \e$. \\
    En particular consideremos $(x_n)_{n \in \N} \subset A - \{x_0\}$ y podemos encontrar $d(x_n, x_0) < \delta, \forall n > n_0 \in \N$ pues $x_n \to x_0$. \\
    Luego $d(f(x_n), y) \geq \e, \forall n > n_0 \therefore (f(x_n))_{n \in \N} \not \to y$.
  \end{proof}
\end{prop}

\begin{prop}
  Sea $A \subset \R^n, x_0 \in A^{\prime}, f, g: A \to \R^k, h: A \to \R$.
  \begin{enumerate}
    \item $lim_{x \to x_0} f(x) = y, lim_{x \to x_0} g(x) = z \Rightarrow$ \\ $lim_{x \to x_0} f(x) + g(x) = lim_{x \to x_0} f(x) + lim_{x \to x_0} g(x) = y+z$.
    \item $lim_{x \to x_0} h(x) = r \Rightarrow lim_{x \to x_0} h(x) \cdot f(x) = r \cdot y$
    \item $lim_{x \to x_0} h(x) = r \neq 0, h(x) \neq 0 (\forall x \in A) \Rightarrow lim_{x \to x_0} \dfrac{1}{h(x)} = \dfrac{1}{lim_{x \to x_0} h(x)} = \dfrac{1}{r}$.
  \end{enumerate}
\end{prop}

\begin{prop}
  Sea $A \subset \R^n, x_0 \in A^{\prime}, f: A \to \R^k, lim_{x \to x_0} f(x) = y$.
  \begin{enumerate}
    \item Si $z \neq y, \exists r > 0 : z \in f(B_r(x_0) - (A -\{x_0\}))$.
    \item Si $y > 0 \Rightarrow \exists r > 0 : f(x) > 0, \forall x \in (B_r(x_0) - (A -
            \{x_0\}))$.
    \item $\exists r > 0 : f$ es acotada sobre $B_r(x_0) \cap A$, o sea, $\{f(x) : x \in B_r(x_0) \cap A\}$ está acotado superiormente.
  \end{enumerate}

  \begin{proof}
    \begin{enumerate}
      \item $y \neq z \Rightarrow d(y, z) > 0 \Rightarrow \exists r > 0 : 0 < d(x, x_0) < r \Rightarrow d(f(x), y) < d(y, z)$.
      \item $x \in B_r(x_0) \cap (A - \{x_0\}) \Rightarrow 0 < d(x, x_0) < r \Rightarrow d(f(x), y) < d(z, y) \therefore f(x) \neq z$, o sea, $z \in f(B_r(x_0) \cap (A - \{x_0\}))$.
      \item Sea $\e = 1 \Rightarrow \exists r > 0 : \forall x \in A$ vale que $0 < d(x,
              x_0) < r \Rightarrow d(f(x), y) < 1 \Rightarrow$ \\ Si $x \in B_r(x_0) \cap (A
              - \{x_0\}), \|f(x)\| = \|f(x) + y - y\| \leq \|f(x) - y\| + \|y\| < 1 + \|y\|$.
            \\ $\therefore$ si $x_0 \notin A, 1+\|y\|$ es cota superior para $\{f(x) : x
              \in B_r(x) \cap A\}$. \\ Si $x_0 \in A \Rightarrow 1 + \|y\| + \|f(x_0)\|$ es
            cota superior. La definición del límite no sirve para $x_0$, por eso se separa
            en casos.
    \end{enumerate}
  \end{proof}
\end{prop}

\begin{theorem}[Teorema del Sandwich]
  Sea $A \subset \R^n, x_0 \in A^{\prime}$ y $f, g, h: A \to \R, \forall x \in A, f(x) \leq g(x) \leq h(x)$, si: \\
  $lim_{x \to x_0} f(x) = lim_{x \to x_0} h(x) = y \Rightarrow lim_{x \to x_0} g(x) = y$.
  \begin{proof}
    Dado $\e > 0, \exists \delta_1, \delta_2 > 0 :$ \begin{equation}
      0 < d(x, x_0) < \delta_1 \Rightarrow d(f(x), y) < \e
    \end{equation}
    \begin{equation}
      0 < d(x, x_0) < \delta_2 \Rightarrow d(h(x), y) < \e
    \end{equation} Sea $\delta = min(\delta_1, \delta_2) \Rightarrow \exists x \in A : 0 < d(x, x_0) < \delta \Rightarrow$
    \begin{equation}
      y - \e < f(x) \leq g(x) \leq h(x) < y + \e \Rightarrow d(g(x), y) < \e
    \end{equation}
  \end{proof}
  $\therefore lim_{x \to x_0} g(x) = y$.
\end{theorem}
