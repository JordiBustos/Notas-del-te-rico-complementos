\section{Límites en varias dimensiones}

\begin{prop}
  Sea \(A \subset \R^n, f: A \to \R^k, f=(f_1, \ldots, f_k), y = (y_1, \ldots, y_k) \in \R^k\) son equivalentes:
  \begin{enumerate}
    \item \(\lim_{x \to x_0} f(x) = y\).
    \item \(\forall i \in \{1, \ldots, k\}, \lim_{x \to x_0} f_i(x) = y_i\).
  \end{enumerate}
  \begin{proof}
    \begin{itemize}
      \item Para la ida \((\forall \e > 0)(\exists \delta > 0) : (\forall x \in A), 0 < d(x, x_0) < \delta \Rightarrow d(f(x), y) < \e \). Luego, si \(d(x, x_0) < \delta \Rightarrow d(f_i(x) - y_i) < d(f(x) - y) < \e \therefore \lim_{x \to x_0} f_i(x) = y_i\).
      \item Para la vuelta, sea \(\e > 0, (\forall i \in \{1, \ldots, k\})(\exists \delta_i > 0) : (\forall x \in A) 0 \leq d(x, x_0) < \delta_i\). \\
            \(\Rightarrow d(f_i(x) - y_i) < \dfrac{\e}{\sqrt{k}} \). Si \(\delta = \min(\delta_i, i = 1, \ldots, k)\) y \(0 \leq d(x, x_0) < \delta \).
            \(\Rightarrow d(f(x) - y) = \sqrt{\sum_{i = 1}^k | f_i(x) - y_i |^2 } < \sqrt{\sum_{i = 1}^k \dfrac{\e^2}{k}} = \e \).
    \end{itemize}
  \end{proof}
\end{prop}

\begin{definition}[Límites laterales]
  Sea \(A \subset R, f: A \to R^k\).
  \begin{enumerate}
    \item Si \(x_0\) es un punto de acumulación de \(A_{x_0}^{+} = A \cap \interval[open right]{x_0}{+\infty} \) y la restricción \(f|_{A_{x_0}^{+}} \to R\) tiene límite igual a \(y\) en \(x_0 \Rightarrow \) decimos que y es límite por derecha de \(f\) en \(x_0 \Rightarrow \lim_{x \to x_0^+} f(x) = y\).
    \item Si \(x_0\) es un punto de acumulación de \(A_{x_0}^{-} = A \cap \interval[open left]{-\infty}{x_0} \) y la restricción \(f|_{A_{x_0}^{-}} \to R\) tiene límite igual a \(y\) en \(x_0 \Rightarrow \) decimos que y es límite por izquierda de \(f\) en \(x_0 \Rightarrow \lim_{x \to x_0^-} f(x) = y\).
  \end{enumerate}
\end{definition}

\section{Continuidad en un punto}

\begin{definition}
  Sea \(A \subset \R^n, x_0 \in A\) decimos que \(f: A \to \R^k\) es continua en \(x_0 \iff (\forall \e > 0)(\exists \delta > 0) : \forall x \in A\) vale que \(0 < d(x, x_0) < \delta \Rightarrow d(f(x), f(x_0)) < \e \).
\end{definition}

\begin{lemma}
  Sea \(A \subset \R^n, x_0 \in A \Rightarrow f: A \to \R^k\) es continua en \(x_0 \iff \) o bien \(x_0\) es aislado en \(A\) o bien \(\lim_{x \to x_0} f(x) = f(x_0)\).
  \begin{proof}
    Si \(f\) es continua en \(x_0\) y \(x_0\) no es aislado en \(A\) quiero ver que entonces \(\lim_{x \to x_0} f(x) = f(x_0)\). Como \(\forall x \in A\) vale que \(d(x, x_0) < \delta \Rightarrow d(f(x), f(x_0)) < \e \). Entonces \(0 < d(x, x_0) < \delta \Rightarrow d(f(x), f(x_0)) < \e \). Si \(x_0\) es punto asilado de \(A \Rightarrow \exists \delta>0, B_{\delta}(x_0) \cap A =  \{x_0\} \) y vale que \(\forall x \in A : d(x, x_0) < \delta \Rightarrow d(f(x), f(x_0)) < \e \), pues \(d(f(x_0), f(x_0)) = 0 < \e \). Si \(x_0\) no es punto aislado de \(A\). Como \(\lim_{x \to x_0} f(x) = f(x_0)\), dado \(\e > 0, \exists \delta > 0 : \forall x \in A, 0 < d(x, x_0) < \delta \Rightarrow d(f(x), f(x_0)) < \e \) y si \(x = x_0\) claramente \(d(f(x), f(x_0)) = 0 < \e \).
  \end{proof}
\end{lemma}

\begin{definition}[Entorno relativo]
  Dado \(A \subset \R^n\), decimos que \(V\) es un entorno de \(x\) relativo a \(A\) si \(\exists N \subset \R^n\) entorno de \(x : V = A \cap N\).
\end{definition}

\begin{definition}[Abierto relativo]
  Dado \(A \subset \R^n\) decimos que \(U\) es abierto relativo a \(A\) si \(\exists \tilde{U} \subset \R^n\) abierto tal que \(U = \tilde{U} \cap A\).
\end{definition}

\begin{definition}[Cerrado relativo]
  Dado \(A \subset \R^n\) decimos que \(F\) es un cerrado relativo a \(A\) si \(\exists \tilde{F} \subset \R^n\) cerrado tal que \(F = \tilde{F} \cap A\).
\end{definition}

\begin{prop}
  Sea \(A \subset \R^n, x_0 \in A, f: A \to \R^k\) son equivalentes:
  \begin{enumerate}
    \item \(f\) es continua en \(x_0\).
    \item Para todo entorno \(V\) de \(x_0, \exists \) un entorno \(U\) de \(x_0\) relativo a \(A\) tal que \(f(U) \subset V\).
    \item Para todo entorno \(V\) de \(f(x_0)\), el conjunto \(f^{-1}(V)\) es un entorno de \(x_0\) relativo a \(A\).
  \end{enumerate}

  \begin{proof}
    (a) \(\Rightarrow \) (b) Sea \(V\) un entorno de \(f(x_0) \Rightarrow (\exists \e > 0) : B_{\e}(f(x_0)) \subset V\). Como \(f\) es continua en \(x_0, (\exists \delta > 0) : (\forall x \in A)\) vale que \(d(x, x_0) < \delta \Rightarrow d(f(x), f(x_0)) < \e \). Si \(U = A \cap B_{\delta}(x_0)\) que es un entorno de \(x_0\) relativo de \(A\) y si \(x \in U \Rightarrow f(x) \in B_{\e}(f(x_0)) \subset V \therefore f(U) \subset V\). \\
    (b) \(\Rightarrow \) (c) Sea \(V\) un entorno de \(f(x_0)\), por hipótesis \(\exists U\) entorno de \(x_0\) relativo a \(A\) tal que \(f(U) \subset V \Rightarrow U \subset f^{-1}(V)\). Luego \(f^{-1}(V)\) es un entorno de \(x_0\) relativo a \(A\). \\
    (c) \(\Rightarrow \) (a) Sea \(\e > 0\) como \(B_{\e}(f(x_0))\) es un entorno de \(f(x_0)\). Por hipótesis \(\exists U\) entorno de \(x_0\) relativo a \(A : f(U) \subset B_{\e}(f(x_0))\). Es decir que \(\exists U_0 \subset \R^n\) entorno de \(x_0 : U = A \cap U_0\). Si \(\delta > 0\) es tal que \(B_{\delta}(x_0) \subset U_0\) y \(x \in A\) cumple que \(d(x, x_0) < \delta \Rightarrow x \in B_{\delta}(x_0) \cap A \subset U_0 \cap A = U \therefore f(x) \in f(U) \subset B_{\e}(f(x_0))  \Rightarrow d(f(x), f(x_0)) < \e \).
  \end{proof}
\end{prop}

\begin{prop}
  Sea \(A \subset \R^n\), \(x_0 \in A\) y sea \(f: A \to \R^k \Rightarrow f\) es continua en \(x_0 \iff \) cada vez que \({(x_n)}_{n \in \N} \subset A\) converge a \(x_0 \Rightarrow f({(x_n)}_{n \in \N})\) converge a \(f(x_0)\).
  \begin{proof}
    Para la ida supongamos \(f\) continua en \(x_0\) y \(x_n \to x_0, {(x_n)}_{n \in \N} \subset A\). Sea \(\e > 0, \exists \delta > 0 : 0 < d(x, x_0) < \delta \Rightarrow d(f(x), f(x_0)) < \e \).
    Además \(\exists n_0 \in \N : (\forall n > n_0)(d(x_n, x_0)) < \delta \) (tomando \(\e = \delta \) en la definición) \(\Rightarrow d(f(x_n), f(x_0)) < \e \therefore \lim_{n \to \infty} f(x_n) = f(x_0)\).

    Para la vuelta supongamos que \(f\) no es continua en \(x_0\), \(x_0\) no es punto aislado de \(A\) y \(\lim_{x \to x_0} f(x) \neq f(x_0)\). Para cada \(n \in \N, \exists x_n \in A : d(x_n, x_0) < \delta \), pero \(d(f(x), f(x_0)) > \e \), es decir que \(\lim_{n \to \infty} x_n = x_0\), pero \(\lim_{n \to \infty} f(x_n) \neq f(x_0)\).
  \end{proof}
\end{prop}

\begin{prop}
  Sea \(A \subset \R^n, x_0 \in A, f: A \to \R^k\) continua en \(x_0\). Si \(B \subset A, x_0 \in B \Rightarrow f|_B: B \to \R^k\) es continua en \(x_0\).
  \begin{proof}
    Ejercicio.
  \end{proof}
\end{prop}

\begin{prop}
  \(A \subset \R^k, x_0 \in A\):
  \begin{enumerate}
    \item \(f, g: A \to \R^k, h: A \to \R \) continuas en \(x_0 \Rightarrow f+g: A \to \R^k, h \cdot f: A \to \R^k\) son continuas.
    \item Si \(f: A \to \R \) es continua en \(x_0\) y \(\forall x \in A, f(x) \neq 0 \Rightarrow \dfrac{1}{f}: A \to \R \) es continua en \(x_0\).
  \end{enumerate}

  \begin{proof}
    Ejercicio
  \end{proof}
\end{prop}

\begin{prop}
  Sea \(A \subset \R^k, x_0 \in A, f: A \to \R^k\) continua en \(x_0\).
  \begin{enumerate}
    \item \(f(x_0) \neq y_0 \Rightarrow \exists U\) entorno de \(x_0\) relativo a \(A\) tal que \(y_0 \neq f(U)\).
    \item \(k \in \R : f(x_0) < k \Rightarrow \exists U\) entorno de \(x_0\) relativo a \(A\): \(f(x) < k \forall x \in U\).
    \item \(\exists U\) entorno de \(x_0\) relativo a \(A : f\) es acotada en \(U\), o sea, \(f(U)\) es acotado.
  \end{enumerate}

  \begin{proof}
    \begin{enumerate}
      \item Sea \(r = d(f(x_0), y_0) > 0 \Rightarrow B_r(f(x_0))\) es un entorno de \(f(x_0)\). Por ser \(f\) continua en \(x_0, \exists U\) entorno relativo a A. \(f(U) \subset B_r(x_0)\) e \(y_0 \notin f(U)\).
      \item Como \((-\infty, k)\) es un entorno de \(f(x_0), \exists U\) entorno de \(x_0\) relativo a \(A\) tal que \(f(U) \subset (-\infty, k)\) es decir que \(\forall x \in U, f(x) < k\).
      \item Si \(r = \|f(x_0)\|, B_r(0)\) es un entorno de \(f(x_0) \Rightarrow \exists U\) entorno de \(x_0\) relativo a \(A\) tal que \(f(U) \subset B_r(0) \therefore f(U)\) es un conjunto acotado.
    \end{enumerate}
  \end{proof}
\end{prop}

\section{Composición de funciones}

\begin{theorem}
  Sea \(A \subset R^l, B \subset \R^k, f: A \to \R^k, g: B \to \R^l : f(A) \subset B\). Sea \(g \circ f: A \to \R^l\). Si \(f\) es continua en \(x_0\) y \(g\) es continua en \(f(x_0) \Rightarrow g \circ f\) es continua en \(x_0\).
  \begin{proof}
    Sea \(W\) un entorno de \(g(f(x_0))\). Como \(g\) es continua en \(f(x_0) \exists V\) un entorno de \(f(x_0)\) relativo a \(B\) tal que \(g(V) \subset W \Rightarrow \exists V_0 \in \R^k : V = V_0 \cap B\). Así que \(V_0\) es entorno de \(f(x_0)\) en \(\R^k\). Como \(f\) es continua en \(x_0, \exists U\) entorno de \(x_0\) relativo a \(A\) tal que \(f(U) \subset V_0\). Como \(f(A) \subset B\), por hipótesis \(\Rightarrow f(U) \subset V_0 \cap B = V\) y \(\therefore g(f(U)) \subset g(V) \subset W\) y \(g \circ f\) es continua en \(x_0\).
  \end{proof}
\end{theorem}

\begin{prop}
  Sea \(A \subset \R^n, x_0 \in A \Rightarrow f: A \to \R^k\) es continua en \(x_0 \iff \) cada una de sus componentes \(f_1, \ldots, f_k: A \to \R \) son continuas en \(x_0\).
  \begin{proof}
    Ejercicio
  \end{proof}
\end{prop}

\section{Continuidad global}

\begin{definition}[Continuidad global]
  \(A \subset \R^n, f: A \to \R^k\) es continua globalmente is es continua \(\forall x \in A\).
\end{definition}

\begin{prop}
  \(A \subset \R^n, f: A \to \R^k\) continua \(\iff \) cada vez que \({(x_n)}_{n \in \N} \subset A, x_n \to x \in A\) se tiene que \(\lim_{n \to \infty} f(x_n) = x\).
  \begin{proof}
    Inmediato.
  \end{proof}
\end{prop}

\begin{prop}
  \(A \subset \R^n, f: A \to \R^k\) son equivalentes:
  \begin{enumerate}
    \item \(f\) es continua.
    \item \(\forall V \subset \R^k\) abierto, \(f^{-1}(V)\) es abierto relativo a \(A\).
    \item \(\forall F \subset \R^k\) cerrado, \(f^{-1}(V)\) es cerrado relativo a \(A\).
  \end{enumerate}
  \begin{proof}
    (1) \(\Rightarrow \) (2)  Si \(x \in f^{-1}(V)\), \(f\) es continua en \(x\) y \(V\) es un entorno de \(f(x)\). Por lo que \(\exists U\) entorno de \(x\) relativo a \(A : f(U) \subset V\). Luego \(U \subset f^{-1}(V)\). Entonces \(f^{-1}(V)\) es un entorno de \(x\) relativo a A y \(f^{-1}(V)\) es abierto relativo a A. \\
    (2) \(\Rightarrow \) (3) Si \(F \subset \R^k\) es cerrado, entonces \(F^c\) es abierto y \(f^{-1}(F^c)\) es abierto relativo a \(A\) por hipótesis, pero \(f^{-1}(F) = A - f^{-1}(F^c)\). Luego \(f^{-1}(F)\) es cerrado relativo a A. \\
    (3) \(\Rightarrow \) (1) Sea \(x \in A\). \(V\) entorno de \(f(x) \Rightarrow f^{-1}(\R^k - V^{\circ})\) es un cerrado relativo a \(A \Rightarrow f^{-1}(V^{\circ}) = A - f^{-1}(\R^k - V^{\circ})\) es abierto relativo a A y \(f(f^{-1}(V^{\circ})) \subset V\). Luego \(f\) es continua en \(x\), pues dado un entorno de \(x\), \(f^{-1}(V^{\circ})\) encontramos un conjunto donde la imagen ``se mete adentro``.
  \end{proof}
\end{prop}

\clearpage

\begin{corollary}
  Sean \(f, g: A \to \R^k\) continuas.
  \begin{enumerate}
    \item \(\{ x \in A : f(x) \neq g(x) \} \) es abierto en \(A\) y \(\{x \in A: f(x) = g(x)\} \) es cerrado en \(A\).
    \item \(\{x \in A : f(x) > 0\} \) es abierto en \(A\).
  \end{enumerate}
  \begin{proof}
    \begin{enumerate}
      \item Sea \(h: A \to \R^k\), \(h = f - g\) es continua y \(\R^k - \{0\} \) es abierto \(\Rightarrow h^{-1}(\R^k - \{0\}) = \{ x \in A : f(x) - g(x) \neq 0 \} \) es abierto en A. Como \( \{0\} \) es cerrado \(\Rightarrow h^{-1}(\{0\}) = \{ x \in A : f(x) - g(x) = 0\} \) es cerrado.
      \item Como \(f\) es continua y \((0, +\infty)\) es abierto, luego \(f^{-1}((0, +\infty)) = \{x \in A : f(x) > 0\} \) es abierto en A.
    \end{enumerate}
  \end{proof}
\end{corollary}

\begin{note}
  La imagen por una función continua de un conjunto abierto o cerrado puede no ser ni abierto ni cerrado.\(f(x) = \dfrac{1}{1+x^2} \) es continua, \(\R \) es abierto y cerrado y \(f(\R) = \interval[open left]{0}{1} \) que no es ni abierto ni cerrado.
\end{note}
