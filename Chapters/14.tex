\section{Conjuntos Conexos y Arco-conexos}

\begin{theorem}
  Todo subconjunto de $\R^n$ no vacío y arco-conexo es conexo.

  \begin{proof}
    Sea $A \neq \varnothing, a_0 \in A$. Cada vez que $a \in A, \exists \sigma_a : [0, 1] \to \R^n$ con valores en $A$ continua y tal que $\sigma_a(0) = a_0$ y $\sigma_a(1) = a$. Como $[0, 1]$ es conexo, $\sigma_a([0, 1])$ es conexo, contiene a $a$ y a $a_0$ y está contenido en $A$. Luego $\bigcup_{a \in A} \sigma_a([0, 1]) = A \therefore$ es conexo.
  \end{proof}
\end{theorem}

\begin{corollary}
  $\R^n$ es conexo y, en particular, los únicos subconjuntos de $\R^n$ que son abiertos y cerrados son $\varnothing, \R^n$.
\end{corollary}

\begin{theorem}
  Un subconjunto abierto de $\R^n$ no vacío es arco-conexo $\iff$ es conexo.

  \begin{proof}
    Sea $A \subset \R^n$ abierto, no vacío y conexo y sea $x_0 \in A$. Considero el conjunto $U$ de todos los puntos de $A$ para los que hay una función continua $\sigma: [0, 1] \to \R^n$ con valores en $A$ tales que $\sigma(0) = x_0$ y $\sigma(1) = x$. Vamos a ver que $U = A$. \\
    Supongamos que $x \in U$. Como $A$ es abierto, $\exists r > 0 : B_r(x) \subset A$. Si $y$ es un punto cualquiera de $B_r(x) \Rightarrow t: [0, 1] \to \R^n$ es continua, verifica que $t(0) = x_0, t(1) = y$ y toma valores en $A$.
    \begin{equation}
      t(\alpha) = \begin{cases}
        \sigma(2\alpha) & \text{si } \alpha \in [0, \frac{1}{2}], \\
        (2 - 2 \alpha)x + (2t-1)y & \text{si } \alpha \in (\frac{1}{2}, 1].
      \end{cases}
    \end{equation} Luego $y \in U \Rightarrow B_r(x) \subset U \Rightarrow U$ es abierto en $\R^n$ y $\therefore$ es abierto en $A$. Veamos que $U$ también es cerrado en $A$. Sea $(u_n)_{n \in \N} \subset U : u_n \to u$ quiero ver que $u \in U$. Como $A$ es abierto $\exists s > 0 : B_s(u) \subset A$ y como $u_n \to u, \exists m : u_m \in B_s(u)$. Como $u_n \subset U, \exists \sigma: [0, 1] \to \R^n$ continua, con valores en $A$ y tal que $\sigma(0) = x_0$, $\sigma(1) = u_m$. Como antes, $t: [0, 1] \to \R^n$ es continua, toma valores en $A$.
    \begin{equation} t(\alpha) = \begin{cases}
        \sigma(2\alpha) & \text{si } \alpha \in [0, \frac{1}{2}], \\
        (2 - 2\alpha) u_m + (2t - 1) u & \text{si } t \in (\frac{1}{2}, 1].
      \end{cases}
    \end{equation}$\Rightarrow t(0) = x_0$, $t_1 = u \Rightarrow u \in U$. \\
    En definitiva $U$ es abierto y cerrado en $A, U \neq \varnothing, x_0 \in U$ y como $A$ es conexo, debe ser $U = A$ y $A$ es arco-conexo.
  \end{proof}
\end{theorem}

\section{Continuidad uniforme}

\begin{definition}[Continuidad uniforme]
  $f: A \to \R^k$ es uniformemente continua si \begin{equation}
    (\forall \e > 0)(\exists \delta > 0) : (\forall x,y \in A), \|x-y\| < \delta \Rightarrow \|f(x) - f(y)\| < \delta
  \end{equation}
\end{definition}

\begin{eg}
  \begin{enumerate}
    \item $f(x) = x$. Tomando $\e = \delta$ en la definición.
    \item $f: \R \to \R, f(x) = sen(x)$, $f$ es continua en $\R$, en particular en $x = 0$. Dado $\e > 0, \exists \delta > 0 : \forall z \in \R$ si $|z-0| < \delta \Rightarrow |sen(z) - 0| < \dfrac{\e}{2} \Rightarrow |f(x) - f(y)| = |2 \cdot sin(\dfrac{x-y}{2}) \cdot cos(\dfrac{x+y}{2})| < 2 \cdot \dfrac{\e}{2} = \e$.
    \item $f: (0, +\infty) \to \R$, $f(x) = cos(\dfrac{\pi}{x})$. Sea $(x_n)_{n \in \N}$, $x_n = \dfrac{1}{n}$, $|x_n - x_{n+1}| < \dfrac{1}{n}$, pero $|f(x_n) - f(x_{n+1})| = 2 (\forall n \in \N)$. Esto implica que $\nexists \delta > 0 : (\forall x,y \in (0, +\infty)) : |x-y| < \delta \Rightarrow |f(x) - f(y)| < \e$. Pues $\delta$ debería ser menor que $\dfrac{1}{n} (\forall n \in \N) \therefore f$ no es uniformemente continua.
  \end{enumerate}
\end{eg}

\begin{theorem}
  $f$ continua, $f: K \subset \R^n \to \R^k$, $K$ compacto $\Rightarrow f$ es uniformemente continua.
  \begin{proof}
    Sea $\e > 0$, como $f$ es continua $(\forall x \in K)(\exists \delta_x > 0) : (\forall x_0 \in K) : \|x - x_0\| < \delta_x \Rightarrow \|f(x) - f(x_0)\| < \frac{\e}{2}$. Sea $U = \{B_{\delta_x} : x \in K\}$ es un cubrimiento abierto de $K$. Como $K$ es compacto $U$ tiene un número de Lebesgue, $\exists \delta > 0 : x, y \in K$ y $ \|x-y\| < \delta \Rightarrow \exists u \in U : x, y \in u$. Dados $x, y \in K : \|x-y\| < \delta \Rightarrow \exists z \in K : x, y \in B_{\delta_z}(z) \Rightarrow \|f(x) - f(z)\| < \frac{\e}{2}$ y $\|f(y) - f(z)\| < \frac{\e}{2}$. 
    \begin{equation}
      \|f(x) - f(y)\| = \|f(x) - f(z) - (f(y) - f(z))\| \leq \|f(x) - f(z)\| + \|f(y) - f(z)\| < \e
    \end{equation} $\therefore f$ es uniformemente continua.
  \end{proof}
\end{theorem}

\begin{theorem}
  Sea $A \subset \R^n$, $f: A \to \R^k$ uniformemente continua $\Rightarrow \exists !$ función continua $\overline{f}: \overline{A} \to \R^k : \overline{f}(x) = f(x) (\forall x \in A)$ y $\overline{f}$ es uniformemente continua.
  \begin{proof}
    \begin{enumerate}
      \item Afirmo que dado $x_0 \in \overline{A}$, $(x_n)_{n \in \N} \subset A$, $x_n \to x_0 \Rightarrow (f(x_n))_{n \in \N}$ converge. En efecto, dado $\e > 0$, $\exists \delta > 0 : \|x-y\| < \delta \Rightarrow \|f(x) - f(y)\| < \e$. Luego $\exists n_0 \in \N : (\forall n, m > n_0) \|x_n - x_m\| < \frac{\delta}{2} \Rightarrow$ \begin{equation} 
        \|x_0 - x_m\| + \|x_m - x_0\| < \delta \iff \|x_n - x_m\| < \delta \iff \|f(x_n) - f(x_m)\| < \e
      \end{equation} $\therefore (f(x_n)_{n \in \N})$ es de Cauchy.
      \item Afirmo que sea $x_0 \in \overline{A}$, $\exists ! L_x \in \R^k : (x_n)_{n \in \N} \subset A$ y $x_n \to x_0$, $f(x_n) \to L_x$. En efecto, supongamos que $x_n \to x_0$, $y_n \to x_0$ y $f(x_n) \to L_x$ y $f(y_n) \to L_y$. Como $f$ es uniformemente continua en $A$, $\exists \delta > 0 : (\forall u, v \in A) : \|u-v\| < \delta \Rightarrow \|f(u) - f(v)\| < \e$. Como $x_n \to x_0$ e $y_n \to x_0 \Rightarrow \exists n_0 \in \N : (\forall n, m > n_0) (\|x_n - x_0\| < \frac{\delta}{2}) (\| y_n - x_0 \| < \frac{\delta}{2})$ \begin{equation}
        \Rightarrow \|x_n - y_n\| \leq \|x_n - x_0\| + \|y_n - x_0\| < \delta \Rightarrow \|f(x_n) - f(y_n)\| < \frac{\e}{3}
      \end{equation} Luego como $\|f(x_n) - L_x\| < \frac{\e}{3}$ y $\|f(y_n) - L_y\| < \frac{\e}{3}$, $(\forall n > n_0) \therefore \|L_x - L_y\| < \e$.
      \item Acabamos de ver que $\exists \overline{f}: \overline{A} \to \R^k : f(x) = L_x$ Veamos que a) $f(A) \subset \overline{f}|_A$ b) $\overline{f}$ es uniformemente continua en $\overline{A}$. \begin{enumerate}
        \item Sea $x \in A$, $x_n = x$ $(\forall n \in \N)$, $x_n \to x$, $f(x_n) \to f(x) = L_x = \overline{f}(x)$, $(f(x))_{n \in \N}$ es constante $f(x) = \overline{f}(x)$.
        \item Dado $\e > 0$, $\exists \delta > 0 : \|x-y\| < \delta \Rightarrow \|f(x) - f(y)\| < \frac{\e}{3}$. Sean $u, v \in \overline{A} : \|u - v\| < \delta \Rightarrow \exists (u_n)_{n \in \N}$, $(v_n)_{n \in \N} \subset \overline{A}$ con $u_n \to u$, $v_n \to v$, $f(u_n) \to \overline{f}(u)$, $f(v_n) \to \overline{f}(v)$. Luego $\exists n_0 \in \N : \|f(u_n) - \overline{f}(u)\| < \frac{\e}{3}$ y $\|f(v_n) - \overline{f}(v)\| < \frac{\e}{3}$. Además $\|u_n - u\| < \frac{1}{2} \cdot (\delta - \|u - v\|)$, $\|v_n - v\| < \frac{1}{2} \cdot (\delta - \|u - v\|)$, $(\forall n > n_0$). Luego \begin{equation}
          \|u_n - v_n\| < \|u_n - u\| + \|v_n - v\| + \|u - v\| < \delta \Rightarrow
        \end{equation}
        \begin{equation}
          \|\overline{f}(u) - \overline{f}(v)\| < \|\overline{f}(u) - f(u_n)\| + \|f(u_n) - f(v_n)\| + \|f(v_n) - \overline{f}(v)\| < \e
        \end{equation} $\therefore \overline{f}$ es uniformemente continua.
      \end{enumerate}
      \item Veamos por último la unicidad, si $g: \overline{A} \to \R^k$ continua y $g(x) = f(x) (\forall x \in A) \Rightarrow g|_A = \overline{f}|_A$. En efecto, como $A \subset \overline{A} \Rightarrow \overline{f} = g$. 
    \end{enumerate}
  \end{proof}
\end{theorem}

% TODO: Revisar esta sección
\section{Diferenciación}

\begin{definition}
  Sea $F : A \subset \R^n \to \R^n$ y sea $p \in A^{\circ}$ si $T_p : \R^n \to \R^n$ es una transformación lineal que verifica:
  \begin{equation}
    lim_{x \to p} \dfrac{\| F(x) - F(p) - T_p(x-p) \|}{\|x-p\|} = 0 \Rightarrow
  \end{equation} Decimos que $F$ es diferenciable en $p$. A la transformación lineal la llamamos diferencial de $F$ en $p$ y lo notamos $DF_p$.
\end{definition}

Si $F = (f_1, \cdots, f_n)$ con $f_i: A \to \R$ $i \in \{1, \cdots, n\}$, $F$ es diferenciable en $p \iff$ cada $f_i$ es diferenciable en $p$.
Además la diferencial de $F$ es la matriz que se obtiene al poner como filas los gradientes de $f_i$.

\begin{equation}
  DF_p = \begin{pmatrix}
    \nabla f_1(p) \\
    \vdots \\
    \nabla f_n(p)
  \end{pmatrix} = \begin{pmatrix}
    \dfrac{\partial f_1}{\partial x_1}(p) & \cdots & \dfrac{\partial f_1}{\partial x_n}(p) \\
    \vdots & \cdots & \vdots \\
    \dfrac{\partial f_n}{\partial x_1}(p) & \cdots & \dfrac{\partial f_n}{\partial x_n}(p)
  \end{pmatrix}
\end{equation}

También podemos observar que si $F$ es diferenciable en $P$ y $v \in \R^n \Rightarrow \alpha(t) = p + t \cdot v$.

\begin{equation}
  0 = lim_{t \to 0} \dfrac{\| F(p + t \cdot v) - F(p) - DF_p(t \cdot v) \|}{|t|} 
\end{equation}
\begin{equation}
  = lim_{t \to 0} \| \dfrac{F(p + tv) - F(p)}{t} - DF_p(v) \| 
\end{equation}
\begin{equation}
  \iff DF_p(v) = lim_{t \to 0} \dfrac{F(p+t \cdot v) - F(p)}{t} \text{  } (\forall v \in \R^n)
\end{equation}

\begin{note}
  Si $G, H: A \subset \R^n \to \R^n$ son diferenciables podemos considerar su producto escalar como $F(x) = \langle G(x), H(x) \rangle$ que es diferenciable.
  Además $lim_{t \to 0} \dfrac{F(p + tx) - F(p)}{t}$ = \begin{equation}
    \lim_{t \to 0} \dfrac{1}{t} ( \langle G(p + tx), H(p+tx) \rangle - \langle G(p), H(p) \rangle )
  \end{equation}
  \begin{equation}
    \lim_{t \to 0} \dfrac{1}{t} ( \langle G(p+tx), H(p+tx) \rangle - \langle G(p), H(p+tx) \rangle + \langle G(p), H(p+tx) \rangle - \langle G(p), H(p) \rangle)
  \end{equation}
  \begin{equation}
    \lim_{t \to 0} \dfrac{1}{t} (\langle G(p+tx) - G(p), H(p + tx) \rangle + \langle G(p), H(p+tx) - H(p) \rangle) 
  \end{equation}
  \begin{equation}
    \langle DG_p(x), H(p) \rangle + \langle G(p), DH_p \rangle \Rightarrow
  \end{equation}
  \begin{equation}
    D\langle G, H \rangle_p = \langle DG_p(x), H)(p) \rangle + \langle G(p), DH_p(x) \rangle
  \end{equation}
\end{note}

\begin{lemma}
  Si $T: \R^n \to \R^n$ es transformación lineal, $\exists c > 0: \|T(x)\| < c \|x\|$ $(\forall x \in \R^n)$. Al mínimo de estas constantes lo llamamos $\|T\|_{\infty}$. Vale que $\|T\|_{\infty} = max_{\|x\| \leq 1} \|T(x)\|$ en particular
  $\|T\| \leq \|T\|_{\infty} \|x\|$.
  \begin{proof}
    $T$ es lineal, luego $T$ es continua. Tomar norma también es continua así que $x \mapsto \|Tx\|$ es continua. Como $\overline{B_1(0)}$ es compacto la función alcanza máximo. Digamos $\|T\|_{\infty} = max_{x \in B_r(0)} \|Tx\| \Rightarrow x \neq 0$ \\
    $\|T(\dfrac{x}{\|x\|})\| \leq \|T\|_{\infty} \Rightarrow \|Tx\| \leq \|x\| \|T\|_{\infty}$ por como la elegimos es la mejor cota.
  \end{proof}
\end{lemma}

\begin{lemma}
  La función $\| \cdot \|: \R^{n \times m} \to \R$ es una norma.
  \begin{enumerate}
    \item $\|T\|_{\infty} > 0$, $\|T\|_{\infty} = 0 \iff T = 0$.
    \item $|\lambda| \|T\|_{\infty} = \|\lambda T\|_{\infty}$.
    \item $\| T + S\| \leq \|T\|_{\infty} + \|S\|_{\infty}$.
  \end{enumerate}
  \begin{proof}
    \begin{enumerate}
      \item $\|T\|_{\infty} > 0$ es trivial. Si $T = 0 \Rightarrow Tx = 0 \Rightarrow \|Tx\|_{\infty} = 0$ y si $\|T\|_{\infty} \neq 0 \Rightarrow Tx=0 (\forall x \in \overline{B_1(0)}$ pues por linealidad si 
        \begin{equation}
          x \neq 0 \Rightarrow Tx = T(\dfrac{x}{\|x\|} \cdot \|x\|) = \|x\| T(\dfrac{x}{x}) = 0 \therefore T = 0
        \end{equation}
      \item Trivial
      \item $S + T \in \R^{n \times m}$ dado $x \in \overline{B_1(0)}$ \begin{equation}
          \|(S+T)(x)\| = \| Tx + Sx \| \leq \|Tx\| + \|Sx\| \leq  \|x\| \cdot (\|Tx\|_{\infty} + \|Sx\|_{\infty}) 
      \end{equation} 
      \begin{equation}
        \leq \|Tx\|_{\infty} + \|Sx\|_{\infty} \therefore \|(T+S)(x)\|_{\infty} \leq \|T\|_{\infty} + \|S\|_{\infty}
      \end{equation}
    \end{enumerate}
  \end{proof}
\end{lemma}

\begin{lemma}
  Si $F: A \subset \R^n \to \R^n$ es diferenciable con $p \in A^{\circ}$, $\exists R > 0$ y $c_r > 0 : x \in B_r(p) \Rightarrow \| F(x) - F(p) \| \leq c_r \| x - p\|$ 
  Esto implica la continuidad uniforme en $B_r(p)$.
  \begin{proof}
    Como $F$ es diferenciable en $p$, $\exists B_r(p)$ donde el cociente de diferenciabilidad es menor que $1$. \begin{equation}
      \dfrac{\|F(x) - F(p)\|}{\|x - p \|} \leq \dfrac{\|F(x) - F(p) - DF_p(x-p) \|}{\|x-p\|} + \dfrac{\| DF_p(x-p) \|}{\|x-p\|}
    \end{equation}
    \begin{equation}
      < 1 + \dfrac{\|DF_p(x-p)\|_{\infty} \|x-p\|}{\|x-p\|} = 1 + \|DF_p(x-p)\|_{\infty}
    \end{equation}
    Si $x \in B_r(p) \Rightarrow$. Luego $c_r = 1 + \|DF_p(x-p)\|_{\infty} \Rightarrow \|F(X) - F(p)\| \leq c_r \|x-p\|$ $(\forall x \in B_r(p))$.
  \end{proof}
\end{lemma}
