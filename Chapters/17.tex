\section{Integral de Riemann}

Consideremos $f: [a, b] \subset \R \to \R$ acotada. Digamos $m = inf\{ f(x) : x \in [a,b] \}$ y $M = sup \{ f(x) : x \in [a, b] \}$.
Una partición de $[a, b]$ es un subconjunto finito $P \subset [a, b]$ tal que $a \in P$ y $b \in P$.
Escribimos $P = \{ t_0, \cdots, t_n \}$ y por convención $t_0 = a$, $t_n = b$ y $t_0 < t_1 < \cdots < t_n$.
Notamos $m_i$, $M_i$ al ínfimo y al supremo de $f$ en $[t_{i-1}, t_i]$ $i \in \{ 1, \cdots, n \}$ de la partición.
\begin{equation}
  s(f, P) = m_1 (t_1 - t_0) + \cdots + m_n (t_n - t_{n-1})
\end{equation}

\begin{equation}
  S(f, P) = M_1 (t_1 - t_0) + \cdots + M_n (t_n - t_{n-1})
\end{equation}

Vale que $m (b-a) \leq s(f, P) \leq S(f, P) \leq M (b-a)$

\begin{definition}
  Si $P$, $G$ son particiones de un mismo intervalo y $P \subset G$ decimos que $G$ es más fina que $P$.
\end{definition}

\begin{lemma}
  Sea $f: [a, b] \to \R$ acotada y sea $P$, $G$ particiones del intervalo, si $P \subset G \Rightarrow s(f, P) \leq s(f, G)$ y $S(f, G) \leq S(f, P)$.
  \begin{proof}
    Si $P = \{ t_0, t_1, \cdots t_n \}$ y $G = \{ t_0, t_1, \cdots, t_{i-1}, r, t_i, \cdots, t_n \}$. Si $m_i = inf_{[t_{i-1}, t_i]}(f)$, $m^{\prime} = inf_{[t_{i-1}, r]}(f)$, $m^{\prime \prime} = inf_{[r, t_i]}(f) \Rightarrow m_i \leq m^{\prime}$, $m_i \leq m^{\prime \prime}$. \begin{equation}
      s(f, G) - s(f, P) = m^{\prime} (r - t_{i-1}) + m^{\prime \prime} (t_i - r) - m_i (t_i - t_{i-1})
    \end{equation}
    \begin{equation}
      = (m^{\prime} - m)(r - t_{i-1}) + (m^{\prime \prime} - m_i) (t_i - r) \geq 0
    \end{equation}
    Análogamente para sumas superiores. Iterando en $S_i$, $P \subset G$ se ve que $s(f, P) \leq s(f, G)$.
  \end{proof}
\end{lemma}

\begin{corollary}
  Sea $f: [a, b] \to \R$ acotada si $P$, $G$ son particiones de $[a, b]$ vale que $s(f, P) \leq S(f, G)$
  \begin{proof}
    Como $P \cup G$ refina a $P$ y a $Q \Rightarrow s(f, P) \leq s(f, P \cup G) \leq S(f, P \cup G) \leq S(f, G)$
  \end{proof}
\end{corollary}

\begin{definition}
  Si $f: [a, b] \to \R$ es acotada, definimos su integral inferior como \begin{equation}
    \lowint_a^b f(x)\,\mathrm{d}x = sup_P s(f, P)
  \end{equation}
  Donde el supremo es sobre todas las particiones $P$ de $[a, b]$. Es decir $\forall$ partición $P$ de $[a, b]$, $s(f, P) \leq \lowint_a^b f(x)\, \mathrm{d}x$.
  Además dado $\e > 0$ $\exists P : \lowint_a^b f(x)\, \mathrm{d}x \leq s(f, P) + \e$.
\end{definition}

\begin{definition}
  Análogamente se define la integral superior de $f$ como \begin{equation}
    \upint_a^b f(x)\, \mathrm{d}x = inf_P S(f, P)
  \end{equation}
\end{definition}

\begin{lemma}
  Sea $a < c < b$. Si consideramos solo las particiones de $[a, b]$ que contengan a $c$, los valores de la integral superior e inferior no cambian.
  \begin{proof}
    Dada una partición de $[a, b]$. Si le agrego $c$ tengo otra partición $P^{\prime} : s(f, P) \leq s(f, P^{\prime}) \Rightarrow$ \\
    Por un lado el supremo sobre todas las particiones es mayor o igual que el supremo de las particiones que contienen a $c$, pero dada cualquier particiones que no contiene a $c$,
    hay una que la refina y si lo contienen (agregando a $c$) y la suma inferior es mayor o igual entonces el supremo es igual sobre los dos conjuntos.
  \end{proof}
\end{lemma}

\clearpage

\begin{theorem}
  Sea $a < c < b$ y $f: [a, b] \to \R$ acotada. Entonces \begin{equation}
    \lowint_a^b f(x)\,\mathrm{d}x = \lowint_a^c f(x)\,\mathrm{d}x + \lowint_c^b f(x)\,\mathrm{d}x
  \end{equation} y \begin{equation}
    \upint_a^b f(x)\, \mathrm{d}x = \upint_a^c f(x)\, \mathrm{d}x + \upint_c^b f(x)\, \mathrm{d}x
  \end{equation}

  \begin{proof}
    Si $A$, $B$ son los conjuntos de sumas inferiores de $f|_{[a, c)}$ y $f|_{[c, b]} \Rightarrow A + B = \{ x+y : x \in A, y \in B \}$ es el conjunto de las sumas inferiores de $f$ respecto a particiones
    de $[a, b]$ que contienen a $c \Rightarrow$ \begin{equation}
      \lowint_a^b f(x)\,\mathrm{d}x = sup(A+B) = sup(A) + sup(B) = \lowint_a^c f(x)\,\mathrm{d}x + \lowint_c^b f(x)\,\mathrm{d}x
    \end{equation}
    Análogamente para las sumas superiores.
  \end{proof}
\end{theorem}

\begin{prop}
  Sean $f, g: [a, b] \to \R$ acotadas $\Rightarrow$
  \begin{equation}
    \lowint_a^b f(x)\,\mathrm{d}x + \lowint_a^b g(x)\,\mathrm{d}x \leq \lowint_a^b f(x)+g(x)\,\mathrm{d}x
  \end{equation}
  \begin{equation}
    \leq \upint_a^b f(x) + g(x)\, \mathrm{d}x \leq \upint_a^b f(x)\, \mathrm{d}x + \upint_a^b g(x)\, \mathrm{d}x
  \end{equation}
  \begin{proof}
    Sea $m_i(f)$, $m_i(g)$, $m_i(f+g)$ los ínfimos de $f$, $g$, $f+g$ respectivamente en $[t_{i-1}, t_i]$ de una partición $P \Rightarrow m_i(f+g) \geq m_i(f) + m_i(g)$. \\
    Por un lado tenemos que \begin{equation}
      \lowint_a^b f(x)+g(x)\, \mathrm{d}x \geq s(f, P) + s(g, P)
    \end{equation} y por otro lado dadas dos particiones $P$, $G$ \begin{equation}
      s(f, P) + s(g, G) \leq s(f, P \cup G) + s(g, P \cup G) \leq \lowint_a^b f(x)+g(x)\,\mathrm{d}x \Rightarrow
    \end{equation}
    \begin{equation}
      \lowint_a^b f(x) \, \mathrm{d}x + \lowint_a^b g(x) \, \mathrm{d}x = sup_{P, G}(s(f, P)+s(g, G)) \leq \lowint_a^b f(x) + g(x)\, \mathrm{d}x
    \end{equation}
  \end{proof}
\end{prop}

\begin{prop}
  Sean $f, g: [a, b] \to \R$ acotadas $\Rightarrow$
  Si $c > 0$ \begin{equation}
    \lowint_a^b c f(x)\,\mathrm{d}x = c \lowint_a^b f(x)\,\mathrm{d}x
  \end{equation} y \begin{equation}
    \upint_a^b c f(x)\, \mathrm{d}x = c \upint_a^b f(x)\, \mathrm{d}x
  \end{equation}
  Si $c < 0$ \begin{equation}
    \lowint_a^b c f(x)\,\mathrm{d}x = c \upint_a^b f(x)\,\mathrm{d}x
  \end{equation} y \begin{equation}
    \upint_a^b c f(x)\,\mathrm{d}x = c \lowint_a^b f(x)\,\mathrm{d}x
  \end{equation}
  \begin{proof}
    $m_i(c f) = c m(f)$ y $M_i(c f) = c M_i(f)$ si $c > 0$ y $m_i(c f) = c M_i(f)$ y $M_i(c f) = c m_i(f)$ si $c < 0$ por propiedades de supremo e ínfimo.
  \end{proof}
\end{prop}

\begin{prop}
  Sean $f, g: [a, b] \to \R$ acotadas $\Rightarrow$ si $f(x) \leq g(x)$ $(\forall x \in [a, b]) \Rightarrow$ \begin{equation}
    \lowint_a^b f(x)\,\mathrm{d}x \leq \lowint_a^b g(x)\,\mathrm{d}x
  \end{equation} y \begin{equation}
    \upint_a^b f(x)\, \mathrm{d}x \leq \upint_a^b g(x)\, \mathrm{d}x
  \end{equation}
  \begin{proof}
    Si $f(x) \leq g(x)$ $(\forall x \in [a, b]) \Rightarrow m_i(f) \leq m_i(g)$ y $M_i(f) \leq M_i(g)$ $\forall$ partición $P$ de $[a, b] \Rightarrow s(f, P) \leq s(g, P)$ y $S(f, P) \leq s(g, P) \Rightarrow$ vale para la integral.
  \end{proof}
\end{prop}

\begin{definition}[Integrable]
  Una función acotada $f: [a, b] \to \R$ se dice integrable si \begin{align*}
    \lowint_a^b f(x) \, \mathrm{d}x = \upint_a^b f(x) \, \mathrm{d}x
  \end{align*}
\end{definition}

\clearpage

\begin{eg}
  $f: [a, b] \to \R$, $f(x) = \begin{cases}
      1 & \text{si } x \in \Q,   \\
      0 & \text{si } x \notin \Q
    \end{cases}$
  No es integrable.
  \begin{proof}
    Para cualquier partición de $[a, b]$, $m_i = 0$, $M_i = 1$ en todos los intervalos de la partición $\Rightarrow s(f, P) = 0$, $S(f, P) = b-a$ $(\forall P)$.
  \end{proof}
\end{eg}

\begin{note}
  Si dada $f: [a, b] \to \R$ acotada llamamos $\sigma$ al conjunto de sumas inferiores y $\sum$ al de las superiores, decir que $f$ es integrable signifca que $sup \sigma = inf \sum$.
  En particular esto pasa $\iff (\forall \e > 0)(\exists s \in \sigma)$, $S in \sum : S - s < \e$.
  \begin{proof}
    Ejercicio
  \end{proof}
\end{note}

\begin{definition}[Oscilación]
  Dada una función $f: [a, b] \to \R$ acotada, definimos su oscilación en $X \subset [a, b]$ como $\omega(f(X)) = sup(f(X)) - inf(f(X))$.
\end{definition}

\begin{note}
  $\omega(f(X)) = sup(\{ |f(x) - f(y)| : x,y \in X \})$.
  \begin{proof}
    Ejercicio
  \end{proof}
\end{note}

\clearpage

\begin{theorem}
  Sea $f: [a, b] \to \R$ acotada. Son equivalentes \begin{enumerate}
    \item $f$ es integrable.
    \item $(\forall \e > 0)(\exists P, G)$ particiones de $[a, b]$ tales que $S(f, G) - s(f, P) < \e$.
    \item $(\forall \e > 0)(\exists P)$ partición de $[a, b]$ tal que $S(f, P) - s(f, P) < \e$.
    \item $(\forall \e > 0)(\exists P = \{ t_0, t_1, \cdots, t_n \})$ partición de $[a, b] : \sum_{i = 1}^n \omega_i (t_i - t_{i-1}) < \e$
  \end{enumerate}

  \begin{proof}
    1) $\iff$ 2) es el lema anterior. \\
    3) $\iff$ 4) vale porque $\sum_{i = 1}^n \omega_i (t_i - t_{i-1}) = S(f, P) - s(f, P)$. \\
    3) $\Rightarrow$ 2) trivial. \\
    2) $\Rightarrow$ 3) Si $S(f, G) - s(f, P) < \e$ tomando $P_0 = P \cup G$. \begin{align*}
      s(f, P) & \leq s(f, P_0) \\
      & \leq S(f, P_0) \leq S(f, Q) \\
      & \Rightarrow S(f, P_0) - s(f, P_0) < \e
    \end{align*}
  \end{proof}
\end{theorem}

\clearpage

\section{Continuidad e integración}

\begin{theorem}
  Sean $f$, $g$ integrables $\Rightarrow$ \begin{enumerate}
    \item Para $a < c < b$, $f|_{[a, c]}$ y $f|_{[c, b]}$ son integrables y $\int_a^b f(x)\, \mathrm{d}x = \int_a^c f(x)\, \mathrm{d}x + \int_c^b f(x)\, \mathrm{d}x$ y vale la recíproca.
    \item $(\forall c \in \R)$ $c f$ es integrable y $\int_a^b c f(x)\, \mathrm{d}x = c \int_a^b f(x)\, \mathrm{d}x$.
    \item $f+g$ es integrable y la integral de la suma es la suma de las integrales.
    \item Si $f(x) \leq g(x)$ $(\forall x \in [a, b]) \Rightarrow \int_a^b f(x) \, \mathrm{d}x \leq \int_a^b g(x) \, \mathrm{d}x$.
    \item $|f(x)|$ es integrable y además $|\int_a^b f(x) \, \mathrm{d}x = \int_a^b |f(x)| \, \mathrm{d}x$.
    \item El producto es integrable.
  \end{enumerate}
  \begin{proof}
    1, 2, 3, 4 equivalente a las demostraciones de suma superior e inferior. \\
    5) $| |f(x)| - |f(y)|| \leq |f(x) - f(y)|$. Para cualquier $x \in [a, b]$ vale que $\omega(|f|, X) \leq \omega(f, X)$. En particular $\forall P$ partición de $[a, b]$ $\omega_i(|f|) \leq \omega_i(f) \therefore |f|$ es integrable y como además $-|f(x)| \leq f(x) \leq |f(x)| \Rightarrow$ 
    \begin{align*}
      -\int_a^b |f(x)| \, \mathrm{d}x & \leq \int_a^b f(x) \, \mathrm{d}x \leq \int_a^b |f(x)| \, \mathrm{d}x \Rightarrow \\
      & | \int_a^b f(x) \, \mathrm{d}x | \leq \int_a^b |f(x)| \, \mathrm{d}x
    \end{align*}
    6) Dada una partición $P$ de $[a, b]$ sean $\omega_i(f \cdot g)$, $\omega_i(f)$, $\omega_i(g)$ las oscilaciones de las funciones en $[t_{i-1}, t_i]$ 
    \begin{align*}
      |fg(x) - fg(y)| & = |f(x)g(x) - f(y)g(y) + f(x)g(y) - f(x)g(y)| \\
      & \leq |f(x)| \cdot |g(x)- g(y)| + |g(y)| \cdot |f(x) - f(y)| \\
      & \leq k \omega(g) + k \omega(f) = k (\omega(g) + \omega(f)) \\
    \end{align*} Luego $f \cdot g$ es integrable.
  \end{proof}
\end{theorem}

\clearpage

\begin{theorem}
  Toda función $f : [a, b] \to \R$ continua es integrable.
  \begin{proof}
    Como $[a, b]$ es compacto, $f$ es acotada y uniformemente continua. Dado $\e > 0$ $\exists \delta > 0 : x, y \in [a, b]$, $|x-y| < \delta \Rightarrow |f(x) - f(y)| < \dfrac{\e}{b-a}$. \\
    Sea $n \in \N : \dfrac{b-a}{n} < \delta$ y consideremos la partición de $[a, b]$ dada por $t_i = a + i \cdot \dfrac{b-a}{n}$ con $i \in \{0, \cdots, n\} \Rightarrow$ si $x, y \in [t_{i-1}, t_i]$ vale que $|f(x) - f(y)| \Rightarrow \sum_{i=1}^n \omega_i (t_{i-1} - t_i) < \e \therefore f$ es integrable.
  \end{proof}
\end{theorem}

\begin{theorem}
  Sea $f: [a, b] \to R$ acotada. Si para cada $c \in [a, b)$ $f|_{[a, c]}$ es integrable $\Rightarrow f$ es integrable.
  \begin{proof}
    Sea $|f(x)| \leq k$ $(\forall x \in [a,b])$. Dado $\e > 0$, sea $c \in [a, b)$ tal que $k \cdot (b-c) < \frac{\e}{4}$. Como $f|_{[a, c]}$ es integrable $\exists \{t_0, \cdots, t_n\}$ partición de $[a, c]$ tal que $\sum_{i = 1}^n \omega_i (t_i - t_{i-1}) < \frac{\e}{2}$. Si agrego $t_{n+1} = b$ tengo una partición de $[a, b]$ y $\omega_{n+1} \leq 2 \cdot k$. Luego $\omega_{n+1} \cdot (t_{n+1} - t_n) = \omega_{n+1} \cdot (b-c) < \frac{\e}{2} \Rightarrow \sum_{i = 1}^n \omega_i (t_i - t_{i-1}) < \e$.
  \end{proof}
\end{theorem}

\begin{corollary}
  $f: [a, b] \to \R$ acotada. Si para $a < b < c < d$ cualesquiera $f|_{[b, c]}$ es integrable $\Rightarrow f$ es integrable.
\end{corollary}

\begin{corollary}
  $f: [a, b] \to \R$ acotada y tiene finitas discontinuidades entonces es integrable.
  \begin{proof}
    Sea $t_0$, $\cdots$, $t_n$ los puntos donde $f$ es discontinua $\Rightarrow f$ es integrable en $[t_{i-1}, t_i]$ por ser continua en $[c, d]$ con $t_{i-1} < c < d < t_i$.
  \end{proof}
\end{corollary}

\clearpage

\begin{eg}
  Sea $f: [a, b] \to \R$ \begin{equation}
    f(x) = \begin{cases}
      0           & \text{si } x = 0, x \in \R - \Q,                    \\
      \frac{1}{q} & \text{si } x = \frac{p}{q} \text{ con } (p, q) = 1.
    \end{cases}
  \end{equation}
  $f$ es discontinua en un conjunto infinito (los racionales de $[a, b]$), pero es integrable en $[a, b]$ y $\int_a^b f(x)\, \mathrm{d}x = 0$.
  \begin{proof}
    Dado $\e > 0$, $F = \{ x \in [a, b] : f(x) \geq \dfrac{\e}{b-a} \}$ es finito. Tomemos una partición de $[a, b]$ tal que la suma de los intervalos de $P$ que contienen a un punto de $F$ sea menor que $\e \Rightarrow F \cap [t_{i-1}, t_i] = \varnothing \Rightarrow 0 \leq f(x) < \dfrac{\e}{b-a} \Rightarrow M_i < \dfrac{\e}{b-a}$. \\
    $S(f, P) = \sum M_i (t_i - t_{i-1}) = \sum M_i (t_i - t_{i-1}) + \sum M_i (t_i - t_{i-1}) < \e + \dfrac{\e}{b-a} \cdot (b-a) < 2 \cdot \e$. Donde la primer suma es sobre los intervalos que contienen algún punto de $F$ y la segunda sobre los que no. Luego $\upint_a^b f(x) \, \mathrm{d}x = 0$ y $f(x) \geq 0 \Rightarrow 0 \leq \lowint_a^b f(x) \, \mathrm{d}x \leq \upint_a^b f(x) \, \mathrm{d}x = 0$.
    Por lo tanto $\int_a^b f(x) \, \mathrm{d}x = 0$.
  \end{proof}
\end{eg}
