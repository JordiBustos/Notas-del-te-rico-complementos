\section{Teorema fundamental del cálculo}

\begin{theorem}[Teorema Fundamental del Cálculo]
  Sea \(f: [a, b] \to \R \) integrable. Si \(f\) es continua en \(c \in [a, b] \Rightarrow F: [a, b] \to \R \), \(F(x) = \int_a^x f(t)\, \mathrm{d}t\) es derivable en \(c\) y \(f(x) = F^{\prime}(c)\).
  \begin{proof}
    \begin{align*}
      \left| \dfrac{F(c+h) - F(c)}{h} - f(c) \right| & = |\dfrac{1}{h}| \cdot \left| \int_c^{c+h} f(t)\, \mathrm{d}t - h \cdot f(c) \right| \\
                                                     & \leq \dfrac{1}{h} \cdot \int_c^{c+h}| f(t) - f(c) | \, \mathrm{d}t *
    \end{align*}
    Como \(f\) es continua es continua en \(c \Rightarrow \) Dado \(\e > 0\), \(\exists \delta > 0 : t \in [a, b]\) y \(|t-c| < \delta \Rightarrow |f(t) - f(c)| < \e \).
    Si \(0 < h < \delta \) y \(c+h \in [a, b]\) tenemos que \(* \leq \dfrac{1}{h} \cdot \e \cdot h = \e \Rightarrow \) \begin{align*}
      \lim_{h \to 0^+} \dfrac{F(c+h) - F(c)}{h} = f(c)
    \end{align*}
    Para \(h < 0\) es análogo, luego \(\lim_{h \to 0^-} \dfrac{F(c+h) - F(c)}{h} = f(c) \Rightarrow F^{\prime}(c) = f(c)\).
  \end{proof}
\end{theorem}

\begin{corollary}[Regla de Barrow]
  \(f: [a, b] \to \R \) integrable y tiene una primitiva \(F: [a, b] \to \R \Rightarrow \int_a^b f(x) \, \mathrm{d}x = F(b) - F(a)\).
  \begin{proof}
    Para cualquier partición de \([a, b]\), \(P = \{ t_0, \ldots, t_n \} \Rightarrow F(b) - F(a) = \sum_{i = 1}^n (F(t_i) - F(t_{i-1})) = \sum_{i = 1}^n F(\alpha_i) (t_i - t_{i-1})\) \((\alpha_i \in (t_i, t_{i-1}))(\forall i)\) Por teorema del valor medio. \\
    Si \(m_i^{\prime} \) y \(M_i^{\prime} \) son el ínfimo y supremo de \(F^{\prime} \) en \([t_i, t_{i-1}] \Rightarrow m_i^{\prime} \leq \alpha_i \leq M_i^{\prime} \Rightarrow \)\
    \begin{align*}
      s(F^{\prime}, P) \leq F(b) - F(a) \leq S(F^{\prime}, P) \Rightarrow F(b) - F(a) = \int_a^b F^{\prime}(t) \, \mathrm{d}t = \int_a^b f(x) \, \mathrm{d}x
    \end{align*}
  \end{proof}
\end{corollary}

\begin{definition}
  Sea \(P = \{t_0, \ldots, t_n\} \) una partición de \([a, b]\), llamamos norma de una partición (\(|P|\)) a la mayor longitud de \(|t_i - t_{i-1}|\)  de los intervalos de \(P\).
\end{definition}

\begin{theorem}
  Sea \(f: [a, b] \to \R \) acotada \(\Rightarrow (\forall \e > 0)(\exists \delta > 0) \): si \(|P| < \delta \Rightarrow S(f, P) < \upint_a^b f(x) \, \mathrm{d}x + \e \)
  \begin{proof}
    Supongamos que \(f\) es positiva en \([a, b]\). Dado \(\e > 0\), \(\exists P_0 = \{t_0, \ldots, t_n\} \) partición de \([a, b]\) tal que \(S(f, P_0) < \int_a^b f(x) \, \mathrm{d}x + \frac{\e}{2} \). \\
    Sea \(M\) el supremo de \(f\) en \([a, b]\) y \(\delta : 0 < \delta < \frac{\e}{2nM} \). Si \(P\) es una partición de norma menor que \(\delta \). Llamo \([r_{\alpha - 1}, r_{\alpha}]\) a los intervalos de \(P\) contenidos en algún intervalo \([t_{i-1}, t_i]\) de \(P_0\) y \([r_{\beta - 1}, r_{\beta}]\) a los demás. Cada uno de estos intervalos contienen al menos un punto \(t_i\) en su interior, así que son como mucho \(n\). Entonces \(S(f, P) = \sum_{\alpha} M_{\alpha} (r_{\alpha} - r_{\alpha - 1}) + \sum_{\beta} M_{\beta} (r_{\beta} - r_{\beta - 1}) \leq \sum_{i = 1}^n M_i (t_i - t_{i-1}) + M \delta n\).
    Si \([r_{\tilde{\alpha}}, r_{\tilde{\alpha}-1}] \subset [t_{i-1}, t_i]\) con \(i\) fijo \begin{align*}
      \sum_{\tilde{\alpha}} M_{\tilde{\alpha}} (r_{\tilde{\alpha}} - r_{\tilde{\alpha}-1}) \leq M_i (t_i - t_{i-1}) \Rightarrow
    \end{align*}
    \begin{align*}
      S(f, P) < S(f, P_0) + \frac{\e}{2} < \upint_a^b f(x) \, \mathrm{d}x + \e
    \end{align*}
    Para el caso general (sin pedir \(f\) positiva, como \(f\) es acotada \(\exists c \in \R \) tal que \(f(x) + c \geq 0\) \((\forall x \in [a, b])\), si \(g(x) = f(x) + c \Rightarrow S(g, P) = S(f, P) + c (b-a)\) y además \(\upint_a^b g(x) \, \mathrm{d}x = \upint_a^b f(x) \, \mathrm{d}x + c (b-a)\) y se reduce al caso anterior.
  \end{proof}
\end{theorem}


\begin{corollary}
  Si \(f: [a, b] \to \R \) es acotada, \(\int_a^b f(x) \, \mathrm{d}x = \lim_{|P| \to 0} S(f, P)\). \\
  Si \(f: [a, b] \to \R \) es acotada, \(\int_a^b f(x) \, \mathrm{d}x = \lim_{|P| \to 0} s(f, P)\).
  \begin{proof}
    \(-s(f, P) = S(-f, P) \Rightarrow - \lowint_a^b f(x)\, \mathrm{d}x = \upint_a^b -f(x) \, \mathrm{d}x\).
  \end{proof}
\end{corollary}

\begin{definition}[Partición punteada]
  Es una partición \(P = \{t_0, \ldots, t_n\} \) en la que elegimos un punto \(\alpha_i\) en cada \([t_{i-1}, t_i]\) lo anotamos \(P^*\).
\end{definition}

\begin{definition}
  Si \(f: [a,b] \to \R \) es una función acotada y \(P^*\) una partición punteada, definimos la suma de Riemann de \(f\) asociada a \(P^*\) como \(\sum (f, P^*) = \sum_{i = 1}^n f(\alpha_i) (t_i - t_{i-1})\).
\end{definition}

\begin{definition}
  Dada \(f: [a, b] \to \R \) acotada, decimos que \(I = \lim_{|P| \to 0} \sum (f, P^*)\) si \(\forall \e > 0\), \(\exists \delta > 0 : | \sum (f, P^*) - I| < \e \) \((\forall P^*) : |P^*| < \delta \).
\end{definition}

\begin{theorem}
  \(f: [a, b] \to \R \) acotada \(\Rightarrow \exists I = \lim_{|P| \to 0} \sum (f, P^*) \iff f\) es integrable y en ese caso \(I = \int_a^b f(x) \, \mathrm{d}x\).
  \begin{proof}
    Si \(f\) es integrable, por el corolario anterior \begin{align*}
      \lim_{|P| \to 0} s(f, P) = \lim_{|P| \to 0} S(f, P) = \int_a^b f(x) \, \mathrm{d}x
    \end{align*}
    Como \begin{align*}s(f, P) \leq \sum(f, P^*) \leq S(f, P) \Rightarrow \lim_{|P| \to 0} \sum(f, P^*) = \int_a^b f(x) \, \mathrm{d}x \end{align*}
    Recíprocamente supongamos que el límite existe \(\Rightarrow (\forall \e > 0) \exists P = \{ t_0, \ldots, t_n \} : |\sum (f, P^*) - I| < \frac{\e}{4} \) cualquiera sea la forma de puntear la partición \(P\). \\
    Fijemos \(P\) y elijamos dos formas de puntearla, para la primera forma elijamos \(\alpha_i : f(\alpha_i) < m_i + \frac{\e}{4 \cdot n \cdot (t_i - t_{i-1})} \). Esto me da \(P^*\) tal que \(\sum (f, P^*) = \sum_{i = 1}^n f(\alpha_i) (t_i - t_{i-1}) < \sum_{i = 1}^n m_i (t_i - t_{i-1}) + \frac{\e}{4} = s(f, P) + \frac{\e}{4} \). \\
    Análogamente, elegimos la segunda \(\tilde{P} : S(f, P) - \frac{\e}{4} < \sum (f, \tilde{P}) \Rightarrow \)
    \begin{align*}
      \sum (f, \tilde{P}) - \frac{\e}{4} < s(f, P) \leq S(f, P) < \sum (f, \tilde{P}) + \frac{\e}{4}
    \end{align*}
    \(\Rightarrow s(f, P)\) y \(S(f, P)\) están en \((I - \frac{\e}{2}, I + \frac{\e}{2}) \Rightarrow S(f, P)- s(f, P) < \e \therefore f\) es integrable y \(\int_a^b f(x) \, \mathrm{d}x = I\).
  \end{proof}
\end{theorem}

\begin{eg}
  Sea \(f: [1, 2] \to \R \), \(f(x) = \frac{1}{x} \). Sabemos que \begin{align*}\int_1^2 \frac{1}{x} \, \mathrm{d}x = \ln(2) - \ln(1) = \ln(2)\end{align*} Para cada \(n \in \N \), \(P_n = \{1, \frac{n+1}{n}, \frac{n+2}{n}, \ldots, \frac{2 \cdot n}{n}\} \). En cada intervalo \([\frac{n+i-1}{n}, \frac{n+i}{n}]\), \(i = 1, \ldots, n\). Elijo \(\alpha_i = \frac{n+i}{n} \). Luego
  \begin{align*}
     & f(\alpha_i) = f(\frac{n+i}{n}) = \frac{n}{n+i} \text{, y } t_i - t_{i-1} = \frac{1}{n}                                                                \\
     & \Rightarrow \sum_{i = 0}^n f(\alpha_i) (t_i - t_{i-1}) = \sum(f, P^*) = \sum_{i = 0}^n \frac{n}{n+i} \cdot \frac{1}{n} = \sum_{i = 0}^n \frac{1}{n+i} \\
     & \Rightarrow \lim_{n \to +\infty} \frac{1}{n+1} + \cdots + \frac{1}{2n} = \ln(2)
  \end{align*}
\end{eg}

\section{Integral de Riemann en varias dimensiones}

\begin{itemize}
  \item Sean \(I_1, \cdots I_n\) intervalos de \(\R \) el conjunto \(I \subset \R^n\) dado por \(I_1 \times \cdots \times I_n = I\). Se llama intervalo n-dimensional.
  \item Se admite el caso degenerado en que uno o más de los \(I_k\) se reduce a un solo punto.
  \item Si todos los \(I_k\) son abiertos o cerrados o acotados, entonces \(I\) tiene la misma propiedad en \(\R^n\).
  \item Notamos con \(|I|\) a la medida de los intervalos, con la siguiente propiedad \(|I| = |I_1| \cdots |I_n|\) con \(|I_k|\) la longitud de \(I_k\).
  \item Si \(I\) es compacto de \(\R^n\) y \(P_k\) es una partición de \(I_k \Rightarrow P = P_1 \times \cdots \times P_n\) es una partición de \(I\).
  \item Una partición \(P^{\prime} \) se dice más fina que \(P\) si \(P \subset P^{\prime} \).
\end{itemize}

\begin{definition}
  Sea \(f\) acotada en un intervalo compacto. Si \(P\) es una partición de \(I\) que define \(n\) subintervalos \(I_1, \ldots, I_n \subset \R^n\) y \(\alpha_k \in I_k\), una suma de Riemann de \(f\) asociada a \(P\) es una suma de la forma \(\sum (f, P^*) = \sum_{k = 1}^n f(\alpha_k) |I_k|\). \\
  Decimos que \(f\) es integrable Riemann en \(I\) si \(\exists A \in R : (\forall \e > 0)(\exists P_{\e})\) partición de \(I : (\forall P)\) más fina que \(P_{\e} \) se tiene que \(|\sum(f, P^*) - A| < \e \).
  Notamos \begin{align*}
    A =  \int_I f(x) \, \mathrm{d}x = \int \int \cdots \int_I f(x_1, \ldots, x_n)\, \mathrm{d}x_1 \cdots \mathrm{d}x_n
  \end{align*}
\end{definition}

\begin{definition}
  Sea \(I \subset \R^n\) un intervalo compacto, si \(P\) es una partición de \(I\) que define \(n\) subintervalos \(I_1, \cdots I_n \subset \R^n\) definimos \(m_k = m_k(f) = \inf \{ f(x) : x \in I_k\} \) y \(M_k = M_k(f) = \sup \{ f(x) : x \in I_k\} \).
  Las sumas inferiores y superiores son \(s(f, P) = \sum m_k |I_k|\) y \(S(f, P) = \sum M_k |I_k|\).
  La integral superior e inferior de Riemann son \begin{align*}
    \upint_I f(x) \, \mathrm{d}x = \inf(S(f, P))
  \end{align*}
  \begin{align*}
    \lowint_I f(x)\, \mathrm{d}x = \sup(s(f, P))
  \end{align*}
  Mantienen las propiedades de \(\R \).
\end{definition}

\begin{theorem}
  \(I \subset \R^n\) compacto, son equivalentes:
  \begin{enumerate}
    \item \(f\) es integrable en \(I\).
    \item \((\forall \e > 0)(\exists P_{\e})\) de \(I : P\) refina a \(P_{\e} \) \(S(f, P) - s(f, P) < \e \).
    \item \(\lowint_I f(x) \, \mathrm{d}x = \upint_I f(x) \, \mathrm{d}x\).
  \end{enumerate}
  \begin{proof}
    Ejercicio, es equivalente a las demostraciones en \(\R \).
  \end{proof}
\end{theorem}

\begin{theorem}
  Sea \(f\) definida en un compacto \(K = [a, b] \times [c, d] \subset \R^2 \Rightarrow \)
  \begin{align*}
    \int \lowint_K f(x, y) \, \mathrm{d}x \mathrm{d}y & \leq \lowint_a^b(\upint_c^d f(x, y)\, \mathrm{d}y)\mathrm{d}x                                                       \\
                                                      & \leq \upint_a^b \upint_c^d f(x, y) \, \mathrm{d}x \mathrm{d}y \leq \int \upint_K f(x, y) \, \mathrm{d}x \mathrm{d}y
  \end{align*}
  \begin{proof}
    Si \begin{align}
       & F(x) = \upint_c^d f(x, y) \, \mathrm{d}y \quad (\forall x \in [a, b])                                                   \\
       & \Rightarrow |F(x)| \leq M (d - c) \quad M = \sup_Q f                                                                    \\
       & \text{ Llamo } \overline{I} = \upint_a^b F(x)\, \mathrm{d}x = \upint_a^b(\upint_c^d f (x, y)\, \mathrm{d}y) \mathrm{d}x \\
       & \text{ y } \underline{I} = \lowint_a^b F(x)\, \mathrm{d}x = \lowint_a^b(\lowint_c^d f (x, y)\, \mathrm{d}y) \mathrm{d}x
    \end{align}
    Si \(P_1 = \{x_0, \ldots, x_n\} \) partición de \([a, b]\), \(P_2 = \{ y_0, \ldots, y_m \} \) partición de \([c, d]\). \(P_1 \times P_2\) es una partición de \(Q\) con \(n \cdot m\) intervalos \(Q_{ij} \). Luego \begin{align*}
      \overline{I}_{ij} = \upint_{x_{i-1}}^{x_i}( \upint_{y_{j-1}}^{y_j}f(x, y)\,\mathrm{d}y)\mathrm{d}x \text{ y }
      \underline{I}_{ij} = \lowint_{x_{i-1}}^{x_i}( \lowint_{y_{j-1}}^{y_j}f(x, y)\,\mathrm{d}y)\mathrm{d}x
    \end{align*} Como \(\upint_c^d f(x, y)\, \mathrm{d}y = \sum_{j = 1}^m \upint_{y_{j-1}}^{y_j} f(x, y)\, \mathrm{d}y \Rightarrow \) \begin{align*}
      \upint_a^b\left(\upint_c^d f(x, y)\, \mathrm{d}y\right)\mathrm{d}x & = \upint_a^b\left( \sum_{j = 1}^m \int_{y_{j-1}}^{y_j}f(x, y)\, \mathrm{d}y\right)\mathrm{d}x                                                                                                                         \\
                                                                         & \leq \sum_{j = 1}^m \upint_a^b\left(\int_{y_{j-1}}^{y_j}f(x, y)\, \mathrm{d}y\right)\mathrm{d}x = \sum_{j = 1}^m \sum_{i = 1}^n \upint_{x_{i-1}}^{x_i}\left(\int_{y_{j-1}}^{y_j}f(x,y)\,\mathrm{d}y\right)\mathrm{d}x \\
                                                                         & \Rightarrow  \overline{I} \leq \sum_{j = 1}^m \sum_{i = 1}^n \overline{I_{ij}} \text{ y,}                                                                                                                             \\
                                                                         & \leq \sum_{j = 1}^m \sum_{i = 1}^n \overline{I_{ij}} \text{ y, } \underline{I} \geq \sum_{j = 1}^m \sum_{i = 1}^n \underline{I_{ij}}
    \end{align*}
    Si escribimos \(m_{ij} = \inf \{ f(x, y) : (x, y) \in Q \} \) y \(M_{ij} = \sup \{ f(x, y) : (x, y) \in Q\} \Rightarrow \) \begin{align*}
      m_{ij} (y_j - y_{j-1}) & \leq \upint_{y_{j-1}}^{y_j} f(x, y)\, \mathrm{d} y \leq M_{ij} (y_j - y_{j-1})                                \\
      m_{ij} |Q_{ij}|        & \leq \lowint_{x_{i-1}}^{x_i} ( \upint_{y_{j-1}}^{y_j} f(x, y)\, \mathrm{d}y )\mathrm{d}x \leq M_{ij} |Q_{ij}| \\
                             & \Rightarrow s(f, P) \leq \underline{I} \leq \overline{I} \leq S(f, P)
    \end{align*}
    Como esto vale para cualquier partición \(P\) de \(Q \Rightarrow \) \begin{align*}
      \int\lowint_Q f \leq \underline{I} \leq \overline{I} \leq \int\upint_Q f
    \end{align*}
  \end{proof}
\end{theorem}

\clearpage

\begin{note}
  También tenemos que

  \begin{enumerate}
    \item Vale reemplazando \(\upint_c^d\) por \(\lowint_c^d\).
    \item \(\int \lowint_K f(x, y) \, \mathrm{d}x \mathrm{d}y \leq \lowint_c^d ( \upint_a^b f(x, y)\, \mathrm{d}x ) \mathrm{d}y \leq \upint_c^d \upint_a^b f(x, y) \, \mathrm{d}x \mathrm{d}y \leq \int \upint_K f(x, y) \, \mathrm{d}x \mathrm{d}y\)
    \item Valen reemplazando \(\upint_a^b\) por \(\lowint_a^b\).
    \item Si \(\exists \int \int_K f(x, y) \, \mathrm{d}x \mathrm{d}y \Rightarrow \) \begin{align*}
            \int \int_K f(x, y) \, \mathrm{d}x \mathrm{d}y & = \int_a^b(\lowint_c^d f(x, y) \, \mathrm{d}y)\mathrm{d}x                                                           \\
                                                           & = \int_c^d \lowint_a^b f(x, y) \, \mathrm{d}x \mathrm{d}y & = \int_c^d(\upint_a^b f(x,y) \, \mathrm{d}x)\mathrm{d}y
          \end{align*}
  \end{enumerate}

  La demostración de 1) Es análoga \(F(x) = \lowint_c^d f(x, y) \, \mathrm{d}y\), (2) y (3) Análogo cambiando \(x\) por \(y\) y (4) es consecuencia de las anteriores.
\end{note}

\section{Teorema de Fubini}
\begin{corollary}[Fubini]
  Si \(f\) es continua en \(Q\) \begin{align*}
    \int\int_Q f = \int_a^b(\int_c^d f(x, y) \, \mathrm{d}x) \mathrm{d}y = \int_c^d(\int_a^b f(x, y)\, \mathrm{d}x)\mathrm{d}y
  \end{align*}
\end{corollary}

\begin{definition}
  Sea \(S\) un subconjunto de un intervalo compacto \(I \subset \R^n\). Para cada partición \(P\) de \(I\) definimos \(\underline{J}(S, P)\) como la suma de las medidas de los subintervalos definidos por la partición que solo contienen los puntos interiores de \(S\) y la medida \(\overline{J}(S, P)\) como la suma de las medidas de los intervalos que contienen puntos de \(\overline{S} \).
\end{definition}

\begin{definition}
  \(\underline{c}(S) = \sup_P \underline{J}(S, P)\) contenido de Jordan del interior de \(S\).
  \(\overline{c}(S) = \inf_P \overline{J}(S, P)\) contenido de Jordan superior.
\end{definition}

\clearpage

\begin{definition}[Contenido de Jordan]
  Si \(\overline{c}(S) = \underline{c}(S)\) notamos \(c(S)\) y se llama contenido de Jordan. Decimos que \(S\) es medible Jordan.
\end{definition}

\begin{note}
  \begin{enumerate}
    \item Se puede ver que \(\overline{c}(S)\) y \(\underline{c}(S)\) solo dependen de \(S\) y no de \(I\).
    \item \(0 \leq \underline{c}(S) \leq \overline{c}(S)\).
    \item Si \(S\) es tal que \(c(S) = 0 \Rightarrow \underline{c}(S) = \overline{c}(S) = 0\). Así que \(S\) se puede cubrir con finitos intervalos cuyas medidas suman menos que \(\e \), \(\forall \e > 0\).
  \end{enumerate}
\end{note}
