\section{Teorema fundamental del cálculo}

\begin{theorem}[Teorema Fundamental del Cálculo]
  Sea $f: [a, b] \to \R$ integrable. Si $f$ es continua en $c \in [a, b] \Rightarrow F: [a, b] \to \R$, $F(x) = \int_a^x f(t)\, \mathrm{d}t$ es derivable en $c$ y $f(x) = F^{\prime}(c)$.
  \begin{proof}
    \begin{equation}
      | \dfrac{F(c+h) - F(c)}{h} - f(c) | = |\dfrac{1}{h}| \cdot | \int_c^{c+h} f(t)\, \mathrm{d}t - h \cdot f(c) |
    \end{equation}
    \begin{equation}
      \leq \dfrac{1}{h} \cdot \int_c^{c+h}| f(t) - f(c) | \, \mathrm{d}t *
    \end{equation}
    Como $f$ es continua es continua en $c \Rightarrow$ Dado $\e > 0$, $\exists \delta > 0 : t \in [a, b]$ y $|t-c| < \delta \Rightarrow |f(t) - f(c)| < \e$.
    Si $0 < h < \delta$ y $c+h \in [a, b]$ tenemos que $* \leq \dfrac{1}{h} \cdot \e \cdot h = \e \Rightarrow$ \begin{equation}
      \lim_{h \to 0^+} \dfrac{F(c+h) - F(c)}{h} = f(c)
    \end{equation}
    Para $h < 0$ es análogo, luego $\lim_{h \to 0^-} \dfrac{F(c+h) - F(c)}{h} = f(c) \Rightarrow F^{\prime}(c) = f(c)$.
  \end{proof}
\end{theorem}

\begin{corollary}[Regla de Barrow]
  $f: [a, b] \to \R$ integrable y tiene una primitiva $F: [a, b] \to \R \Rightarrow \int_a^b f(x) \, \mathrm{d}x = F(b) - F(a)$.
  \begin{proof}
    Para cualquier partición de $[a, b]$, $P = \{ t_0, \cdots, t_n \} \Rightarrow F(b) - F(a) = \sum_{i = 1}^n (F(t_i) - F(t_{i-1})) = \sum_{i = 1}^n F(\alpha_i) (t_i - t_{i-1})$ $(\alpha_i \in (t_i, t_{i-1}))(\forall i)$ Por teorema del valor medio. \\
    Si $m_i^{\prime}$ y $M_i^{\prime}$ son el ínfimo y supremo de $F^{\prime}$ en $[t_i, t_{i-1}] \Rightarrow m_i^{\prime} \leq \alpha_i \leq M_i^{\prime} \Rightarrow$\
    \begin{equation}
      s(F^{\prime}, P) \leq F(b) - F(a) \leq S(F^{\prime}, P) \Rightarrow F(b) - F(a) = \int_a^b F^{\prime}(t) \, \mathrm{d}t = \int_a^b f(x) \, \mathrm{d}x
    \end{equation}
  \end{proof}
\end{corollary}

\begin{definition}
  Sea $P = \{t_0, \cdots, t_n\}$ una partición de $[a, b]$, llamamos norma de una partición ($|P|$) a la mayor longitud de $|t_i - t_{i-1}|$  de los intervalos de $P$.
\end{definition}

\begin{theorem}
  Sea $f: [a, b] \to \R$ acotada $\Rightarrow (\forall \e > 0)(\exists \delta > 0) : $ si $|P| < \delta \Rightarrow S(f, P) < \upint_a^b f(x) \, \mathrm{d}x + \e$ 
  \begin{proof}
    Supongamos que $f$ es positiva en $[a, b]$. Dado $\e > 0$, $\exists P_0 = \{t_0, \cdots, t_n\}$ partición de $[a, b]$ tal que $S(f, P_0) < \int_a^b f(x) \, \mathrm{d}x + \frac{\e}{2}$. \\
    Sea $M$ el supremo de $f$ en $[a, b]$ y $\delta : 0 < \delta < \frac{\e}{2nM}$. Si $P$ es una partición de norma menor que $\delta$. Llamo $[r_{\alpha - 1}, r_{\alpha}]$ a los intervalos de $P$ contenidos en algún intervalo $[t_{i-1}, t_i]$ de $P_0$ y $[r_{\beta - 1}, r_{\beta}]$ a los demás. Cada uno de estos intervalos contienen al menos un punto $t_i$ en su interior, así que son como mucho $n$. Entonces $S(f, P) = \sum_{\alpha} M_{\alpha} (r_{\alpha} - r_{\alpha - 1}) + \sum_{\beta} M_{\beta} (r_{\beta} - r_{\beta - 1}) \leq \sum_{i = 1}^n M_i (t_i - t_{i-1}) + M \delta n$.
    Si $[r_{\tilde{\alpha}}, r_{\tilde{\alpha}-1}] \subset [t_{i-1}, t_i]$ con $i$ fijo \begin{equation}
      \sum_{\tilde{\alpha}} M_{\tilde{\alpha}} (r_{\tilde{\alpha}} - r_{\tilde{\alpha}-1}) \leq M_i (t_i - t_{i-1}) \Rightarrow
    \end{equation}
    \begin{equation}
      S(f, P) < S(f, P_0) + \frac{\e}{2} < \upint_a^b f(x) \, \mathrm{d}x + \e
    \end{equation}
    Para el caso general (sin pedir $f$ positiva, como $f$ es acotada $\exists c \in \R$ tal que $f(x) + c \geq 0$ $(\forall x \in [a, b])$, si $g(x) = f(x) + c \Rightarrow S(g, P) = S(f, P) + c (b-a)$ y además $\upint_a^b g(x) \, \mathrm{d}x = \upint_a^b f(x) \, \mathrm{d}x + c (b-a)$ y se reduce al caso anterior.
  \end{proof}
\end{theorem}


\begin{corollary}
  Si $f: [a, b] \to \R$ es acotada, $\int_a^b f(x) \, \mathrm{d}x = \lim_{|P| \to 0} S(f, P)$. \\
  Si $f: [a, b] \to \R$ es acotada, $\int_a^b f(x) \, \mathrm{d}x = \lim_{|P| \to 0} s(f, P)$. 
  \begin{proof}
    $-s(f, P) = S(-f, P) \Rightarrow - \lowint_a^b f(x)\, \mathrm{d}x = \upint_a^b -f(x) \, \mathrm{d}x$.
  \end{proof}
\end{corollary}

\begin{definition}[Partición punteada]
  Es una partición $P = \{t_0, \cdots, t_n\}$ en la que elegimos un punto $\alpha_i$ en cada $[t_{i-1}, t_i]$ lo anotamos $P^*$.
\end{definition}

\begin{definition}
  Si $f: [a,b] \to \R$ es una función acotada y $P^*$ una partición punteada, definimos la suma de Riemann de $f$ asociada a $P^*$ como $\sum (f, P^*) = \sum_{i = 1}^n f(\alpha_i) (t_i - t_{i-1})$. 
\end{definition}

\begin{definition}
  Dada $f: [a, b] \to \R$ acotada, decimos que $I = \lim_{|P| \to 0} \sum (f, P^*)$ si $\forall \e > 0$, $\exists \delta > 0 : | \sum (f, P^*) - I| < \e$ $(\forall P^*) : |P^*| < \delta$.
\end{definition}

\begin{theorem}
  $f: [a, b] \to \R$ acotada $\Rightarrow \exists I = \lim_{|P| \to 0} \sum (f, P^*) \iff f$ es integrable y en ese caso $I = \int_a^b f(x) \, \mathrm{d}x$. 
  \begin{proof}
    Si $f$ es integrable, por el corolario anterior \begin{equation}
      \lim_{|P| \to 0} s(f, P) = \lim_{|P| \to 0} S(f, P) = \int_a^b f(x) \, \mathrm{d}x
    \end{equation}
    Como \begin{equation}s(f, P) \leq \sum(f, P^*) \leq S(f, P) \Rightarrow \lim_{|P| \to 0} \sum(f, P^*) = \int_a^b f(x) \, \mathrm{d}x \end{equation}
    Recíprocamente supongamos que el límite existe $\Rightarrow (\forall \e > 0) \exists P = \{ t_0, \cdots, t_n \} : |\sum (f, P^*) - I| < \frac{\e}{4}$ cualquiera sea la forma de puntear la partición $P$. \\
    Fijemos $P$ y elijamos dos formas de puntearla, para la primera forma elijamos $\alpha_i : f(\alpha_i) < m_i + \frac{\e}{4 \cdot n \cdot (t_i - t_{i-1})}$. Esto me da $P^*$ tal que $\sum (f, P^*) = \sum_{i = 1}^n f(\alpha_i) (t_i - t_{i-1}) < \sum_{i = 1}^n m_i (t_i - t_{i-1}) + \frac{\e}{4} = s(f, P) + \frac{\e}{4}$. \\
    Análogamente, elegimos la segunda $\tilde{P} : S(f, P) - \frac{\e}{4} < \sum (f, \tilde{P}) \Rightarrow$
    \begin{equation}
      \sum (f, \tilde{P}) - \frac{\e}{4} < s(f, P) \leq S(f, P) < \sum (f, \tilde{P}) + \frac{\e}{4}
    \end{equation}
    $\Rightarrow s(f, P)$ y $S(f, P)$ están en $(I - \frac{\e}{2}, I + \frac{\e}{2}) \Rightarrow S(f, P)- s(f, P) < \e \therefore f$ es integrable y $\int_a^b f(x) \, \mathrm{d}x = I$.
  \end{proof}
\end{theorem}

\begin{eg}
  Sea $f: [1, 2] \to \R$, $f(x) = \frac{1}{x}$. Sabemos que \begin{equation}\int_1^2 \frac{1}{x} \, \mathrm{d}x = ln(2) - ln(1) = ln(2)\end{equation} Para cada $n \in \N$, $P_n = \{1, \frac{n+1}{n}, \frac{n+2}{n}, \cdots, \frac{2 \cdot n}{n}\}$. En cada intervalo $[\frac{n+i-1}{n}, \frac{n+i}{n}]$, $i = 1, \cdots, n$. Elijo $\alpha_i = \frac{n+i}{n}$. Luego 
  \begin{equation}
    f(\alpha_i) = f(\frac{n+i}{n}) = \frac{n}{n+i} \text{, y } t_i - t_{i-1} = \frac{1}{n}
  \end{equation}
  \begin{equation}
    \Rightarrow \sum_{i = 0}^n f(\alpha_i) (t_i - t_{i-1}) = \sum(f, P^*) = \sum_{i = 0}^n \frac{n}{n+i} \cdot \frac{1}{n} = \sum_{i = 0}^n \frac{1}{n+i}
  \end{equation} y
  \begin{equation}
    \lim_{n \to +\infty} \frac{1}{n+1} + \cdots + \frac{1}{2n} = ln(2)
  \end{equation}
\end{eg}

\section{Integral de Riemann en varias dimensiones}

\section{Teorema de Fubini}
