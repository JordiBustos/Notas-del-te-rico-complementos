\section{Ejemplo de medida de Jordan}

\begin{eg}
  $X = \Q \cap [a, b]$ con $a < b$ no tiene contenido nulo.
  \begin{proof}
    Si fuese $c(X) = 0$, Dado $0 < \e < b-a$ $\exists$ partición $P$ de $[a, b]$ tal que la suma de las longitudes de los intervalos de $P$ que contienen puntos de $X$ sería $< \e$.
    Como la suma de todas las longitudes de todos los intervalos de $P$ tiene que ser $b-a$ alguno tendría puntos racionales, absurdo!
  \end{proof}
\end{eg}

\begin{lemma}
$X \subset [a, b]$ Si $\forall \e > 0$ $\exists$ abiertos $I_1$, $\cdots$, $I_k$ y un subconjunto finito $F \subset X : X - F = I_1 \cup \cdots \cup I_k$ y $|I_1| + \cdots + |I_k| \mapsto \e$.
  \begin{proof}
    Dado $\e > 0$ $\exists F \subset X$ finito en intervalos abiertos $I_1$, $\cdots$, $I_k$ tales que $\sum_{i = 1}^k |I_i| < \frac{\e}{2}$ y $X - F \subset I_1 \cup \cdots \cup I_k$. Por otro lado,
    podemos encontrar intervalos abiertos $I_{k+1}$, $\cdots$, $I_n : F \subset I_{k+1} \cup \cdots \cup I_n$ y $\sum_{j = k+1}^n |I_j| < \frac{\e}{2} \Rightarrow X \subset I_1 \cup \cdots \cup I_n$ y $\sum_{i = 1}^n |I_i| < \e \therefore c(X) = 0$. 
  \end{proof}
\end{lemma}

\begin{eg}
  El conjunto de Cantor tiene contenido nulo.
  \begin{proof}
    Por el lema anterior vale que $C \subset F_n$ $\forall n \in \N$ y la suma de las longitudes de los intervalos de $F_n$ es $1 - \frac{1}{3} \sum_{i = 0}^n (\frac{2}{3})^i = (\frac{2}{3})^{n+1} < \e$ $\forall n > n_0 \Rightarrow c(C) = 0$.
  \end{proof}
\end{eg}

\begin{theorem}
  $S \subset \R^n \Rightarrow \overline{c}(\partial S) = \overline{c}(S) - \underline{c}(S)$. En particular $S$ es medible Jordan $\iff \partial S$ tiene contenido de Jordan nulo. 
  \begin{proof}
    Sea $I \subset \R^n$ un intervalo compacto que contiene a $\overline{S} \Rightarrow$ Para toda partición $P$ de $I$, $J(\partial S, P) = \overline{J}(S, P) - \underline{J}(S, P) \Rightarrow \overline{J}(\partial S, P) \geq \overline{c}(S) - \underline{c}(S) \Rightarrow \overline{c}(\partial S) \geq \overline{c}(S) - \underline{c}(S)$. Para la desigualdad opuesta, dado $\e > 0$ sea $P_1$ tal que $\overline{J}(S, P_1) < \overline{c}(S) + \frac{\e}{2}$ y sea $P_2$ tal que $\underline{J}(S, P_2) > \underline{c}(S) - \frac{\e}{2}$. Si $P = P_1 \cup P_2$, como refinar aumenta las sumas interiores y achica las exteriores \begin{equation}
      \overline{c}(\partial S) \leq \overline{J}(\partial S, P) = \overline{J}(S, P) - \underline{J}(S, P) \leq \overline{J}(S, P_1) - \underline{J}(S, P_2) \leq \overline{c}(S) - \underline(c)(S) + \e
    \end{equation}
    Como vale $\forall \e > 0 \Rightarrow \overline{c}(\partial S) \leq \overline{c}(S) - \underline{c}(S)$.
  \end{proof} y juntando las dos desigualdades se tiene que $\overline{\partial S} = \overline{c}(S) - \underline{c}(S)$.
\end{theorem}

\begin{definition}[Medida de Lebesgue]
  Un conjunto $S \subset \R^n$ tiene medida de Lebesgue $0$ si $\forall \e > 0$ $S$ se puede cubrir por a lo sumo numerables intervalos cuyas medidas suman menos que $\e$. Notamos $|S| = 0$. 
\end{definition}

\begin{eg}
  Si $c(X) = 0 \Rightarrow |X| = 0$, por ejemplo el conjunto de Cantor.
\end{eg}

\begin{eg}
  $|\Q| = 0$, más aún, toda unión numerable de conjuntos de medida nula, tiene medida nula.
  \begin{proof}
    Numero a los racionales, digamos $\Q = \{r_1, r_2, \cdots\}$ y dado $\e > 0$ tomo para uno el intervalo $I_k = (r_k - \frac{\e}{2^{k+1}}, r_k + \frac{\e}{2^{k+1}}) \Rightarrow \Q \subset \bigcup_{k + 1}^{+\infty} I_k$ y $\sum_{k = 1}^{+\infty} |I_k| = \sum_{k = 1}^{+\infty} \frac{\e}{2^k} = \e$.
    Si ahora $F = \{F_1, F_2, \cdots\}$ son conjuntos de medida nula, dado $\e > 0$ $\forall k \in \N$ puedo elegir un cubrimiento por a lo sumo numerables intervalos abiertos tales que la suma de sus medidas sea menor que $\frac{\e}{2^k} \Rightarrow$ la unión de todos los intervalos es numerable y la suma de todas las medidas es menor que $\e$. 
  \end{proof}
\end{eg}

\section{Integración en conjuntos}
