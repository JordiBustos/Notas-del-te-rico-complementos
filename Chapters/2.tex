\section{Cuerpo Arquimediano}

Si $\mathbb{K}$ es un cuerpo ordenado, $\mathbb{N} \subset \mathbb{Z} \subset \mathbb{Q} \subset \mathbb{K}$, pero esto no necesariamente implica que $\mathbb{N}$ es no acotado. 

\begin{eg}
    $\mathbb{Q}(t):$ cuerpo de funciones racionales con coeficientes en $\mathbb{Q}$, $r(t) = p(t)/q(t)$, $p, q \in \mathbb{Q}, q \neq 0$. Este cuerpo puede ser ordenado diciendo que $r(t)$ es positivo si y sólo si el coeficiente de mayor grado del polinomio $pq$ es positivo.
    En este cuerpo observemos que $p(t)=t=t/1$ cumple que $\forall n \in \mathbb{N}, p(t)=t-n \in P \Rightarrow t>n, \forall n \in \mathbb{N}$. Es decir que en $\mathbb{Q}(t)$, $\mathbb{N}$ es un conjunto acotado, por ejemplo por $t$.
\end{eg}

\begin{theorem}
    En un cuerpo ordenado $\mathbb{K}$ son equivalentes: \\
    1) $\mathbb{N} \subset \mathbb{K}$ no es acotado superiormente. \\ 
    2) Dados $a, b \in \mathbb{K}$ con $a>0, \exists n \in \mathbb{N}:m \cdot a>b$. \\
    3) Dado cualquier $0<a \in \mathbb{K}, \exists n \in \mathbb{N} : 0 < \dfrac{1}a < a$. \\
    Cuando vale cualquiera decimos que $\mathbb{K}$ es arquimediano.
    \begin{proof}
        1) $\Rightarrow$ 2) \\
        Como $\mathbb{N}$ es no acotado, dados $a,b \in \mathbb{K}$, $a>0$, $\exists m \in \mathbb{N} : m > \dfrac ba$, caso contrario $\dfrac ba$ sería cota de $\mathbb{N} \Rightarrow ma>b$ \\
        2) $\Rightarrow$ 3) \\
        Dado $a>0, \exists n \in \mathbb{N} : ma>1$ (tomando $b=1$ en 2.) $\Rightarrow a>\dfrac 1m >0$ \\
        3) $\Rightarrow$ 1) \\
        Dado $0<a \in \mathbb{K} \Rightarrow \forall n \in \mathbb{N}, 0<\dfrac 1n<\dfrac 1a$, pues 3. vale para todo $\mathbb{K} \Rightarrow b<n \Rightarrow$ es no acotado (pues ningún $b>0$ puede ser cota superior).
    \end{proof}
\end{theorem}


\section{Supremo e ínfimo}

\begin{definition}
    Dados un cuerpo ordenado $\mathbb{K}$ y un subconjunto $X \subset \mathbb{K}$ acotado superiormente, decimos que $b \in \mathbb{K}$ es supremo de $X$ si es la menor de las cotas superiores de $X$ en $\mathbb{K}$.
\end{definition}

Es decir, se cumple:

1) $\forall x \in X, x \leq b$ \\
2) Si $c \in \mathbb{K}$ y $x \leq c$, $\forall x \in X \Rightarrow b \leq c$. \\
3) Dado $c < b$ en $\mathbb{K}, \exists x \in X : c<x$.

\begin{note}
    1) El supremo de un conjunto, si existe es único. \\
    2) Si un conjunto tiene máximo, es el supremo. \\
    3) Si $X = \varnothing$, todo $b \in \mathbb{K}$ es cota superior, como $\mathbb{K}$ no tiene primer elemento, se sigue que $\varnothing$ no tiene supremo en $\mathbb{K}$.
\end{note}

\begin{definition}
    Dados un cuerpo ordenado $\mathbb{K}$ y un subconjunto $X \subset \mathbb{K}$ acotado inferiormente, decimos que $b \in \mathbb{K}$ es ínfimo de $X$ si es la mayor de las cotas inferiores de $X$ en $\mathbb{K}$.
\end{definition}

Es decir, se cumple:

1) $\forall x \in X, x \geq b$ \\
2) Si $c \in \mathbb{K}$ y $x \geq c$, $\forall x \in X \Rightarrow b \geq c$. \\
3) Dado $c > b$ en $\mathbb{K}, \exists x \in X : b < x < c$.

\begin{eg}
    Dados $a<b$ en $\mathbb{K}$. Si $X=(a,b) \Rightarrow inf(X)=a$, $sup(X)=b$. \\
    1) Por definición $a$ es cota inferior y $b$ superior. \\
    2) $a<c \in \mathbb{K}$, no es cota inferior. En efecto, si $c \geq b$ trivial. Si $c < b \Rightarrow \dfrac{a+c}{2} \in X, a < \dfrac{a+c}{2} < c \Rightarrow a < c \therefore c$ no es cota inferior. \\
    Luego por 1) y por 2) $a$ es ínfimo de $X$.
\end{eg}

\begin{eg}
    $Y = \{ y \in \mathbb{Q} : y=\dfrac{1}{2^n}, n\in \mathbb{N} \}$. Veamos que $inf(Y) =0, sup(Y) = \dfrac12$. \\
    $\dfrac{1}{2} \in Y, \dfrac{1}{2^n}<\dfrac{1}{2} \forall n \in \mathbb{N} \Rightarrow \dfrac{1}{2} = sup(Y)$. \\
    
    Como $0 < \dfrac{1}{2^n} \forall n \in \mathbb{N} \Rightarrow 0$ es cota inferior. Sea $0<c \in \mathbb{K}$, $2^n = (1+1)^n \leq 1+n \leq \dfrac{1}{c} \Rightarrow n \leq \dfrac{1}{c} -1 \Rightarrow \dfrac{1}{2^n} < c \Rightarrow c$ no puede ser cota inferior por la propiedad 3 de la arquimedianidad $\therefore 0$ es el ínfimo de $Y$. 
\end{eg}

El problema más serio de los racionales desde el punto de vista del análisis es que algunos conjuntos acotados de números racionales no tienen súpremo (o ínfimo) en $\mathbb{Q}$.

\begin{eg}
    Sean $X = \{ X \in \mathbb{Q} : x \geq 0, x² < 2 \}$,
    $Y \{ y \in \mathbb{Q} : y>0, y² > 2 \}$. Notemos que si $z>2 \Rightarrow z² > 4 \Rightarrow z \notin X \Rightarrow X \subset [0, 2]$ y $X$ es un conjunto acotado. Además $Y \subset (0, +\infty)$ por lo que es un conjunto acotado inferiormente. Veamos que $\nexists$ supremo e ínfimo en $\mathbb{Q}$. \\
    
    1) Quiero ver que $X$ no tiene máximo. Dado $x \in X$ quiero encontrar $r \in \mathbb{Q}$ tal que $0<r<1$ y $x+r \in X \iff (x+r)² =x^2+2xr+r² < 2$. Como $r<1 \Rightarrow (x+r)²<x²+2xr+r=x²+r(2x+1)<2 \therefore x+r \in X$. \\
    2) Quiero ver que $Y$ no tiene elemento mínimo, dado $y \in Y$ tomo $r \in \mathbb{Q} : 0 < r < \dfrac{y²-2}{2y}$
    \begin{equation}
        (y-r)²=y²-2yr+r²>y²-2yr>2
    \end{equation}
    \begin{equation}
        y²-2 > 2yr
    \end{equation}
    \begin{equation}
        \dfrac{y²-2}{2y}>r
    \end{equation}
    Es decir que $y-r \in Y$ e $y-r < y$ \\
    
    3) Si $x \in X, y \in Y \Rightarrow x < y$
    $x² < 2 < y² \Rightarrow x² < y²$. \\

    Veamos que por 1, 2, 3 $\nexists sup(X), inf(Y)$. Supongamos $0< \alpha = sup(X)$, no puede ser $\alpha²<2$ porque si no $\alpha \in X$ y $X$ no tiene máximo. Tampoco puede ser $\alpha² > 2$ pues estaría en $Y$ e $Y$ no tiene mínimo, pues habría un $\beta \in Y$ con $\beta < \alpha$ y por 3) sería $x < \beta < \alpha, \forall x \in X$ lo que contradice que $sup(X) = \alpha$, pues sería $\beta$ el supremo.
    En definitiva si existiese $sup(X) = \alpha$, debe ser $\alpha² = 2 \notin \mathbb{Q}$. Luego $X$ no tiene supremo en $\mathbb{Q}$.
    Ínfimo, ejercicio (análogo).
\end{eg}

\section{Cuerpo completo}

\begin{definition}
    Si $\mathbb{K}$ es un cuerpo ordenado no Arquimediano, $\mathbb{N} \subset \mathbb{K}$ es acotado superiormente.
\end{definition}

si $b \in \mathbb{K}$ es una cota superior de $\mathbb{N} \Rightarrow n+1 \in \mathbb{N}<b, \forall n \in \mathbb{N} \Rightarrow n < b-1$ o sea $b-1$ es cota superior de $\mathbb{N} \therefore \nexists sup(\mathbb{N})$ en $\mathbb{K}$.

\begin{definition}
    Un cuerpo ordenado $\mathbb{K}$ se dice completo cuando dado un subconjunto no vacío y acotado superiormente tiene supremo en $\mathbb{K}$.
\end{definition}

\begin{note}
    1) Si el cuerpo es ordenado y completo $\Rightarrow$ es arquimediano. \\
    2) En un cuerpo ordenado completo $\mathbb{K}$ todo subconjunto no vacío y acotado inferiormente tiene ínfimo en $\mathbb{K}$.
    \begin{proof}
        Sea $Y \subset \mathbb{K}$, no vacío y acotado inferiormente. Sea $X = -Y = \{ -y:y \in Y \} \Rightarrow X$ es no vacío y acotado superiormente $\Rightarrow \exists sup(X)=a \Rightarrow -a=inf(Y)$.
    \end{proof}
\end{note}

Axioma: Existe un cuerpo ordenado llamado $\mathbb{R}$.


Ejercicio: Dados $0<a \in \mathbb{R}, m \in \mathbb{N} \Rightarrow \exists! 0<b\in \mathbb{R} : b^m =a$. Sugerencia Definir $X = \{ x \in \mathbb{R} : x\geq0, x^n<a \}$, $Y = \{ y \in \mathbb{R} : y>0, y^n >a \}$ e imitar la demostración anterior. Probar y usar que dado $x>0$ $\exists$ para cada $m \in \mathbb{N}$ un número real positivo que depende de $x$ tal que $(x+d)^m \leq A_nd+x^n, \forall 0<d<1.$.

\begin{definition}
    Un conjunto $X \subset \mathbb{R}$ se dice denso en $\mathbb{R}$ si todo intervalo abierto $(a, b)$ contiene algún punto de $X$.
\end{definition}

\begin{eg}
    $\mathbb{Q}$ es denso en $\mathbb{R}$.
    \begin{proof}
        Como $b-a>0, \exists p\in \mathbb{N}: 0 < \dfrac{1}{p}<b-a$. \\
        Sea $A = \{ m \in \mathbb{Z}: \dfrac{m}{p} \geq b \}$. Como $\mathbb{R}$ es Arquimediano, $A$ es un conjunto de números enteros no vacío y acotado por $bp$. Sea $m_0$ el menor elemento de A entonces $b\leq \dfrac{m_0}{p}$ y $\dfrac{m_0-1}{p} <b$. También $\dfrac{m_0-1}{p} > 0$, si no tendríamos que 
        \begin{equation}
            \dfrac{m_0-1}{p} \leq a \leq b \leq \dfrac{m_0}{p}
        \end{equation}
        Luego
        \begin{equation}
            b-a \leq \dfrac{m_0}{p} - \dfrac{m_0-1}{p} = \dfrac1p
        \end{equation}
        Absurdo! $\therefore \mathbb{Q}$ es denso en $\mathbb{R}$.
    \end{proof}
\end{eg}

\begin{eg}
    $\mathbb{R-Q}$ es denso en $\mathbb{R}$. \\
    Para ver que $\mathbb{R-Q}$ es denso usamos la misma idea tomando $p \in \mathbb{N} :0<\dfrac1p<\dfrac{b-a}{\sqrt2}$ por Arquimedianidad y $\dfrac{\sqrt2}{p} < b-a$ por longitud del intervalo. 
    Ejercicio terminar la demostración.
\end{eg}

\section{Cardinalidad - introducción}

\begin{definition}
    Decimos que dos conjuntos $X, Y$ tienen el mismo cardinal (coordinables o equipotentes) si $\exists f: X \to Y$ biyectiva. Notamos $X \sim Y$ o $card(X) = card(Y)$ o $\#X=\#Y$ y $\sim$ es una relación de equivalencia.
\end{definition}

Dado $n \in \mathbb{N}$ definimos $\mathbb{I}_n = \{1, 2, 3,\cdots, n\}$.

\begin{theorem}
    Sean $n, m \in \mathbb{N}$. Entonces, $\mathbb{I}_n \sim \mathbb{I}_m \iff n = m$.
    \begin{proof}
        Sabemos que si $\mathbb{I}_n \sim \mathbb{I}_m$, entonces $\exists f: \mathbb{I}_n \to \mathbb{I}_m$ biyectiva. Supongamos que $n < m$. \\
        Esto implica que puedo definir $g: \mathbb{I}_m \to \mathbb{I}_{n+1}$ suryectiva como:
        \[
        g(k) = 
        \begin{cases} 
            k & \text{si } 1 \leq k \leq n+1, \\
            1 & \text{si } k > n+1.
        \end{cases}
        \]
        $g \circ f: \mathbb{I}_n \to \mathbb{I}_{n+1} \Rightarrow$ basta probar que $\nexists$ funciones $h: \mathbb{I}_n \to \mathbb{I}_{n+1}$ suryectivas para probar el absurdo. 
        Por inducción: \\
        Si $m=1$ luego $h$ no puede ser suryectiva. Supongamos que vale si $1 \leq k \leq n-1$, si $\exists h: \mathbb{I}_n \to \mathbb{I}_{n+1}$ suryectiva $\exists k : f(n) = k$. Defino una permutación $r$ de $\mathbb{I}_{n+1}$ tal que $r(k) = n+1 \Rightarrow$ \\
        $r \circ h: \mathbb{I}_n \to \mathbb{I}_{n+1}$ es suryectiva y $(r \circ f)(n) = r(k) = n+1$. \\
        $\Rightarrow$ la restricción $r \circ f|_{\mathbb{I}_{n-1}}:\mathbb{I}_{n-1} \to \mathbb{I}_n$ y es suryectiva. Abusrdo, por Hipotesis inductiva no existen suryectivas de $\mathbb{I}_{n-1} \to \mathbb{I}_n$ \\

        $\Leftarrow$ trivial.
    \end{proof}
\end{theorem}

\begin{definition}
    $X$ es finito si $\exists n: X$ es coordinable con $\mathbb{I}_n$ y escribimos $card(X)=n$. Decimos que $X$ es infinito si no existe tal $n$.
\end{definition}