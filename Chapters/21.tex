\section{Diferenciación de sucesiones de funciones}

\begin{theorem}
  Sea $(f_n)_{n \in \N}$ una sucesión de funciones derivables en $[a, b]$. Si $\exists c \in [a, b] : (f_n(c))_{n \in \N}$ es convergente y si las derivadas $f_n^{\prime}$ convergen uniformemente en $[a, b]$, digamos $f_n^{\prime} \rightrightarrows g_n \Rightarrow f_n \rightrightarrows f$ en $[a, b]$ y $f^{\prime} = g$.
  Es decir que $(\lim_{n \to +\infty} f_n)^{\prime} = (\lim_{n \to +\infty} f_n^{\prime})$ siempre que las derivadas converjan uniformemente.
  \begin{proof}
    Veamos que $(f_n)_{n \in \N}$ es de Cauchy. Vamos a usar Teorema de Valor Medio aplicado en $f_m - f_n$. \begin{equation}
      (f_m(x) - f_n(x)) - (f_m(c) - f_n(c)) = (x-c)(f_m^{\prime}(d) - f_n^{\prime}(d)) \text{  } (\forall x \in [a, b])(d \in (x, c))
    \end{equation}
    \begin{equation}
      f_m(x) - f_n(x) = f_m(c) - f_m(c) + (x-c) (f_m^{\prime}(d) - f_n^{\prime}(d)) \Rightarrow
    \end{equation} Como $f_n^{\prime}$, $f_m^{\prime}$ convergen uniformemente y $f_n(c)$, $f_m(c)$ convergen $\therefore (f_n)_{n \in \N}$ es de Cauchy. Entonces $f_n$ converge uniformemente, digamos $f_n \rightrightarrows f$. Reescribo la igualdad anterior con $x_0$ en vez de $c$.
    \begin{equation}
      \dfrac{f_m(x_0) - f_n(x) - (f_m(x_0) - f_n(x_0))}{x - x_0} = f_m^{\prime}(d) - f_n^{\prime} \text{  } d \in (x, x_0)
    \end{equation}
    \begin{equation}
      \dfrac{f_m(x) - f_m(c)}{x - c} - \dfrac{f_n(x) - f_n(c)}{x - c} = f_m^{\prime}(d) - f_n^{\prime}(d)
    \end{equation}
    Sea $q_n(x) = \dfrac{f_n(x)-f_n(x_0)}{x-x_0}$ y $q(x) = \dfrac{f(x)-f(x_0)}{x - x_0}$ con $x \neq x_0$. Por la igualdad anterior $(q_n)_{n \in \N}$ es de Cauchy $\therefore q_n \rightrightarrows q$. Así que \begin{equation}
      \lim_{x \to x_0} (\lim_{n \to +\infty} q_n(x)) = \lim_{n \to +\infty}(\lim_{x \to x_0} q_n(x)) = q(x)
    \end{equation} O sea, \begin{equation}
      \lim_{x \to x_0} \dfrac{f(x) - f(x_0)}{x-x_0} = \lim_{n \to +\infty} q_n(x_0) = g(x_0) \Rightarrow f^{\prime} = g
    \end{equation}
  \end{proof}
\end{theorem}

\begin{corollary}
  Sea $\sum_{n \geq 1} f_n$, $f_i$ derivable en $[a, b]$ $\forall i$. Si $\sum_{n \geq 1} f_n(c)$ converge con $c \in [a, b]$ y $\sum f_n^{\prime} = g \Rightarrow \sum_{n \geq 1} f_n \rightrightarrows f$ en $[a, b]$ y $f$ es derivable con $f^{\prime} = g$.
\end{corollary}

\section{Series de potencia}

Una serie de la forma $\sum_{n \geq 0} a_n(x - x_0)^n$ se llama serie de potencias. Por simplicidad consideramos $\sum_{n \geq 0} a_n x^n$ y el caso general se reduce a éste por un cambio de variables.

\begin{theorem}
  Sea $\sum_{n \geq 0} a_n x^n$, o bien converge solo en $x = 0$ o bien $\exists r > 0$ (puede ser $+\infty$) tal que converge absolutamente en el intervalo $(-r, r)$ y diverge en $(-r, r)^c$. Además $\frac{1}{r} = \limsup \sqrt[n]{|a_n|}$.
  \begin{proof}
    Si $\limsup \sqrt[n]{|a_n|} = +\infty \Rightarrow (\sqrt[n]{|a_n|})_{n \in \N}$ es no acotada, así que tampoco lo es $(|x| \cdot \sqrt[n]{|a_n|})_{n \in \N}$, $x \neq 0 \Rightarrow |a_n x|^n \not \to 0 \Rightarrow$ diverge salvo en x = 0.
    Si $\limsup \sqrt[n]{|a_n|} = 0 \Rightarrow \lim_{n \to +\infty} \sqrt[n]{|a_n|} = 0 \Rightarrow \lim_{n \to +\infty} \sqrt[n]{a_n} |x| = 0$ $(\forall x \in \R) \Rightarrow \sum a_n x^n$ converge absolutamente en $\R$.
    Si $\limsup \sqrt[n]{|a_n|} = \frac{1}{r} \in (0, +\infty) \Rightarrow \limsup \sqrt[n]{|a_n| x^n} = |x| \lim_{n \to +\infty} \sqrt[n]{|a_n|} = \frac{1}{r}$. Luego si $\frac{|x|}{r} < 1$ Por el criterio la serie converge absolutamente $\Rightarrow -r < x < r \Rightarrow \sum a_n x^n$ converge $\forall x \in (-r , r)$.
  \end{proof}
\end{theorem}

\begin{theorem}
  Una serie de potencias converge uniformemente en todo intervalo compacto contenido en su intervalo de convergencia.
  \begin{proof}
    Sea $(-r, r)$ el intervalo de convergencia de $\sum a_n x^n$ y $0 < s < r$. $(\forall x \in [-s, s])$ vale que $|a_n x^n| \leq |a_n| s^n$ y como $\sum a_n s^n$ converge absolutamente, por el criterio de Weierstrass la serie $\sum a_n x^n$ converge uniformemente en $[-s, s]$.
  \end{proof}
\end{theorem}

\begin{corollary}
  La función $f: (-r, r) \to \R$, $f(x) = \sum a_n x^n$ es continua en el intervalo de convergencia $(-r, r)$.
\end{corollary}

Notemos que una serie de potencias puede no converger uniformemente en $(-r, r)$. En efecto si una serie de funciones continuas converge uniformemente en un conjunto, también converge uniformemente en su clasura. La demostración queda como ejercicio.

\begin{eg}
  $\sum x^n$ no converge uniformemente en $(-1, 1)$ porque tendría que converger en $|x| = 1$.
\end{eg}

\begin{lemma}
  Sea $\sum_p \alpha_p$ una serie no necesariamente convergente, con sumas parciales acotadas. Es decir que si $S_p = \alpha_1 + \cdots + \alpha_p$ $\exists k > 0 : |S_p| \leq k$ $(\forall p \in \N)$.
  Si además $b_1 \geq b_2 \geq \cdots \geq b_p \geq 0$ es decreciente $\Rightarrow \forall p \in \N$ vale que $|\alpha_1 b_1 + \cdots + \alpha_p b_p| \leq k b_1$.
  \begin{proof}
    \begin{equation}
      |\alpha_1 b_1 + \cdots + \alpha_p b_p| = |S_1 b_1 + (S_2 - S_1) b_2 + \cdots (S_p - S_{p-1}) b_p|
    \end{equation}
    \begin{equation}
      = |S_1 (b_1 - b_2) + \cdots + S_{p-1} (b_{p-1} - b_p) + S_p b_p| \leq k b_1
    \end{equation}
  \end{proof}
\end{lemma}

\begin{theorem}[Abel]
  Sea $\sum a_n x^n$ con radio de convergencia $0 < r < +\infty$. Si $\sum a_n r^n$ converge $\Rightarrow \sum a_n x^n$ converge en $[0, r]$. En particular $\lim_{x \to r^-} (\sum a_n x^n) = \sum a_n r^n$.
  \begin{proof}
    Dado $\e > 0$, $\exists n_0 \in \N : |a_{n+1} r^{n+1} + \cdots + a_{n+p} r^{n+p}| < \e$ $(\forall p \in \N) (\forall n > n_0)$. Si llamo $\alpha_p = a_{n+p} r^{n+p}$ $(\forall p \in \N)(n > n_0) \Rightarrow \alpha_p$
    cumplen las hipótesis del lema anterior con $k = \e$, $(\forall x \in [0, r])$ tenemos que \begin{equation}
      |a_{n+1} r^{n+1} + \cdots + a_{n+p} r^{n+p}| = |\alpha_1 \frac{x}{r} + \cdots + \alpha_p (\frac{x}{r})^p| (\frac{x}{r})^n
    \end{equation} Así que usando el lema para $b_p = (\frac{x}{r})^p$, vale que $\forall n > n_0$, $\forall x \in [0, r]$
    \begin{equation}
      |a_{n+1} x^{n+1} + \cdots + a_{n + p} x^{n+p}| < \e \cdot (\frac{x}{r} \cdot (\frac{x}{r})^n) \leq \e
    \end{equation} $\therefore \sum a_n x^n$ converge uniformemente en $[0, r] \Rightarrow \sum a_n x^n = f(x)$ es continua en $[0, r] \Rightarrow \sum a_n r^n = f(r) = \lim_{x \to r^-} f(x) = \lim_{x \to r^-} (\sum a_n x^n)$.
  \end{proof}
\end{theorem}

\begin{note}
  \begin{enumerate}
    \item El mismo resultado vale con $-r$ en lugar de $r$ tomando $\sum (-1)^n a_n r^n$.
    \item $\sum a_n x^n$ converge uniformemente en su intervalo de convergencia si y sólo si converge en los bordes.
  \end{enumerate}
\end{note}

\begin{theorem}
  Si $\sum a_n x^n$ converge en $[\alpha, \beta] \Rightarrow \int_{\alpha}^{\beta} \sum a_n x^n dx = \sum \frac{a_n}{n+1} (\beta^{n+1} - \alpha^{n+1})$.
  \begin{proof}
    Debe ser $[\alpha, \beta] \subset [-r, r] \Rightarrow$ por el Teorema de Abel la convergencia es uniforme.
  \end{proof}
\end{theorem}

Notemos que si $\sum a_n x^n$ no converge en el extremo $r$ de su intervalo de convergencia igual se puede integrar término a término la integral impropia. $\int_0^r \sum a_n x^n dx$ si $\sum \frac{a_n \cdot r^{n+1}}{n+1}$ converja.
Esto es porque $\int_0^r \sum a_n x^n dx = \lim_{t \to r^-} \int_0^t (\sum a_n x^n) dx = \\lim_{t \to r^-} \sum \frac{a_n}{n+1} t^{n+1} = \sum \frac{a_n \cdot r^{n+1}}{n+1}$.

\begin{theorem}
  $f(x) = \sum a_n x^n$ es derivable en todo el intervalo de convergencia $(-r, r)$ y $f^{\prime}(x) = \sum a_n \cdot n \cdot x^{n-1}$ con mismo radio.
  \begin{proof}
    Sabemos que la serie converge uniformemente en todo intervalo compacto contenido en $(-r, r)$ y que podemos derivar término a término la serie original si la serie de derivadas converge uniformemente.
    Basta ver que $\sum a_n \cdot n \cdot x^{n-1}$ tienen el mismo intervalo de convergencia. $\sum a_n \cdot n \cdot x^{n-1}$ converge $\iff x \sum a_n x^{n-1} = \sum a_n x^n$ converge. Esta serie tiene $\frac{1}{r} = \limsup \sqrt[n]{|a_n| \cdot n} = \limsup \sqrt[n]{|a_n|}$.
  \end{proof}
\end{theorem}

\begin{corollary}
  $f(x) = \sum a_n x^n$ tiene derivadas de cualquier orden en todo su intervalo de convergencia y sus derivadas sucesivas se pueden calcular derivando término a término. Entonces $\forall x \in (-r, r)(\forall k \in \N) \Rightarrow f^k(x) = \sum n \cdot (n-1) \cdots (n-k+1) \cdot a_n \cdot x^{n-k}$.
  En particular $a_k = \dfrac{f^k(0)}{k!}$. Así que la serie de potencias que converge a $f$ en $(-r, r)$ es su serie de Taylor alrededor de $x = 0$.
\end{corollary}

\section{Familias equicontinuas}

Dada una familia $E$ de funciones continuas con dominio común queremos ver cuando podemos garantizar que toda sucesión $(f_n)_{n \in \N}$ contenida en $E$ tiene una subsucesión uniformemente convergente.

A diferencia de lo que ocurre con sucesiones numéricas no basta con que la sucesión sea acotada. Por ejemplo si $f_n: [0, 1] \to \R$, $f(x) = x^n(1 - x^n) \Rightarrow f_n \to 0$, pero $f_n(\sqrt[n]{\frac{1}{2}}) = \frac{1}{4}$ $(\forall n \in \N)$.
Así que ninguna subsucesión puede converger uniformemente a $0$.

\begin{definition}[Equicontinuidad]
  Sea $X \subset \R$ y $E$ una familia de funciones $f: X \to \R$ dado $x_0 \in X$, decimos que la familia $E$ es equicontinua en $x_0$ si dado $\e > 0$, $\exists \delta > 0 : |x - x_0| < \delta \Rightarrow |f(x) - f(x_0)| < \e$ ($\forall f \in E)$ con el mismo $\delta$ $(\forall f)$.
\end{definition}

Una familia (o sucesión) es equicontinua si es equicontinua en todo punto del dominio.

\begin{theorem}
  Sea $f_n: X \to \R$ continuas, $f: X \to \R$ y $f_n \rightrightarrows f \Rightarrow (f_n)_{n \in \N}$ es equicontinua.
  \begin{proof}
    Sea $\e > 0$, $x_0 \in X$, $\exists n_0 \in \N : |f_n(x) - f(x)| < \frac{\e}{3}$ $(\forall x \in X)$ $(\forall n > n_0)$ por la convergencia uniforme.
    Como $f$ y $f_n$ son continuas $\exists \delta > 0 : |x - x_0| < \delta_i \Rightarrow |f(x) - f(x_0)| < \frac{\e}{3}$ y $|f_i(x) - f_i(x_0)| < \frac{\e}{3}$ $(\forall i \in \{1, \cdots, n_0\})$.
    Sea $\delta = min(\delta_1, \delta_2, \cdots, \delta_{n_0}) \Rightarrow$ Si $n > n_0$ y $|x - x_0| < \delta \Rightarrow$ \begin{equation}
      |f_n(x) - f_n(x_0)| \leq |f_n(x) - f(x)| + |f(x) - f(x_0)| + |f(x_0) - f_n(x_0)| < \e
    \end{equation}
  \end{proof}
\end{theorem}

\begin{eg}
  $f_n(x) = \dfrac{sen(nx)}{n}$ es equicontinua en todo $\R$.
\end{eg}

\begin{eg}
  $f_n(x) = nx$ no es equicontinua en ningún $x_0 \in \R$. Sea $\e = \frac{1}{2}$. $\forall \delta > 0$, $\exists n \in \N : \frac{1}{n} < \delta \Rightarrow x = x_0 + \frac{1}{n}$ cumple que
  $|x - x_0| = \frac{1}{n} < \delta$, pero $|f_n(x) - f_n(x_0)| = |nx - nx_0| = |n x_0 + 1 - n x_0| = 1 > \e = \frac{1}{2}$.
\end{eg}
