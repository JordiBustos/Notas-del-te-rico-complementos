\section{Conjuntos numerables}

\begin{definition}[Conjunto numerable]
    Un conjunto \(X\) es numerable si \(X \sim \N \). Cada biyección se llama una enumeración de los elementos de \(X\).
\end{definition}

\begin{definition}[Conjunto a lo sumo numerable]
    Decimos que un conjunto es a lo sumo numerable (contable) si es finito numerable.
\end{definition}

\begin{eg}
    Los números pares, \(P = \{ 2n:n \in \N \} \).
    \begin{proof}
        \(f(n) = 2n\), \(f: \N \to P\) es biyectiva \(\therefore \) es numerable.
    \end{proof}
\end{eg}

\begin{eg}
    \(\Z \) es numerable.
    \begin{proof}
        Definimos \(f: \Z \to \N \) como \(f(n) = \begin{cases}
                2n    & \text{si } n >0,     \\
                -2n+1 & \text{si } n \leq 0.
            \end{cases} \) y \(f^{-1} \) es una enumeración.
    \end{proof}
\end{eg}

\begin{theorem}
    Sea \(X\) un conjunto y \(P(X) = \{ A : A \subset X \} \Rightarrow card(X) \neq card(P)\)
    \begin{proof}
        Supongamos que \(\exists f:X \to P(X)\) biyectiva, en particular, \(f\) es suryectiva y dado \(x \in X \) puede ocurrir que \(x \in f(x)\) o \(x \notin f(x)\). Definimos \(B = \{ x \in X: x \notin f(x) \} \subset X\). Como \(f\) es suryectiva se tiene que \(\exists y \in P(X): f(y) = B\).\begin{itemize}
            \item Si \(y \notin B=f(y) \Rightarrow y \notin f(y) \Rightarrow y \in B\).
            \item Si \(y \in B = f(y) \Rightarrow y \in f(y) \Rightarrow y \notin B\).
        \end{itemize}
        Absurdo en ambos casos, luego \(\Rightarrow \nexists f\) biyectiva \(\therefore \) \(card(X) \neq card(P)\).
    \end{proof}
\end{theorem}

\begin{definition}
    Decimos que \(card(X) \leq card(Y)\) si \(\exists f: X \to Y\) inyectiva. \\
    \(card(X) < card(Y)\) si \(card(X) < card(Y)\), pero \(\neg(X\sim Y)\).
\end{definition}

\begin{prop}
    Es una relación de orden
    \begin{proof}
        \begin{itemize}
            \item \(card(X) \leq card(X)\) porque la identidad es inyectiva.
            \item \(card(X) \leq card(Y)\), \(card(Y) \leq card(Z) \Rightarrow card(X) \leq card(Z)\) pues la composición de funciones es inyectiva.
            \item \(card(X) = card(Y) \Rightarrow X \sim Y\).
        \end{itemize}
    \end{proof}
\end{prop}

\section{Cantor, Schröder, Bernstein}

\begin{theorem}[Cantor, Schröder, Bernstein]
    Si \(\exists f: X \to Y\) y \(g: Y \to X\) inyectivas \(\Rightarrow \exists h:X \to Y\) biyectiva.

    \begin{proof}
        Vamos a probar que existen dos particiones distintas de \(X\) e \(Y\). Sea \(X = X_1 \cup X_2\) y \(Y = Y_1 \cup Y_2 : f: X_1 \to Y_1\) y \(g: X_2 \to Y_2\) son biyectivas. Podemos definir a \(h: X \to Y\) como \begin{align*} h(x) = \begin{cases}
                       f(x)      & \text{si } x \in X_1, \\
                       g^{-1}(x) & \text{si } x \in X_2.
                   \end{cases}
        \end{align*}
        Que es biyectiva. Definimos \(\phi(x):P(X) \to P(X)\), \(\phi(A) = X-g(Y-f(A))\). Veamos primero que \(\phi\) es creciente i.e \(A \subseteq B \Rightarrow \phi(A) \subseteq \phi(B)\).
        \begin{proof}
            \(A \subseteq B \iff f(A) \subseteq f(B) \iff y - f(B) \subseteq y - f(A) \iff g(y-f(B)) \subseteq g(y-f(A)) \iff X - \phi(A) \subseteq X - \phi(B)\)
        \end{proof}

        Sea \(\mathscr{C} = \{ C \subset X: \phi(C) \subset C \} \neq \varnothing \) pues \(X \in \mathscr{C} \) y \(A = \bigcap_{C \in \mathscr{C}} C \neq\varnothing \), \(A \subset C, \forall C \in \mathscr{C} \) y \(\phi\) es creciente y tenemos que \(\phi(C) \subset C \Rightarrow \phi(A) \subset \phi(C) \subset C \Rightarrow \phi(C) \in A\). Además, usando otra vez que \(\phi\) es creciente, \(\phi(\phi(A)) \subset \phi(A) \Rightarrow \phi(A) \in \mathscr{C} \Rightarrow A \subset \phi(A) \Rightarrow A = \phi(A) \).
        Sean \(X_1 = A\), \(X-X_1=X_2=g_2(Y_2)\) \\
        \(Y_1 = f(A)\), \(Y_2 = Y - f(A) \Rightarrow \) \\
        \(A = \phi(A) = X-g(Y-f(A)) \iff X-A = g(Y-f(A)) \iff X-X_1=g(Y_2) = X_2 \therefore f:X_1 \to Y_1\) y \(g: X_2 \to Y_2\) son biyectivas.
    \end{proof}
\end{theorem}

\begin{eg}
    \(\N \sim \Z \sim \Q \).
    \begin{proof}
        \(f:\Z \to \Q, f(x) = x\) es inyectiva. \\
        Sea \(a \in \Z, b \in \N \), \(f: \Q \to \Z, f\left(\dfrac{a}{b}\right)=sign(a) \cdot 2^a \cdot 3^b\) es inyectiva por Teorema Fundamental de la Aritmética. Luego por el teorema anterior \(\Z \sim \Q \).
    \end{proof}
\end{eg}

\begin{eg}
    \((\N \times \N)\) es numerable.
    \begin{proof}
        \(f: \N \to \N\times \N, f(n)=(1, n)\) es inyectiva.
        \(f: \N\times \N \to \N , f((n, m))= 2^n \cdot 3^m\) es inyectiva por Teorema Fundamental de la Aritmética \(\therefore \) es numerable.
    \end{proof}
\end{eg}

\section{Los Reales son no numerables}

\begin{theorem}
    \(\R \)  no es numerable.
    \begin{proof}
        Supongamos que es numerable \(\Rightarrow \)
        \(\exists f: \N \to \R \) biyectiva. A cada número real \(x_n\) le asignamos un intervalo centrado en ese punto de longitud \(2^{-n} \). La unión de todos esos intervalos tiene longitud menor o igual a la suma de las longitudes (se pueden superponer).
        \begin{align*}
            \left|\bigcup_{n \in \N} I_{x_n}\right| \leq \sum_{n \geq 1} |I_{x_n}| = \sum_{n \in \N}\dfrac{1}{2^n} = 1
        \end{align*}
        Se cubrió un intervalo de longitud \(1\) de toda la recta real, por lo tanto quedan reales afuera y eso es un absurdo \(\therefore \R \) no es numerable.
    \end{proof}
\end{theorem}

\begin{eg}
    \(A = \{ {(a_n)}_{n \geq 1} : a_n \in \{0, 1\} \} = {\{0, 1\}}^{\N} \). Es decir las sucesiones de ceros y unos no es un conjunto numerable.
    \begin{proof}
        Supongamos que si es numerable \(\Rightarrow \) podemos escribir \\ \(A = \{ {(a_n^1)}_{n\geq1}, \ldots, {(a_n^j)}_{n\geq1}, \cdots \} \), todas las sucesiones de ceros y unos están contenidas en \(A\). La sucesión donde \(a_i=1-a_n^n\) (en el i-ésimo lugar tiene lo contrario de lo que la n-ésima sucesión tiene en el lugar n) debería ser una de ellas, pero eso es absurdo \(\therefore \) es no numerable.
    \end{proof}
\end{eg}

La idea del último ejemplo (argumento diagonal), se puede adaptar para probar que \( \interval[open left]{0}{1} \) es no numerable.

\subsection{Principio de encaje de Intervalos}

\begin{theorem}
    Sea \(I_1 \supset I_2 \supset I_3 \supset \cdots \) una sucesión de intervalos cerrados y acotados \(I_n = [a_n, b_n] \Rightarrow \)
    La intersección de todos es no vacía y \(\bigcap_{n=1}^{\infty} I_n = [a,b]\) con \(a = \sup(a_n), b=\inf(b_n)\).
    \begin{proof}
        Sea \(n \in \N \), tenemos que \(I_{n+1} \subseteq I_n\) con \(I_n = [a_n\text{, } b_n] \, \forall n \in \N \). Entonces, \begin{align*}
            a_n \leq a_{n+1} \leq b_{n+1} \leq b_n
        \end{align*}
        Sea \(A = \{ a_n : n \in \N \} \) y \(B = \{ b_n : n \in \N \} \). Claramente \begin{align*}
            a_1 \leq a_n \leq b_n \leq b_1 \, \forall n \in \N
        \end{align*}
        Luego \(A\) y \(B\) están acotados y podemos llamar \(\alpha = \sup(A)\), \(\beta = \inf(B)\) con \(\alpha \leq \beta \). Luego, \begin{align*}
             & a_1 \leq \cdots \leq a_n \leq \alpha \leq \beta \leq b_n \leq \cdots \leq b_1 \quad \forall n \in \N \\
             & \Rightarrow [\alpha\text{, }\beta] \subseteq I_n \quad \forall n \in \N                              \\
             & \Rightarrow [\alpha\text{, }\beta] \subseteq \bigcap_{n \geq 1} I_n                                  \\
             & \Rightarrow \bigcap_{n \geq 1} I_n \neq \varnothing
        \end{align*}
        Notemos que si \(x < \alpha \) entonces \(\exists a_i \in A\) tal que \(x < a_i \Rightarrow x \notin I_i \Rightarrow x \notin \bigcap_{n \geq 1} I_n\).
        Análogamente si \(x < \beta \therefore \bigcap_{n \geq 1} I_n = [\alpha\text{, }\beta]\).
    \end{proof}
\end{theorem}

\begin{theorem}
    \(\R \)  no es numerable.
    \begin{proof}
        Supongamos que \(\exists f:\N \to \R : f(n) = x_n\). \\
        Definimos la sucesión de \(\mathbb{I}_n\) de la siguiente forma: \\
        Tomamos \([0, 1]\) y divido en \(3\) cerrados iguales, luego, al menos uno no contiene a \(x_1\), lo elijo como \(\mathbb{I}_1\) (si  hay dos que no lo contienen elijo alguno). Inductivamente lo divido en \(3\) intervalos iguales y al menos \(1\) de ellos no contiene a \(x_{n+1} \) y lo elijo como \(\mathbb{I}_{n+1} \). Por el principio anterior la \(\mathbb{I} = \bigcap_{n=1}^\infty \mathbb{I}_n \neq \varnothing \) (tiene un único elemento y la longitud de los intervalos tiende a cero). \\
        Si \(x \in \mathbb{I} \) no puede ser igual o mayor que \(x_1\) (por construcción se los excluye en algún paso) \(\Rightarrow \R \) es no numerable, pues ese \(x\) queda afuera.
    \end{proof}
\end{theorem}

\section{Propiedades}

\begin{theorem}
    Sea \(X\) numerable, \(Y \subset X \Rightarrow \) \(Y\) es a lo sumo numerable.
    \begin{proof}
        Supongamos que \(Y\) no es finito. Como \(X \sim Y\) puedo pensar \(X = {(x_n)}_{n\geq1} \) y defino
        \begin{align*}
            n_1= \min \{ n \in \N: x_n \in Y \}
        \end{align*}
        E inductivamente elegimos \(n_1, n_2, \ldots , n_k, \cdots \). Definimos \(n_k = \min \{ n \in \N : n > n_k, x_n \in Y \} \Rightarrow \) Tenemos la sucesión estrictamente creciente de naturales y podemos definir \(g: \N \to Y, g(K) = x_{n_k} \text{. } g\) es inyectiva si \(K\neq Y, n_k \neq n_j\) por ser estrictamente creciente y es suryectiva, si \(y \in Y \Rightarrow y = x_j\). Para ningún \(j \Rightarrow \exists k :n_k\leq j \leq n_k+1\). Como \(j \leq n_{k+1}=\{ n>n_k:x_n \in Y \} \) debe ser \(j = n_k\).
    \end{proof}
\end{theorem}

\begin{corollary}
    \(f: X \to Y\) inyectiva e \(Y\) numerable \(\Rightarrow \) \(X\) es a lo sumo numerable.
    \begin{proof}
        \(f\) es inyectiva \(\Rightarrow X \sim f(X)\) y como \(f(X) \subset Y \Rightarrow \) es a lo sumo numerable por el teorema anterior.
    \end{proof}
\end{corollary}

\begin{theorem}
    \(f: X \to Y\) suryectiva, \(X\) a lo sumo numerable \(\Rightarrow Y\) es a lo sumo numerable.
    \begin{proof}
        \(f: X \to Y\) es suryectiva \(\exists g: Y \to X\) inversa a derecha tal que \(f \circ g = id_Y \Rightarrow f\) es inversa a izquierda de \(g \Rightarrow g\) es inyectiva \(\therefore \) por el corolario anterior \(Y\) es a lo sumo numerable.
    \end{proof}
\end{theorem}

\begin{theorem}
    Para cada \(m \in \N \). Sea \(x_n\) un conjunto numerable \(\Rightarrow \bigcup_{n \in \N}x_n = X\) es numerable.
    \begin{proof}
        \(x_n\) es numerable \(\Rightarrow \exists f:\N \to x_n\) biyectiva. Sea \(f:\N \times \N \to \bigcup_{n \in \N} x_n\) definida como \(f(n, n) = f_n(n)\). Veamos que es suryectiva, dado \(x \in X\), \(\exists n \in \N:x \in x_n \Rightarrow \exists m:x=f_n(m)\) luego \(X=f_n(m) = f(n, m)\). Como \(\N \times \N \) es numerable y \(f: \N \times \N \to \bigcup_{n \geq 1}{x_n} \) es suryectiva \(\Rightarrow \bigcup_{n \geq 1}x_n\) es a lo sumo numerable \(\therefore \) como es infinito, es numerable.
    \end{proof}
\end{theorem}

\begin{eg}
    \(\Q(x) \sim \N \).
    \begin{proof}
        \(\Q_k[x] = \{ p \in \Q(x) : gr(p) \leq k \} \) tenemos \(f_n : \Q^{n+1} \to \Q_n[x]\), \(f(a_0, \ldots, a_{n+1}) = a_0 + \cdots + a_n \cdot x^n\), cada \(f_n\) es biyectiva \(\Rightarrow \Q^{n+1} \) es numerable y como la unión numerable de conjuntos numerables es numerable \(\therefore \Q(x)\) es numerable.
    \end{proof}
\end{eg}

\begin{definition}
    Un número se dice algebraico si es raíz de algún polinomio con coeficientes enteros, por ejemplo \(\sqrt{2} \) es raíz de \(x² = 2\).
\end{definition}

\begin{definition}
    Si un número real no es algebraico se lo llama trascendente. Queda como ejercicio demostrar que el conjunto de números algebraicos es numerable.
\end{definition}