\section{Operaciones con cardinales}

Dados dos cardinales $n,m$ (no necesariamente finitos) y $X, Y$ conjuntos disjuntos tales que $card(X) = n$, $card(Y) = m$ podemos definir: \\

1) Suma: $n+m = card(X \cup Y)$.

2) Producto: $n \cdot m = card(X \times Y$.

3) Potencia: $n^m = card(\{ f: Y \to X \}) = card(X^Y)$. 

\begin{note}
    Suponer que $X \cap Y = \varnothing$ no es restrictivo porque $X \sim (X \times \{1\})$ e $Y \sim (Y \times \{ 2\})$ y $(X \times \{1\}) \sim (Y \times \{ 2\})$ y son disjuntos.
\end{note}

\begin{note}
    Hay que probar que la definición es independiente de los conjuntos $X, Y$ que elegimos.
    Si $n = card(\tilde{X})$ y $m = card(\tilde{Y})$, $\tilde{X} \cap \tilde{Y} = \varnothing \Rightarrow n+m=card(\tilde{X} \cup \tilde{Y})$. Vale porque existen biyecciones entre $\tilde{X}$ y $X$ y entre $Y$ y $\tilde{Y}$.
    Sea $f: X \to \tilde{X}$, $g:Y \to \tilde{Y}$ con lo cual $h: X \cup Y \to \tilde{X} \cup \tilde{Y}$ dada por $h(z) = \begin{cases}
        f(z) & \text{si } z\in X, \\
        g(z) & \text{si } z \in Y.
    \end{cases}$ es biyectiva.
    Similar para el producto y la potencia.
\end{note}

Supongamos $card(X) = n$, $card(Y) = m$, $card(Z) = p$, no necesariamente finitos con $X, Y,Z$ disjuntos dos a dos. La suma cumple las siguientes propiedades:
\begin{prop}
    1) Conmutatividad: $n+m=m+n$, pues $X \cup Y = Y \cup X$. \\
    2) Asociatividad: $(n+m)+p=n+(m+p)$. \\
    3) Existencia del neutro: $0+n=n$, $\varnothing \cup X=X$. \\
\end{prop}

El producto cumple las siguientes propiedades:
\begin{prop}
    1) Conmutativa: $n \cdot m = m \cdot n$, pues $X \times Y \sim Y \times X$. \\
    2) $0 \cdot n = 0 $, pues $\varnothing \times X = \varnothing $. \\
    3) $1 \cdot n = n$, pues $\{ 1 \} \times X \sim X$. Pues $f: \{ 1 \} \times X \to X$ saca el $1$ y $g: X \to \{ 1 \} \times X$ agrega el $1$. Ambas biyectivas.
\end{prop}

\begin{prop}
    Distributiva del producto en la suma: $n \cdot (m+p) = n \cdot m + n \cdot p$ porque $X \times (Y \cup Z) \sim (X \times Y) \cup (X \times Z)$.
\end{prop}

\begin{note}
    No vale la ley de cancelación: $n+m=n+p$ $\cancel{\Rightarrow}$ $m=p$.
    $n \cdot m = n \cdot p \cancel{\Rightarrow} m=p$.
\end{note}

\section{Hipotesis del continuo}

\section{Construcción de los Reales}