\section{Construcción de los Reales}

\begin{note}
	Dado \(q \in \Q \) definimos \(\tilde{q} \) como la clase de equivalencia de la sucesión constante como \((q, q, q, \cdots) \Rightarrow \Q \subset \R \).
\end{note}

\begin{definition}
	\(\tilde{s}, \tilde{t} \in \R \), \(s = [(a_n)], t = [(b_n)]\) definimos:
	\begin{itemize}
		\item \(s+t = [(a_n + b_n)]\) la clase de equivalencia de \((a_n + b_n)_{n \in \N} \).
		\item \(s \cdot t = [(a_n b_n)]\) la clase de equivalencia de \((a_n b_n)_{n \in \N} \).
	\end{itemize}
	Veamos que están bien definidas, si: \begin{align*}
		[(a_n)] = [(c_n)] \text{ y } [(b_n)] = [(d_n)] \Rightarrow a_n - c_n \to 0 \text{ y } b_n - d_n \to 0.
	\end{align*}
	En efecto,
	\begin{align*}
		([a_n] + [(b_n)]) - ([c_n] +[d_n]) & = ([a_n] - [(c_n)]) + ([b_n] - [(d_n)])        \\
		                                   & = 0                                            \\
		                                   & \Rightarrow \text{la suma está bien definida.}
	\end{align*}
	\begin{align*}
		a_n b_n - c_n d_n & = a_n b_n - c_n b_n + c_n b_n - c_n d_n                                                 \\
		                  & = b_n (a_n - c_n) + c_n (b_n - d_n)                                                     \\
		                  & \Rightarrow |a_n \cdot b_n - c_n \cdot b_n| \leq |b_n| |a_n - c_n| + |c_n| |b_n - d_n|. \\
	\end{align*}
	Como \((b_n)_{n \in \N} \) y \((c_n)_{n \in \N} \) están acotadas \(\exists M : |b_n| < M\), \(|c_n| < M, \forall n \in \N \). \\
	\(|a_n b_n - c_n b_n| \leq M (|a_n - c_n| + |b_n - d_n|) < \e, \forall n > n_0\). \\
	Porque \(a_n - c_n \to 0\), \(b_n - d_n \to 0\).
\end{definition}

\begin{prop}
	\(\R \) es un cuerpo con esta definición.
	\begin{proof}
		Ejercicio.
	\end{proof}
\end{prop}

\clearpage

\begin{theorem}
	Dado \(s \in \R - \{0\}, \exists t \in \R : s \cdot t = 1\).
	\begin{proof}
		\(s = [(a_n)]\) sabemos que \(s \notin \tilde{0} \), o sea \(a_n \not \to 0\).
		Podría pasar que algunos de los terminos de \((a_n)_{n \in \N} \) si sean \(0 \), lo que pasa es que \(a_n \neq 0.\) para \(n\) lo suficiente grande.
		Como:
		\begin{align*}
			 & a_n \not \to 0 \Rightarrow \exists \e_1 > 0, \exists \text{ infinitos valores de } M: |a_M - 0| > \e_1                                                       \\
			 & \text{Si } \e = \dfrac{\e_1}{2}\text{, como } {(a_n)}_{n \in \N} \text{ es de cauchy, } \exists n_0 : |a_n - a_m| < \dfrac{\e_1}{2} \quad \forall n, m > n_0 \\
			 & \text{Si } M > n_0 \text{ (puedo porque son infinitos) } : |a_M| > \e_1 \Rightarrow |a_m - a_M| < \dfrac{\e_1}{2} \quad \forall m > n_0                      \\
			 & \dfrac{-\e_1}{2} < a_n - a_m < \dfrac{\e_1}{2} \text{ o } -\dfrac{\e_1}{2} < a_m - a_n < \dfrac{\e_1}{2}
		\end{align*}\begin{itemize}
			\item Si \(a_M > 0 \Rightarrow \dfrac{\e_1}{2} < a_M - \dfrac{\e_1}{2} < a_n < a_M + \dfrac{\e_1}{2}, \forall n > n_0\).
			\item Si \(a_M < 0\), \(\dfrac{\e_1}{2} < -a_M - \dfrac{\e_1}{2} < -a_n<-a_m+ \dfrac{\e_1}{2} \Rightarrow a_n < -\dfrac{\e_1}{2}, \forall n > n_0\).
		\end{itemize}
		O sea que: \(\forall n > n_0, a_n\) tiene el mismo signo que \(a_M\) en particular \(a_n \neq 0, \forall n > n_0\). Sabiendo esto veamos que \(\exists \) el inverso: \begin{align*}
			S = [(a_n)] \neq 0 \Rightarrow \exists n_0 : a_n \neq 0 \quad \forall n > n_0
		\end{align*}
		Sea \({(b_n)}_{n \in \N} \subset \Q \) definida por:
		\[
			b_n =
			\begin{cases}
				0              & \text{si } n < n_0,    \\
				\dfrac{1}{a_n} & \text{si } n \geq n_0,
			\end{cases}
			\quad \Rightarrow \quad
			a_n b_n =
			\begin{cases}
				0 & \text{si } n < n_0,    \\
				1 & \text{si } n \geq n_0.
			\end{cases}
		\]
		\( \Rightarrow (1, 1, \ldots) - {(a_n \cdot b_n)}_{n \in \N} \to 0 \therefore [(a_n b_n)] = [(1, 1, \cdots)]\), es decir \(t = [(b_n)]\) cumple que \(t \cdot s = 1\).
	\end{proof}
\end{theorem}

\section{Cuerpo ordenado}

Para probar que \(\R \) es un cuerpo ordenado bajo esta definición hay que definir qué es ser positivo. \\
Sea \(s \in \R \) decimos que \(s\) es positivo si \(s \neq 0\) y \(s = [(a_n)]\) tal que \(a_n > 0 \quad \forall n > n_0\). O sea, todos los terminos son positivos a partir de un punto.

\begin{definition}
	Decimos que \(s > t\) si \(s-t > 0\).
	Ejercicio probar que está bien definido.
\end{definition}

Veamos un ejemplo de como se prueban los axiomas de orden.

\clearpage

\begin{theorem}
	Sean \(s, t \in \R : s > t, r \in \R \Rightarrow s+r > t+r\).
	\begin{proof}
		\(s = [(a_n)], t = [(b_n)], r = [(c_n)]\). \\
		Como \(s > t, \exists n_0 : a_n - b_n > 0, \forall n > n_0, a_n - b_n \not \to 0 \Rightarrow (a_n + c_n) - (b_n + c_n) = a_n + b_n > 0\) y \((a_n + c_n) - (b_n + c_n) \not \to 0 \Rightarrow \) \\
		\(s + r - (t + r) > 0 \Rightarrow s+r > t+r\).
	\end{proof}
\end{theorem}

\begin{theorem}
	\(\R \) con esta construcción es Arquimediano.
	\begin{proof}
		Sea \( a \in \R \), \( a > 0 \) quiero ver que \( \exists m \in \N \) tal que \( 0 < \frac{1}{m} \leq a \). Sabemos que:\begin{align*}
			 & a = [{(a_n)}_{n \in \N}]                                                       \\
			 & \forall \e > 0 \quad \exists n_0 \in \N : |a_n - a| < \e \quad \forall n > n_0
		\end{align*}
		En particular, dado \( \e = \frac{a}{2} \) se cumple que \( a_n \in (\frac{a}{2}\text{, } \frac{3a}{2}) \, \forall n > n_0 \in \N \). Consideremos \( p \in (0\text{, } \frac{a}{2}) \cap \Q \), como \( \Q \) es arquimediano,
		vale que: \begin{align*}
			 & \exists m \in \N : 0 < \frac{1}{m} \leq p < \frac{a}{2} < a_n \quad \forall n > n_0 \\
			 & \Rightarrow 0 < \frac{1}{m} < a_n \quad \forall n > n_0                             \\
			 & \Rightarrow 0 < \frac{1}{m} \leq a
		\end{align*}
		Por lo tanto, \( \R \) es arquimediano.
	\end{proof}
\end{theorem}

\begin{theorem}
	\(\Q \) es denso en \(\R \). Es decir dado \(r \in \R \) y \(\e > 0, \exists q \in \Q : |r-q| < \e \).\(r = [(a_n)]\), con \({(a_n)}_{n \in \N} \subset \Q \) es de Cauchy.

	\begin{proof}
		Dado \( \e > 0 \)\begin{align*}
			\exists n_0 : |a_n - a_m| < \e \quad \forall n, m > n_0
		\end{align*}
		Elijo algún \( l > n_0 \) y defino \begin{align*}
			 & q = [(a_l, a_l, \cdots)]                                       \\
			 & \Rightarrow r-q = [(a_n - a_l)] \text{ y } q-r = [(a_l - a_n)]
		\end{align*}
		Como \( l > n_0 \Rightarrow (\forall n > n_0)(a_n-a_l < \e) \) y \( (a_l - a_n < \e) \Rightarrow |r-q| < \e \).
	\end{proof}
\end{theorem}

\section{R tiene la propiedad del supremo}

Sea \(S \subset \R, S \neq \varnothing, M\) cota superior de \(S\). Vamos a construir dos sucesiones \({(u_n)}_{n \in \N}, {(l_n)}_{n \in \N} \). \\
Como \(S \neq \varnothing, \exists s_0 \in S\). Defino \(u_0 = M, l_0 = s_0\). \\
Si ya están definidos \(u_m, l_m\), llamo \(m_n = \dfrac{l_n+u_n}{2} \) al punto medio.\begin{enumerate}
	\item[(i)] Si \(m_n\) es cota superior de \(S\) definimos \(u_{m+1} = m_n, l_{n+1} = l_n\).
	\item[(ii)] Si \(m_n\) no es cota superior de S definimos \(u_{n+1} = u_n\) y \(l_{n+1} = m_n\).
\end{enumerate}

Como \(s_0 < M\) es fácil ver que \({(u_n)}_{n \in \N} \) es decreciente y que \({(l_n)}_{n \in \N} \) es creciente. Queda como ejercicio demostrarlo.
\begin{lemma}
	\({(u_n)}_{n \in \N} \) y \({(l_n)}_{n \in \N} \) son sucesiones de Cauchy de números reales.
	\begin{proof}
		Por construcción se tiene que \(l_n \leq M, \forall n \in \N \Rightarrow \) \\
		\({(l_n)}_{n \in \N} \) es creciente y acotada *\(\Rightarrow \) Es de cauchy. \\

		Como \(u_n > s_0, \forall n \in \N \Rightarrow -u_n \leq s_0, (-u_n)_{n \in \N} \) es creciente \(\Rightarrow \) Es de cauchy. \\

		* \begin{proof}
			Supongamos que \({(l_n)}_{n \in \N} \) no es de Cauchy. Entonces existe \(\e > 0 : \forall n_0, \exists n, m \geq n_0 : l_n - l_m \geq \e \). \\
			Como \({(l_n)}_{n \in \N} \) es creciente, \(l_n - l_{n_0} \geq \e \), inductivamente consigo: \\
			\(n_1 > n_0 : l_{n_1} - l_{n_0} \geq \e \) \\
			\(n_2 > n_1 : l_{n_2} - l_{n_1} \geq \e \) \\
			\vdots \\

			Por otro lado por la arquimedianidad \(\exists k \in \N : k \cdot \e > M - l_{n_0} \Rightarrow \) \\
			\(l_{n_k} - l_{n_0} = (l_{n_k} - l_{n_{k-1}}) + (l_{n_{k-1}} - l_{n_{k-2}}) + \cdots + (l_{n_1} - l_{n_0}) > k \cdot \e > M - l_{n_0} \Rightarrow \) \\
			\(l_{n_k} > M\). Absurdo!
		\end{proof}
	\end{proof}
\end{lemma}

\begin{lemma}
	\(\exists u \in \R : u_n \to u\).
	\begin{proof}
		Sea \(u_n\) un termino de \({(u_n)}_{n \in \N} \Rightarrow \exists q_n \in \Q : |u_n - q_n| < \dfrac{1}{n} \) \\
		Consideremos \((q_1, q_2, \cdots) \subset \Q \). \\
		Afirmo que \((q_n)_{n \in \N} \) es de Cauchy. Dado \(\e > 0\), como \({(u_n)}_{n \in \N} \) es de Cauchy \(\Rightarrow \) \\
		\(\exists n_0 : \forall n, m > n_0, |u_n - u_m| < \dfrac{\e}{3} \). \\
		Por arquimedianidad \(\exists n_1 : \dfrac{1}{m}, \forall n > n_1 \Rightarrow \) si \(n > \max(n_0, n_1) \Rightarrow \) \\
		\(|q_n - q_m| \leq |q_n - u_n| + |u_n - u_m| + |u_m - q_m| < \e \Rightarrow \) \\
		\(u = [(q_n)] \in \R \), falta ver que \(u_n \to u\). \\
		Si \(\tilde{q}_n = [(q_n, q_n, \cdots)] \in \R \Rightarrow \tilde{q}_n - u \to 0\) pues \(q_n\) es de Cauchy y por construcción \(u_n - q_n < \dfrac{1}{n} \Rightarrow u_n - \tilde{q}_n \to 0\) y como \(\tilde{q}_n - u \to 0 \Rightarrow \) \\
		\(u_n \to u\).
	\end{proof}
\end{lemma}

\begin{lemma}
	\(l_n \to u\)
	\begin{proof}
		Según las posibles definiciones de \(l_n\) tenemos que: \begin{align*}
			 & u_{n+1} - l_{n+1} = m_n - l_n = \dfrac{u_n+l_n}{2} - l_n = \dfrac{u_n-l_n}{2} \text{ o bien }     \\
			 & u_{n+1} - l_{n+1} = u_{n+1} - m_n = u_n - \dfrac{u_n - l_n}{2} = \dfrac{u_n - l_n}{2} \Rightarrow \\
			 & u_1 - l_1 = \dfrac{1}{2} (M-s)                                                                    \\
			 & u_2 - l_2 = \dfrac{1}{2} (u_1 - l_1) = (\dfrac{1}{2})² (u - s)                                    \\
			 & \vdots                                                                                            \\
			 & u_n - l_n = (\dfrac{1}{2})^n (M - s)
		\end{align*}
		Por arquimedianidad de \(\R \) \begin{align*}
			\forall \epsilon > 0\text{, } \dfrac{1}{2^n} (M - s) < \e \quad \forall n > n_0 \Rightarrow
		\end{align*}
		\(u_n - l_n \to 0 \therefore \) \(l_n \to u\), pues \(u_n \to u\).
	\end{proof}
\end{lemma}

\begin{theorem}
	\(\R \) tiene la propiedad del supremo.
	\begin{proof}
		(1) Veamos que \(u\) es cota superior, si no \(u < s, s \in S \Rightarrow \e = s - u > 0\), como \(u_n \to u\) y es decreciente \(\exists n : u_n - u < \e \Rightarrow u_n < u + \e = u + s - u = s\) Absurdo, por construcción \(u_n\) era cota superior de \(S, \forall n\). \\

		(2) Veamos que es la menor de las cotas superiores. \\
		Sabemos que \(l_n\) no es cota superior de \(S\), así que \((\forall n \in \N)(\exists s_n \in S) : l_n \leq s_n\). Como \(l_n \to u\) y \(l_n\) es creciente \(\Rightarrow \) \\
		\(\forall \e > 0, \exists n_0 : l_n > u - \e, \forall n>n_0 \Rightarrow s_n \geq l_n > u - \e, \forall n > n_0\). Es decir que para todo \(\e > 0\) tengo un \(s_n\) más grande en \(S \therefore u\) es la menor de las cotas superiores.
	\end{proof}
\end{theorem}

