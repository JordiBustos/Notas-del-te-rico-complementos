\section{Construcción de los Reales}

\begin{note}
  Dado $q \in \Q$ definimos $\tilde{q}$ como la clase de equivalencia de la sucesión constante como $(q, q, q, \cdots) \Rightarrow \Q \subset \R$.
\end{note}

\begin{definition}
  $\tilde{s}, \tilde{t} \in \R$, $s = [(a_n)], t = [(b_n)]$ definimos $s+t = [(a_n + b_n)]$ la clase de equivalencia de $(a_n + b_n)_{n \in \N}$. \\
  $s \cdot t = [(a_n b_n)]$ la clase de equivalencia de $(a_n b_n)_{n \in \N}$. \\
  Veamos que están bien definidas, si $[(a_n)] = [(c_n)]$ y $[(b_n)] = [(d_n)]$. \\
  $a_n - c_n \to 0$ y $b_n - d_n \to 0$. \\
  \begin{align*}
    ([a_n] + [(b_n)]) - ([c_n] +[d_n])) & =                                              \\
                                        & = ([a_n] - [(c_n)]) + ([b_n] - [(d_n)])        \\
                                        & = 0                                            \\
                                        & \Rightarrow \text{la suma está bien definida.}
  \end{align*}
  \begin{align*}
    a_n b_n - c_n d_n & =                                                                                       \\
                      & = a_n b_n - c_n b_n + c_n b_n - c_n d_n                                                 \\
                      & = b_n (a_n - c_n) + c_n (b_n - d_n)                                                     \\
                      & \Rightarrow |a_n \cdot b_n - c_n \cdot b_n| \leq |b_n| |a_n - c_n| + |c_n| |b_n - d_n|. \\
  \end{align*}
  Como $(b_n)_{n \in \N}$ y $(c_n)_{n \in \N}$ están acotadas $\exists M : |b_n| < M$, $|c_n| < M, \forall n \in \N$. \\
  $|a_n b_n - c_n b_n| \leq M (|a_n c_n| + |b_n - d_n|) < \e, \forall n > n_0$. \\
  Porque $a_n - c_n \to 0$, $b_n - d_n \to 0$.
\end{definition}

\begin{prop}
  $\R$ es un cuerpo con esta definición.
  \begin{proof}
    Ejercicio.
  \end{proof}
\end{prop}

\begin{theorem}
  Dado $S \in \R - \{0\}, \exists t \in \R : s \cdot t = 1$.
  \begin{proof}
    $S = [(a_n)]$ sabemos que $S \notin \tilde{0}$, o sea $a_n \not \to 0$. Podría pasar que algunos de los terminos de $(a_n)_{n \in \N}$ si sean $0$, lo que pasa es que $a_n \neq 0.$ para $n$ lo suficiente grande. \\
    Como $a_n \not \to 0$, $\exists \e_1 > 0, \exists$ infinitos valores de $M: |a_M - 0| > \e_1$, si $\e = \dfrac{\e_1}{2}$, como $(a_n)_{n \in \N}$ es de cauchy, $\exists n : |a_n - a_m| < \dfrac{\e}{2}, \forall n, m > n_0$. \\
    Si $M > n_0$ (puedo porque son infinitos) $: |a_M| > \e_1 \Rightarrow |a_m - a_M| < \dfrac{\e_1}{2}, \forall m > n_0$. \\
    $-\e_1/2 < a_n - a_m < \e_1 / 2$ o $-\e_1 / 2 < a_m - a_n < \e_1 / 2$. \\
    Si $a_M > 0 \Rightarrow \dfrac{\e_1}{2} < a_M - \dfrac{\e_1}{2} < a_n < a_M + \dfrac{\e_1}{2}, \forall n > n_0$. \\
    Si $a_M < 0$, $\dfrac{\e_1}{2} < -a_M - \dfrac{\e_1}{2} < -a_n<-a_m+ \dfrac{\e_1}{2} \Rightarrow a_n < -\dfrac{\e_1}{2}, \forall n > n_0$. \\
    O sea que $\forall n > n_0, a_n$ tiene el mismo signo que $a_M$ en particular $a_n \neq 0, \forall n > n_0$. \\
    Sabiendo esto veamos que $\exists$ el inverso: \\
    $S = [(a_n)] \neq 0$ por lo anterior $\exists n_0 : a_n \neq 0, \forall n > n_0$. \\
    Sea $(b_n)_{n \in \N} \subset \Q$ como \\
    $b_n = \begin{cases}
        0              & \text{si } n < n_0, \\
        \dfrac{1}{a_n} & \text{si } n > n_0.
      \end{cases} \Rightarrow$ \\

    $a_n b_n = \begin{cases}
        0 & \text{si } n < n_0, \\
        1 & \text{si } n > n_0,
      \end{cases} \Rightarrow$ \\

    $(1, 1, \cdots) - (a_n \cdot b_n)_{n \in \N} \to 0 \therefore [(a_n b_n)] = [(1, 1, \cdots)]$, es decir $t = [(b_n)]$ cumple que $t \cdot s = 1$.
  \end{proof}
\end{theorem}

\section{Cuerpo ordenado}

Para probrar que $\R$ es un cuerpo ordenado bajo esta definición hay que definir qué es ser positivo. \\
Sea $s \in \R$ decimos que $s$ es positivo si $s \neq 0$ y $s = [(a_n)]$ tal que $a_n > 0 \forall n > n_0$. O sea, todos los terminos son positivos a partir de un punto.

\begin{definition}
  Decimos que $s > t$ si $s-t > 0$.
  Ejercicio probrar que está bien definido.
\end{definition}

Veamos un ejemplo de como se prueban los axiomas de orden.

\clearpage

\begin{theorem}
  Sean $s, t \in \R : s > t, r \in \R \Rightarrow s+r > t+r$.
  \begin{proof}
    $s = [(a_n)], t = [(b_n)], r = [(c_n)]$. \\
    Como $s > t, \exists n_0 : a_n - b_n > 0, \forall n > n_0, a_n - b_n \not \to 0 \Rightarrow (a_n + c_n) - (b_n + c_n) = a_n + b_n > 0$ y $(a_n + c_n) - (b_n + c_n) \not \to 0 \Rightarrow$ \\
    $s + r - (t + r) > 0 \Rightarrow s+r > t+r$.
  \end{proof}
\end{theorem}

\begin{theorem}
  $\R$ con esta construcción es Arquimediano.
  \begin{proof}
    Sean $s, t > 0, s, t \in \R$ quiero ver que $\exists m \in \N : m \cdot s > t$. \\
    Es decir si $s = [(a_n)]$, $t = [(b_n)]$ quiero ver que $[(m \cdot a_n)] > [(b_n)]$. O sea que $m \cdot a_n - b_n \not \to 0$ y que $\exists n_0 : m \cdot a_n - b_n > 0, \forall n> n_0$. \\

    Supongamos que $\forall m, n_0, \exists n>n_0 : m \cdot a_n \leq b_n$.
    Como $(b_n)_{n \in \N}$ es de Cauchy $\Rightarrow$ $\exists M \in \Q : b_n < M, \forall n \in \N \Rightarrow$. \\
    $a_n \leq \dfrac{b_n}{m} \leq \dfrac{M}{m}$, para algún $n > n_0, \forall n_0$. \\
    Como $\Q$ es arquimediano, dado $\e > 0, \exists m : \dfrac{M}{m} < \dfrac{\e}{2}$, elijo $m$ así y tengo que $a_n \leq \dfrac{b_n}{m} < \dfrac{M}{m} < \dfrac{\e}{2}, n > n_0, \forall n_0$. \\
    Como $(a_n)_{n \in \N}$ es de Cauchy $\exists n_0 : \forall n,k > n_0, |a_n - a_k| < \e/2$. Para este $n_0, \exists$ algún $n > n_0$ tal que $a_n < \e / 2 \Rightarrow$ \\
    $\forall k > n_0, a_k - a_n < \e / 2 \Rightarrow a_k < a_n + \e / 2 < \e / 2 + \e / 2 < \e \Rightarrow a_n \to 0$. Absurdo! $([a_n]) = S > 0$. \\

    Luego $\exists m \in \N : m \cdot a_n - b_n > 0, \forall n > n_0$. Queda ver que $m \cdot a_n - b_n \cancel{\to} 0$, si $m \cdot a_n - b_n \cancel{\to} 0$ nada que probar. Caso contrario tomamos $m+1$ en vez de $m$. \\
    $(m+1) \cdot a_n - b_n = m \cdot a_n + a_n - b_n > a_n > 0, \forall n > n_0 \Rightarrow m \cdot a_n - b_n \to 0$ y $a_n \cancel{\to} 0 \therefore$ \\
    $\R$ es arquimediano.
  \end{proof}
\end{theorem}

\begin{theorem}
  $\Q$ es denso en $\R$. Es decir dado $r \in \R$ y $\e > 0, \exists q \in \Q : |r-q| < \e$. $r = [(a_n)]$, con $(a_n)_{n \in \N} \subset \Q$ es de Cauchy.

  \begin{proof}
    Dado $\e > 0, \exists n_0 : |a_n - a_m| < \e, \forall n, m > n_0$. \\
    Elijo algún $l > n_0$ y defino $q = [(a_l, a_l, \cdots)] \Rightarrow r-q = [(a_n - a_l)]$ y $q-r = [(a_l - a_n)]$.
    Como $l > n_0 \Rightarrow (\forall n > n_0)(a_n-a_l < \e)(a_l - a_n < \e) \Rightarrow |r-q| < \e$.
  \end{proof}
\end{theorem}

\section{R tiene la propiedad del supremo}

Sea $S \subset \R, S \neq \varnothing, M$ cota superior de $S$. Vamos a construir dos sucesiones $(u_n)_{n \in \N}, (l_n)_{n \in \N}$. \\
Como $S \neq \varnothing, \exists s_0 \in S$. Defino $u_0 = M, l_0 = s_0$. \\
Si ya están definidos $u_m, l_m$, llamo $m_n = \dfrac{l_n+u_n}{2}$ al punto medio. \\
i) Si $m_n$ es cota superior de $S$ definimos $u_{m+1} = m_n, l_{n+1} = l_n$. \\
ii) Si $m_n$ no es cota superior de S definimos $u_{n+1} = u_n$ y $l_{n+1} = m_n$. \\
Como $s_0 < M$ es fácil ver que $(u_n)_{n \in \N}$ es decreciente y que $(l_n)_{n \in \N}$ es creciente. Queda como ejercicio demostrarlo.

\begin{lemma}
  $(u_n)_{n \in \N}$ y $(l_n)_{n \in \N}$ son sucesiones de Cauchy de números reales.
  \begin{proof}
    Por construcción se tiene que $l_n \leq M, \forall n \in \N \Rightarrow$ \\
    $(l_n)_{n \in \N}$ es creciente y acotada *$\Rightarrow$ \\
    Es de cauchy. \\

    Como $u_n > s_0, \forall n \in \N \Rightarrow -u_n \leq s_0, (-u_n)_{n \in \N}$ es creciente $\Rightarrow$ \\
    Es de cauchy. \\

    * \begin{proof}
      Supongamos que $(l_n)_{n \in \N}$ no es de Cauchy. Entonces existe $\e > 0 : \forall n_0, \exists n, m \geq n_0 : l_n - l_m \geq \e$. \\
      Como $(l_n)_{n \in \N}$ es creciente, $l_n - l_{n_0} \geq \e$, inductivamente consigo: \\
      $n_1 > n_0 : l_{n_1} - l_{n_0} \geq \e$ \\
      $n_2 > n_1 : l_{n_2} - l_{n_1} \geq \e$ \\
      \vdots \\

      Por otro lado por la arquimedianidad $\exists k \in \N : k \cdot \e > M - l_{n_0} \Rightarrow$ \\
      $l_{n_k} - l_{n_0} = (l_{n_k} - l_{n_{k-1}}) + (l_{n_{k-1}} - l_{n_{k-2}}) + \cdots + (l_{n_1} - l_{n_0}) > k \cdot \e > M - l_{n_0} \Rightarrow$ \\
      $l_{n_k} > M$. Absurdo!
    \end{proof}
  \end{proof}
\end{lemma}

\clearpage

\begin{lemma}
  $\exists u \in \R : u_n \to u$.
  \begin{proof}
    Sea $u_n$ un termino de $(u_n)_{n \in \N} \Rightarrow \exists q_n \in \Q : |u_n - q_n| < \dfrac{1}{n}$ \\
    Consideremos $(q_1, q_2, \cdots) \subset \Q$. \\
    Afirmo que $(q_n)_{n \in \N}$ es de Cauchy. Dado $\e > 0$, como $(u_n)_{n \in \N}$ es de Cauchy $\Rightarrow$ \\
    $\exists n_0 : \forall n, m > n_0, |u_n - u_m| < \dfrac{\e}{3}$. \\
    Por arquimedianidad $\exists n_1 : \dfrac{1}{m}, \forall n > n_1 \Rightarrow$ si $n > max(n_0, n_1) \Rightarrow$ \\
    $|q_n - q_m| \leq |q_n - u_n| + |u_n - u_m| + |u_m - q_m| < \e \Rightarrow$ \\
    $u = [(q_n)] \in \R$, falta ver que $u_n \to u$. \\
    Si $\tilde{q}_n = [(q_n, q_n, \cdots)] \in \R \Rightarrow \tilde{q}_n - u \to 0$ pues $q_n$ es de Cauchy y por construcción $u_n - q_n < \dfrac{1}{n} \Rightarrow u_n - \tilde{q}_n \to 0$ y como $\tilde{q}_n - u \to 0 \Rightarrow$ \\
    $u_n \to u$.
  \end{proof}
\end{lemma}

\begin{lemma}
  $l_n \to u$
  \begin{proof}
    Según las posibles definiciones de $l_n$ tenemos que: \\
    $u_{n+1} - l_{n+1} = m_n - l_n = \dfrac{u_n+l_n}{2} - l_n = \dfrac{u_n-l_n}{2}$ o \\
    $u_{n+1} - l_{n+1} = u_{n+1} - m_n = u_n - \dfrac{u_n - l_n}{2} = \dfrac{u_n - l_n}{2} \Rightarrow$ \\
    $u_1 - l_1 = \dfrac{1}{2} (M-s)$ \\
    $u_2 - l_2 = \dfrac{1}{2} (u_1 - l_1) = (\dfrac{1}{2})² (u - s)$ \\
    \vdots \\
    $u_n - l_n = (\dfrac{1}{2})^n (M - s)$ \\

    Por arquimedianidad de $\R, \forall \epsilon > 0, \dfrac{1}{2^n} (M - s) < \e, \forall n > n_0 \Rightarrow$ \\
    $u_n - l_n \to 0 \therefore$ $l_n \to u$, pues $u_n \to u$.
  \end{proof}
\end{lemma}

\clearpage

\begin{theorem}
  $\R$ tiene la propiedad del supremo.
  \begin{proof}
    1) Veamos que $u$ es cota superior, si no $u < s, s \in S \Rightarrow \e = s - u > 0$, como $u_n \to u$ y es decreciente $\exists n : u_n - u < \e \Rightarrow u_n < u + \e = u + s - u = s$ Absurdo, por construcción $u_n$ era cota superior de $S, \forall n$. \\

    2) Veamos que es la menor de las cotas superiores. \\
    Sabemos que $l_n$ no es cota superior de $S$, así que $(\forall n \in \N)(\exists s_n \in S) : l_n \leq s_n$. Como $l_n \to u$ y $l_n$ es creciente $\Rightarrow$ \\
    $\forall \e > 0, \exists n_0 : l_n > u - \e, \forall n>n_0 \Rightarrow s_n \geq l_n > u - \e, \forall n > n_0$. Es decir que para todo $\e > 0$ tengo un $s_n$ más grande en $S \therefore u$ es la menor de las cotas superiores.
  \end{proof}
\end{theorem}

