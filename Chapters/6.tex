\section{Sucesiones}

Una sucesión de números reales es una función $x: \N \to \R$. Notamos $x(n) = x_n$ y lo llamamos el n-ésimo término de la sucesión. Indicamos la sucesión como $(x_n)_{n \in \N}$ o $(x_1, x_2, \cdots)$.

Una subsucesión de $x$ es la restricción de $x$ a un subconjunto infinito $A = \{ n_1 < n_2 < \cdots \} \subset \N$. Escribimos $(x_n)_{n \in A}$ para indicar la subsucesión.

\begin{note}
  Estrictamente la subsucesión no tiene dominio $\N$, pero es trivial considerarla como una función definida en $\N$ componiendo con $1 \mapsto x_1, 2 \mapsto x_2, \cdots$ \\
  Por esto se usa la notación $(x_{n_i})_{i \in \N}$.
\end{note}

\begin{definition}
  Decimos que $a = \lim_{n \to \infty} x_n \iff (\forall \e > 0)(\exists n_0 \in \N) : |x_n - a| < \e, \forall n > n_0$. \\
  Equivalentemente, si $\forall \e > 0$ el intervalo $(a-\e, a+\e)$ contiene a todos los términos de la sucesión salvo quizás un número finitos.
\end{definition}

\begin{theorem}[Unicidad del límite]
  Si $\lim_{n \to \infty} x_n = a$ y $lim_{n \to \infty} x_n = b \Rightarrow a=b$
  \begin{proof}
    Sea $\e > 0$, $\exists n_0$, $n_1 \in \N$ tales que $|x_n - a| < \e$ y $|x_n - b| < \e$, $\forall n > \max(n_0\text{, } n_1)$. Luego, \begin{align*}
      |a - b| & \leq |a - x_n| + |x_n - b|                                        \\
              & < \e + \e = 2\e\text{, } \forall n > \max(n_0\text{, } n_1)       \\
              & \Rightarrow |a - b| < 2\e \quad \forall \e > 0 \Rightarrow a = b.
    \end{align*}
  \end{proof}
\end{theorem}

\begin{theorem}
  Si $lim_{n \to \infty} x_n = a \Rightarrow$ toda subsucesión de $(x_n)_{n \in \N}$ converge a $a$.
  \begin{proof}
    Dado $(x_{n_1}, x_{n_2}, \cdots)$ una subsucesión de $(x_n)_{n \in \N}$. Por hipotesis dado $\e > 0, \exists n_0 \in \N : |x_n - a| < \e, \forall n > n_0$. Como los índices de la subsucesión son infinitos, $\exists n_{i_0} > n_0 \Rightarrow$ si $n_i > n_{i_0} \Rightarrow |x_{n_i} - a| < \e, (n_i > n_{i_0} > n_0) \Rightarrow lim_{n \to \infty} x_{n_i} = a$.
  \end{proof}
\end{theorem}

\begin{theorem}
  Toda sucesión convergente es acotada.
  \begin{proof}
    Sea $a = lim_{n \to \infty} x_n$. Tomando $\e = 1, \exists n_{\e} : x_n \in (a - 1, a+1), \forall n > n_{\e}$. \\
    $A = \{ x_1, x_2, \cdots, x_{n_{\e}}, a-1, a+1 \}$, $c = min(A)$, $d = max(A) \Rightarrow x_n \in [c, d], \forall n \in \N \therefore $ la sucesión es acotada.
  \end{proof}
\end{theorem}

\begin{theorem}
  Toda sucesión monótona y acotada es convergente.
  \begin{proof}
    Supongamos que $(x_n)_{n \in \N}$ es creciente y acotada y quiero ver que $lim_{n \to \infty} x_n = a = sup\{x_n\}_{n \in \N}$. \\
    Dado $\e > 0$, como $a-\e < a, a-\e$ no puede ser cota superior de $\{x_n\} \Rightarrow \exists n_0 : x_{n_0} > a - \e$.
    Como $(x_n)_{n \in \N}$ es monótona, si $n > n_0 \Rightarrow x_n > x_{n_0} > a - \e \Rightarrow$ \\
    $a-\e < x_n \leq a < a+\e, \forall n > n_0 \therefore x_n \to a$. \\
    Análogamente para $(x_n)_{n \in \N}$ es decreciente y acotada.
  \end{proof}
\end{theorem}

\begin{corollary}
  Si una sucesión monótona tiene una subsucesión convergente $\Rightarrow$ es convergente.
  \begin{proof}
    $(x_n)_{n \in \N}$ es acotada porque tiene una subsucesión acotada.
  \end{proof}
\end{corollary}

\begin{eg}
  $x_n = a^n, a \in \R$. Si $a=0$ o $a = 1$ la sucesión es constante. \\
  Si $a = -1$, la sucesión diverge porque $x_{2n} \to 1$ y $x_{2n+1} \to -1$. \\
  Si $a > 1$, la sucesión es creciente y no acotada $\Rightarrow$ diverge. \\
  Si $a < -1$, la sucesión es decreciente y no acotada $\Rightarrow$ diverge. \\
  Si $0 < a < 1$, la sucesión es convergente por ser subsucesión de $\dfrac{1}{n}$ y más aún $a^n \to 0$. \\
  Si $-1 < a < 0$, la sucesión converge pues $|a^n| = |a|^n = a^n \to 0$.
\end{eg}

\clearpage

\section{Propiedades de límites}

\begin{theorem}
  Si $lim_{n \to \infty} x_n = 0$ e $(y_n)_{n \in \N}$ una sucesión acotada $\Rightarrow lim_{n \to \infty} x_n \cdot y_n = 0$.
  \begin{proof}
    $\exists c : |y_n| < c, \forall n \in \N$, pues $y_n$ es acotado. \\
    Dado $\e > 0$, como $x_n \to 0, \exists n_0 : |x_n| < \e/c, \forall n > n_0 \Rightarrow$ \\
    $|x_n y_n| < c (\e/c) = \e, \forall n > n_0 \therefore x_n \cdot y_n \to 0$.
  \end{proof}
\end{theorem}

Sea $lim_{n \to \infty} x_n = a$ y $lim_{n \to \infty} y_n = b$.

\begin{prop}
  $x_n + y_n \to a+b$.
  \begin{proof}
    Dado $\e > 0, \exists n_0 \in \N : |x_n - a| < \dfrac{\e}{2}, \forall n > n_0$ y \\ $\exists n_1 \in \N : |y_n - b| < \dfrac{\e}{2}, \forall n > n_1$.
    Sea $n_2 = max(n_0, n_1)$ \\
    Si $n > n_2, |(x_n + y_n)-(a + b)| \leq |x_n - a| + |y_n - b| \leq \dfrac{\e}{2} + \dfrac{\e}{2} = \e$.
  \end{proof}
\end{prop}

\begin{prop}
  $x_n - y_n \to a-b$.
  \begin{proof}
    Análogo.
  \end{proof}
\end{prop}

\begin{prop}
  $x_n \cdot y_n \to a \cdot b$.
  \begin{proof}
    $x_n \cdot y_n - ab = x_n \cdot y_n - x_n \cdot b + x_n \cdot b - a \cdot b = x_n \cdot (y_n - b) + b \cdot (x_n - a)$. \\
    Como $y_n - b \to 0$ y $x_n$ es acotada, pues es convergente $\Rightarrow x_n \cdot (y_n-b) \to 0$.
    Además $x_n - a \to 0 \Rightarrow b(x_n - a) \to 0 \Rightarrow x_n \cdot y_n = a \cdot b$.
  \end{proof}
\end{prop}

\begin{prop}
  $\dfrac{x_n}{y_n} \to \dfrac{a}{b}$.
  \begin{proof}
    Si $y_n \not \to 0 \Rightarrow y_n \neq 0$ salvo quizá finitos términos. En efecto, como $b \neq 0$, si $\e = |b|$ resulta que $0 \notin (b - \e, b + \e) \Rightarrow$
    \begin{align*}
      \exists n_{\e} : y_n \in (b-\e, b+\e) \quad \forall n > n_{\e}
    \end{align*} Luego $y_n \neq 0, \forall n > n_{\e}$. Ahora escribo \begin{align*}
      \dfrac{x_n}{y_n} - \dfrac{a}{b} = \dfrac{x_n \cdot b - a \cdot y_n}{y_n \cdot b}
    \end{align*} Quiero ver que $\dfrac{1}{y_n \cdot b}$ es acotada. Como
    \begin{align*}
       & y_n \cdot b \to b² \text{ si } \e = \dfrac{b²}{2} \, \exists n_0 : y_n \cdot b > \dfrac{b²}{2}, \forall n > n_0 \Rightarrow                   \\
       & 0 < \dfrac{1}{y_n \cdot b} < \dfrac{2}{b²} \quad \forall n > n_0 \Rightarrow \left(\dfrac{1}{y_n \cdot b}\right)_{n \in \N} \text{ es acotada}.
    \end{align*}
  \end{proof}
\end{prop}

\section{Ejemplo subsucesiones}

\begin{eg}
  $x_n = \sqrt[n]{a}, \ a > 0$. \\
  $(x_n)_{n \in \N}$ es monótona (decreciente si $a > 1$, creciente si $0 < a < 1$) y acotada $0 < l = \lim_{n \to \infty} \sqrt[n]{a}$. \\
  Para ver que $l = 1$, considero la subsucesión $(a^{\frac{1}{n(n+1)}})_{n \in \N}$ convergente a $l$. \\
  $l = \lim_{n \to \infty} a^{\frac{1}{n(n+1)}} = \lim_{n \to \infty} a^{\frac{1}{n} - \frac{1}{n+1}} =
    \frac{\lim_{n \to \infty} a^{\frac{1}{n}}}{\lim_{n \to \infty} a^{\frac{1}{n+1}}} = \frac{l}{l} = 1.$ \\
  $\therefore l = 1$.
\end{eg}

\begin{eg}
  $x_n = \sqrt[n]{n}$. \\
  Veamos si es monótona. $\sqrt[n]{n} > \sqrt[n+1]{n+1} \iff n^{n+1} > (n+1)^{n+1} \iff n > (\dfrac{n+1}{n})^{n+1}$. \\
  Esto pasa para $\forall n \geq 3$ porque $(1+\frac{1}{n})^n < 3, \forall n \in \N$. Se puede demostrar por inducción. \\
  Luego, es decreciente a partir del tercer término. \\
  Además está acotada inferiormente por $0, (\sqrt[n]{n} > 0) \therefore \exists lim_{n \to \infty} \sqrt[n]{n} = l = inf\{ \sqrt[n]{n} \}_{n \in \N}$. \\
  Como $\sqrt[n]{n} > 1 \Rightarrow l \geq 1$ y $l \neq 0$. \\
  Considero la subsucesión $(2n)^{\frac{1}{2n}} \Rightarrow l^2 = (lim_{n \to \infty} (2n)^{\frac{1}{2n}})^2 = lim_{n \to \infty} 2^{\frac{1}{n}} \cdot lim_{n \to \infty} n^{\frac{1}{n}} = 1 \cdot l$. \\
  Luego $l^2 = l$ y $l \neq 0 \therefore l = 1$.
\end{eg}

\begin{note}
  La definición de límite puede ser reformulada de la siguiente forma: \\
  Dado $a \in \R$ es el límite de $(x_n)_{n \in \N} \iff \forall \e > 0$ el conjunto $\{ n \in \N : x_n \in (a - \e, a + \e) \}$ tiene complemento finito (o vacío) en $\N$.
\end{note}

Vamos a ver que $a \in \R$ es el límite de alguna subsucesión de $(x_n)_{n \in \N} \iff \forall \e > 0$ el conjunto $\{ n \in \N : x_n \in (a - \e, a + \e) \}$ es subconjunto infinito de $\N$
\begin{theorem}
  $a \in \R$ es el límite de alguna subsucesión de $(x_n)_{n \in \N} \iff \forall \e > 0, \exists$ infinitos índices $n : x_n \in (a-\e, a+\e)$.
  \begin{proof}
    Para la ida tenemos que $a = lim_{n \in A} x_n$, con $A = \{ n_1 < n_2 < \cdots \}$. \\
    $\forall \e > 0, \exists i_0 : x_{n_i} \in (a - \e, a + \e), \forall i > i_0$. \\
    Como existen infinitos $i > i_0 : n_i \in A \Rightarrow \exists$ infinitos $n_i \in \N$ tales que $x_{n_i} \in (a - \e, a + \e)$. \\

    Recíprocamente si tomamos sucesivamente $\e = 1, 1/2, 1/3, \cdots$. Puedo obtener $A = \{ n_1, n_2, \cdots \} : lim_{n \in A} x_n = a$ pues: \\
    Sea $n_1 \in \N : x_{n_1} \in (a-1, a+1)$. Supongamos por inducción que $n_1 < n_2 < \cdots < n_i$ fueron definidos tales que $x_{n_2} \in (a - 1/2, a + 1/2), \cdots x_{n_i} \in (a - 1/i, a + 1/i)$. \\
    Como el conjunto $\{ n \in \N : x_n \in (a - \frac{1}{1+i}, a + \frac{1}{1+i}) \}$ es infinito, contiene algún $x_n$ con $n > n_i$ y lo tomo como $x_{n_i + 1}$. \\
    Como $| x_{n_i} - a | < \frac{1}{i}, \forall i \in \N \Rightarrow lim_{i \to \infty} x_{n_i} = a$.
  \end{proof}
\end{theorem}

\section{Punto de acumulación}

\begin{definition}
  $a \in \R$ se llama punto de acumulación de $(x_n)_{n \in \N}$ si es límite de alguna subsucesión de $(x_n)_{n \in \N}$.
\end{definition}

\begin{definition}
  Sea $(x_n)_{n \in \N}$ una sucesión acotada, digamos $\alpha \leq x_n \leq \beta, \forall n \in \N$.
  Si llamamos $X_n = \{ x_n, x_{n+1}, \cdots \}$ tenemos que $[\alpha, \beta] \supset X_1 \supset X_2 \supset \cdots$. \\
  Así que llamando $a_n = inf(X_n)$, $b_n = sup(X_n)$ tenemos que \\
  $\alpha \leq a_1 \leq a_2 \leq \cdots \leq a_n \leq \cdots \leq b_n \leq \cdots \leq \cdots \leq b_2 \leq b_1 \leq \beta \Rightarrow$. \\
  $\exists a = lim_{n \to \infty} a_n = sup(a_n)_{n \in \N} = sup(inf(X_n)_{n \in \N})$. \\
  $b = lim_{n \to \infty} b_n = inf(b_n)_{n \in \N} = inf(sup(X_n)_{n \in \N})$. \\

  $a$ se llama el límite inferior de la sucesión $(x_n)_{n \in \N}$. \\
  $a = \liminf_{n \to \infty} x_n$. \\

  $b$ se llama el límite superior de la sucesión $(x_n)_{n \in \N}$. \\
  $b = \limsup_{n \to \infty} x_n$.
\end{definition}

\clearpage

\begin{eg}
  \begin{align*}
     & (x_n)_{n \in \N} = (-1, 2, -1/2, 3/2, -1/3, 4/3, \cdots),                      \\
     & (x_{2n+1})_{n \in \N} = -\frac{1}{n}, \ (x_{2n})_{n \in \N} = 1 + \frac{1}{n}, \\
     & \text{Luego, } \inf(X_{2n-2}) = \inf(X_{2n-1}) = -\frac{1}{n},                 \\
     & \sup(X_{2n-2}) = \sup(X_{2n-1}) = 1 + \frac{1}{n} \Rightarrow                  \\
     & \liminf_{n \to \infty} x_n = 0, \ \limsup_{n \to \infty} x_n = 1.
  \end{align*}
\end{eg}