\section{Límite superior e inferior}

\begin{theorem}
  $(x_n)_{n \in \N}$ acotada $\Rightarrow \liminf x_n$ es el menor punto de acumulación de la sucesión y $\limsup x_n$ es el mayor.

  \begin{proof}
    Veamos que $a = \liminf_{n \to \infty} x_n$ es punto de acumulación de $(x_n)_{n \in \N}$. \\
    Como $a = \lim_{n \to \infty} a_n$ (los ínfimos de $X_n$). Dados $\e$ y $n_0 \in \N, \exists n_1 > n_0 : a-\e < a_{n_1} < a+\e$, $a_{n_1} = inf(X_{n_1}) \Rightarrow a + \e$ no puede ser cota inferior de $X_{n_1}$. \\
    $\Rightarrow \exists n \geq n_1 : a_{n_1} \leq x_n \leq a+\e$, $(n \geq n_1 \geq n_0$ y $(a - \e < x_n < a + \e$), luego $a$ es punto de acumulación de $(x_n)_{n \in \N}$. \\
    Falta ver que si $c < a \Rightarrow c$ no es punto de acumulación. \\
    Como $a = \lim_{n \to \infty} a_n$ y $c < a \Rightarrow \exists n_0 \in \N : c < a_{n_0} \leq a$. Como $a_{n_0}$ es el ínfimo de $X_{n_0}$ si $n \geq n_0 \Rightarrow c < a_{n_0} \geq x_n$. Tomando $\e = a_{n_0} - c \Rightarrow c+\e = a_{n_0}$ y entonces $\forall x_n: n \geq n_0, x_n \notin (c - \e, c + \e) \therefore c$ no es punto de acumulación.
  \end{proof}
\end{theorem}

\begin{theorem}
  Toda sucesión acotada de $\R$ tiene una subsucesión convergente.
  \begin{proof}
    $\limsup x_n$ es punto de acumulación de $x_n$ así que alguna subsucesión converge a él.
  \end{proof}
\end{theorem}

\begin{corollary}
  Una sucesión acotada de $\R$ es convergente $\iff \limsup x_n = \liminf x_n$, es decir, posee un único punto de acumulación.
  \begin{proof}
    Para la ida tenemos que si $(x_n)_{n \in \N}$ converge, toda subsucesión converge a lo mismo y en particular $\limsup x_n = \liminf x_n = lim_{n \to \infty} x_n$. \\
    Para la vuelta, si $\limsup x_n = \liminf x_n = \lim_{n \to \infty} a_n = \lim_{n \to \infty} b_n \Rightarrow$ \\
    Dado $\e > 0, \exists n_0 \in \N : a - \e \leq a_{n_0} \leq a \leq b_{n_0} \leq a + \e$. \\
    Como $n \geq n_0 \Rightarrow a_{n_0} \leq x_n \leq b_{n_0}$.
  \end{proof}
\end{corollary}

\section{Sucesiones de Cauchy}

Dada una sucesión $(x_n)_{n \in \N} \subset \R$, decimos que es de Cauchy si dado $\e > 0, \exists n_0 \in \N : |x_n - x_m| < \e, \forall n,m > n_0$. Al igual que hicimos con las sucesiones de Cauchy en $\Q$.

\begin{enumerate}
  \item Toda sucesión convergente es de Cauchy
  \item Toda sucesión de Cauchy es acotada
\end{enumerate}

\begin{lemma}
  Si una sucesión de Cauchy tiene una subsucesión que converge a $a \in \R \Rightarrow \lim_{n \to \infty} x_n = a$.
  \begin{proof}
    Dado $\e > 0 : \exists n_0 : |x_n - x_m| < \e / 2, \forall n,m > n_0$. \\
    Como $a$ es el límite de una subsucesión $\exists n_1 > n_0 : |x_n - a| < \e / 2$. \\
    $\Rightarrow$ si $n > n_0, |x_n - a| \leq |x_n - x_{n_1}| + |x_{n_1} - a| < \e / 2 + \e / 2 = \e$.
  \end{proof}
\end{lemma}

\begin{theorem}
  Toda sucesión de Cauchy en $\R$ es convergente.
  \begin{proof}
    Por ser de Cauchy es acotada, por ser acotada alguna subsucesión converge al $\limsup$ o $\liminf$ y por el lema anterior, toda la sucesión converge.
  \end{proof}
\end{theorem}

\section{Límites infinitos}

\begin{definition}
  Decimos que $\lim_{n \to \infty} x_n = + \infty$ si dado $a > 0, \exists n_0 \in \N : x_n > a \forall n > n_0$.
\end{definition}

\begin{definition}
  Decimos que $\lim_{n \to \infty} x_n = - \infty$ si dado $a > 0, \exists n_0 \in \N : x_n < -a \forall n > n_0$.
\end{definition}


\begin{prop}
  $\lim_{n \to \infty} x_n = + \infty$ e $(y_n)_{n \in \N}$ es acotada inferiormente $\Rightarrow \lim_{n \to \infty} (x_n + y_n) = +\infty$.
  \begin{proof}
    Sea $A > 0, \exists c \in \R : c < y_n, \forall n \in \N$.
    Como $\lim_{n \to \infty} x_n = + \infty, \exists n_0 \in \N : x_n > A - c, \forall n > n_0 \Rightarrow$
    Si $n > n_0, x_n + y_n > A - c + c = A \therefore \lim_{n \to \infty} (x_n + y_n) = +\infty$.
  \end{proof}
\end{prop}

\begin{prop}
  $\lim_{n \to \infty} x_n = +\infty$ y $\exists c > 0 : y_n > c, \forall n \in \N \Rightarrow \lim_{n \to \infty} (x_n \cdot y_n) = + \infty$.
  \begin{proof}
    Dado $A > 0, \exists n_0 \in \N : x_n > A/c, \forall n > n_0 \Rightarrow$ si $n > n_0, x_n \cdot y_n > (A/c) \cdot c = A \therefore \lim_{n \to \infty} x_n \cdot y_n = + \infty$.
  \end{proof}
\end{prop}


\begin{prop}
  $x_n > 0 \forall n \in \N \Rightarrow \lim_{n \to \infty} x_n = 0 \iff \lim_{n \to \infty} = \frac{1}{x_n} = +\infty$.
  \begin{proof}
    Para la ida, $\lim_{n \to \infty} x_n = 0$. Dado $A > 0, \exists n_0 \in \N : 0 < x_n < 1/A, \forall n > n_0 \Rightarrow \frac{1}{x_n} > A, \forall n > n_0 \therefore \lim_{n \to \infty} \frac{1}{x_n} = +\infty$. \\
    Para la vuelta $\lim_{n \to \infty} \frac{1}{x_n} = +\infty$. Dado $\e > 0, \exists n_0 \in \N : \frac{1}{x_n} > \frac{1}{\e}, \forall n > n_0 \Rightarrow$ \\
    $0 < x_n < \e, \forall n > n_0 \therefore \lim_{n \to \infty} x_n = 0$.
  \end{proof}
\end{prop}

\begin{prop}
  Sean $(x_n)_{n \in \N}, \forall n \in \N$ e $(y_n)_{n \in \N}$ sucesiones de números positivos $\Rightarrow$ \\
  a) Si $\exists c > 0 : x_n > c, \forall n \in \N$ y si $\lim_{n \to \infty} y_n = 0 \Rightarrow \lim_{n \to \infty} \frac{x_n}{y_n} = +\infty$. \\
  b) $(x_n)_{n \in \N}$ es acotada y $\lim_{n \to \infty} y_n = +\infty \Rightarrow \lim_{n \to \infty} \frac{x_n}{y_n} = 0$.
  \begin{proof}
    a) Dado $A > 0, \exists n_0 \in \N : 0 < y_n < c/A, \forall n > n_0 \Rightarrow \frac{x_n}{y_n} > (A/c) \cdot c = A, \forall n > n_0 \therefore \lim_{n \to \infty} \frac{x_n}{y_n} = + \infty$. \\
    b) Dado $\e > 0, \exists n_0 \in \N : y_n > k/\e, \forall n > n_0$, donde $k > 0$ es cota superior de $x_n \Rightarrow \forall n > n_0, 0 < \frac{x_n}{y_n} < k \cdot (\e/k) = \e \therefore \lim_{n \to \infty} \frac{x_n}{y_n} = 0$.
  \end{proof}
\end{prop}

\section{Series numéricas}

Dada una sucesión $(a_n)_{n \in \N} \subset \R$ formamos una nueva sucesión $(S_n)_{n \in \N}$. Dada por $S_1 = a_1, S_2 = a_1 + a_2, \cdots, S_n = a_1 + \cdots + a_n$; que llamamos sumas parciales de la serie $\sum_{n \geq 1} a_n$.


\begin{theorem}
  $\sum_{n \geq 1} a_n$ converge $\Rightarrow \lim_{n \to \infty} a_n = 0$.
  \begin{proof}
    $\lim_{n \to \infty} S_n = \lim_{n \to \infty} S_{n-1} \Rightarrow 0 = s - s = \lim_{n \to \infty} S_n - \lim_{n \to \infty} S_{n-1} = \lim_{n \to \infty} S_n - S_{n-1} = \lim_{n \to \infty} a_n$.
  \end{proof}
\end{theorem}

\clearpage

\begin{eg}
  El recíproco del teorema anterior es falso pues $\sum_{n \geq 1} \frac{1}{n}$ diverge.
  \begin{proof}
    \begin{align*}
      S_{2^n} & = 1 + 1/2 + (1/3 + 1/4) + (1/5 + 1/6 + 1/7 + 1/8) + \cdots + (\frac{1}{2^{n-1}+1} + \cdots + \frac{1}{2^n}) \\
              & > 1 + 1/2 + 2/4 + 4/8 + \cdots + \frac{2^{n-1}}{2^n} = 1 + n/2
    \end{align*}
    Luego $\lim_{n \to \infty} S_{2^n} = + \infty$.
  \end{proof}
\end{eg}

\begin{theorem}
  $a_n \geq 0, \forall n > 0 \Rightarrow$ \\
  $\sum_{n \geq 1} a_n$ converge $\iff S_n$ forman una sucesión acotada.
  \begin{proof}
    Por ser $a_n \geq 0$, $(S_n)_{n \in \N}$ es monótona creciente, luego converge $\iff$ es acotada.
  \end{proof}
\end{theorem}

\begin{corollary}[Criterio de comparación]
  Sean $a_n, b_n \geq 0, \forall n \in \N$, dadas $\sum_{n \geq 1} a_n$, $\sum_{n \geq 1} b_n \Rightarrow$ Si \begin{equation}
    \exists c > 0, n_0 \in \N : a_n \leq c \cdot b_n, \forall n > n_0 \Rightarrow
  \end{equation}
  \begin{enumerate}
    \item Si $\sum b_n$ converge entonces $\sum a_n$ converge.
    \item Si $\sum a_n$ diverge entonces $\sum b_n$ diverge.
  \end{enumerate}

  \begin{note}
    $\sum a_n$ converge $\iff$ Dado $n_0 \in \N, \sum_{n \geq n_0} a_n$ converge.
  \end{note}
  \begin{proof}
    Sea $(s_n)_{n \in \N}$ la primera serie, $(t_n)_{n \in \N}$ la segunda, $t_n = s_{n + n_0} - s_{n_0}$.
  \end{proof}
\end{corollary}

\begin{theorem}[Criterio de Cauchy]
  $\sum a_n$ converge $\iff \forall \e > 0, \exists n_0 \in \N : | a_{n+1} + \cdots + a_{n+p} | < \e, \forall n > n_0, p \in \N$.
  \begin{proof}
    $|s_{n+p} - s_n| = |a_{n+1} + \cdots + a_{n+p}| < \e, \forall n > n_0 \in \N \Rightarrow (s_n)_{n \in \N}$ es de Cauchy y por lo tanto convergente.
  \end{proof}
\end{theorem}

\begin{definition}
  $\sum a_n$ es absolutamente convergente si $\sum |a_n|$ converge. \\
  Si $\sum a_n$ converge y $\sum |a_n|$ diverge entonces decimos que es condicionalmente convergente.
\end{definition}

\begin{theorem}
  Toda serie absolutamente convergente es convergente.
  \begin{proof}
    Como $\sum |a_n|$ converge $\Rightarrow \forall \e > 0, \exists n_0 \in \N : | |a_{n+1}| + \cdots + |a_{n+p}| < \e, \forall n > n_0, p \in \N \Rightarrow$ \\
    $|a_{n+1} + \cdots + a_{n+p}| \leq |a_{n+1}| + \cdots + |a_{n+p}| < \e, \forall n > n_0, p \in \N \therefore \sum a_n$ converge.
  \end{proof}
\end{theorem}
