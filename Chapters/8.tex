\section{Ejemplo de convergencia condicional}

\begin{eg}
  $\sum_{n \geq 1} \dfrac{(-1)^{n+1}}{n}$ converge, pero no lo hace absolutamente.
  \begin{proof}
    i) Las sumas parciales pares son: 

    \begin{itemize}
      \item $S_2 = 1 - 1/2$
      \item $S_4 = (1 - 1/2) + (1/3 - 1/4)$
      \item $S_6 = (1 - 1/2) + (1/3 - 1/4) + (1/5 - 1/6)$
    \end{itemize}

    En general $S_2 < S_4 < S_6 < \cdots < S_{2n} < \cdots < 1$, es creciente y acotada, luego converge.

    ii) Las sumas parciales impares son:

    \begin{itemize}
      \item $S_1 = 1$
      \item $S_3 = 1 - (1/2 - 1/3)$
      \item $S_5 = 1 - (1/2 - 1/3) - (1/4 - 1/5)$
    \end{itemize}

    En general $S_1 > S_2 > \cdots > S_{2n+1} > \cdots > 0$. Es decreciente y acotada, luego converge.

    Por i) e ii) $\exists \tilde{s} = lim_{n \to \infty} S_{2n}$ y $\exists s^{\prime} = lim_{n \to \infty} S_{2n+1}$, como $S_{2n+1} - S_{2n} = \dfrac{1}{2n+1} \to 0 \Rightarrow \tilde{s} = s^{\prime} \therefore$ la serie converge.
  \end{proof}
\end{eg}

\section{Corolarios de series}

\begin{corollary}
  Sea $\sum b_n$ una serie convergente de términos positivos si $\exists k > 0$ y $n_0 \in \N$ tales que $|a_n| \leq k \cdot b_n, \forall n > n_0 \Rightarrow$ \\
  $\sum a_n$ converge absolutamente.

  \begin{proof}
    Sencilla aplicación del criterio de comparación.
  \end{proof}
\end{corollary}

\begin{corollary}
  $\forall n > n_0, |a_n| \leq k \cdot c^n$ con $0 < c < 1, k > 0 \Rightarrow$ \\
  $\sum a_n$ es absolutamente convergente. 
  \begin{proof}
    Como $c \in (0, 1), \sum c^n$ converge (serie geométrica).
  \end{proof}
\end{corollary}

\begin{note}
  Tomando $k = 1, |a_n| \leq c^n \iff \sqrt[n]{a_n} \leq c < 1, \forall n > n_0 \in \N$. Que esto valga para algún $n_0$ específico significa que $\limsup_{n \to \infty} \sqrt[n]{a_n} < 1$. 
\end{note}

\section{Criterios de convergencia}

\subsection{Criterio de la raíz}

\begin{corollary}
  Si $\limsup_{n \to \infty} \sqrt[n]{|a_n|} < 1 \Rightarrow \sum a_n$ converge absolutamente.
  \begin{note}
    Si existen infinitos índices tales que $\sqrt[n]{|a_n|} > 1 \Rightarrow \sum a_n$ diverge pues $a_n \not \to 0$.
    Si $\limsup_{n \to \infty} \sqrt[n]{|a_n|} = 1$ el criterio no concluye ($\sum \frac{1}{n}$, $\sum \frac{1}{n^2})$.
  \end{note}
\end{corollary}

\subsection{Criterio del cociente y D'Alembert}

\begin{theorem}[Criterio del cociente]
  Sean $\sum_{n \geq 1} a_n$ una serie tal que $a_n \neq 0, \forall n \in \N$ y $\sum_{n \geq 1} b_n$ una serie de términos positivos y convergente $\Rightarrow$ \\
  Si $\exists n_0 \in \N : \dfrac{|a_{n+1}|}{|a_n|} \leq \dfrac{|b_{n+1}|}{|b_n|}, \forall n > n_0 \Rightarrow \sum_{n \geq 1} a_n$ es absolutamente convergente.

  \begin{proof}
    \begin{equation}
      \dfrac{|a_{n_0 + 2}|}{|a_{n_0 + 1}|} \leq \dfrac{|b_{n_0+2}|}{|b_{n_0+1}|}, \cdots, \dfrac{|a_n|}{|a_{n-1}|} \leq \dfrac{|b_n|}{|b_{n-1}|}, \forall n > n_0
    \end{equation}
    Multiplicando todas las desigualdades obtenemos
    \begin{equation}
      \dfrac{|a_n|}{|a_{n_0+1}|} \leq \dfrac{|b_n|}{|b_{n_0+1}|} \iff |a_n| \leq \dfrac{|a_{n_0+1}| \cdot b_n}{b_{n_0+1}}
    \end{equation}
    $\therefore$ por criterio de comparación $\sum_{n \geq 1} a_n$ converge.
  \end{proof}
\end{theorem}

\begin{corollary}[Criterio de D'Alembert]
  Si $\exists c \in (0, 1) : \dfrac{|a_{n+1}|}{a_n} \leq c, \forall n > n_0 \Rightarrow \sum_{n \geq 1} a_n$ converge absolutamente. \\
  Equivalentemente si $\limsup \dfrac{|a_{n+1}|}{|a_n|} < 1 \Rightarrow \sum_{n \geq 1} a_n$ converge absolutamente.

  \begin{proof}
    Tomamos $b_n = c^n$ en el teorema anterior pues $\sum_{n \geq 1} c^n$ converge si $c \in (0 ,1)$.
  \end{proof}
\end{corollary}

\begin{note}
  Si el cociente es $1$ el criterio no decide. \\
  Si el cociente es mayor que $1$ la serie diverge.
\end{note}

\begin{theorem}
  $(a_n)_{n \in \N}$ acotada $a_n > 0, \forall n \in \N \Rightarrow$ \\
  \begin{equation}
    \liminf_{n \to \infty} \dfrac{a_{n+1}}{a_n} \leq \liminf_{n \to \infty} \sqrt[n]{a_n} \leq \limsup_{n \to \infty} \sqrt[n]{a_n} \leq \limsup_{n \to \infty} \dfrac{a_{n+1}}{a_n}
  \end{equation}
  En particular si $\exists lim_{n \to \infty} \dfrac{a_{n+1}}{a_n} \Rightarrow \exists lim_{n \to \infty} \sqrt[n]{a_n}$ y son iguales.

  \begin{proof}
    Veamos que $\limsup_{n \to \infty} \sqrt[n]{a_n} \leq \limsup_{n \to \infty} \dfrac{a_{n+1}}{a_n}$. \\
    Supongamos que no lo es $\Rightarrow$
    \begin{equation}
      \exists c : \limsup_{n \to \infty} \dfrac{a_{n+1}}{a_n} < c < \limsup_{n \to \infty} \sqrt[n]{a_n}
    \end{equation}
    Por la primer desigualdad $\exists n_0 \in \N : \forall n > n_0, \dfrac{a_{n+1}}{a_n} < c \Rightarrow$ \\
    \begin{equation}
      \dfrac{a_{n_0+1}}{a_{n_0}} < c, \dfrac{a_{n_0+2}}{a_{n_0+1}} < c, \cdots, \dfrac{a_n}{a_{n+1}} < c \Rightarrow
    \end{equation}
    Multiplicando termino a termino 
    \begin{equation}
      \dfrac{a_n}{a_{n_0}} < c^{n - n_0} \Rightarrow a_n < \dfrac{a_{n_0}}{c^{n_0}} \cdot c^n
    \end{equation}
    Llamemos $k = a_{n_0}/c^{n_0} \in \R \Rightarrow$
    $lim_{n \to \infty} \sqrt[n]{k} = 1 \Rightarrow lim_{n \to \infty} \sqrt[n]{k \cdot c^n} = c \cdot lim_{n \to \infty} \sqrt[n]{k} = c$ \\
    Tendríamos que $\limsup_{n \to \infty} \sqrt[n]{a_n} \leq \limsup_{n \to \infty} \sqrt[n]{k \cdot c^n} = c$ Absurdo! \\
    Luego debe ser $\limsup_{n \to \infty} \sqrt[n]{a_n} \leq \lim_{n \to \infty} \dfrac{a_{n+1}}{a_n}$. \\


  \end{proof}
\end{theorem}

\begin{eg}
  Puede existir el límite de la raíz y no del cociente.

  \begin{proof}
    Sean $0 < a < b$ y la sucesión $(x_n)_{n \in \N}$ que se obtiene alternando cada término por $a$ ó $b$. \\
    $x_1 = a$, $x_2 = a \cdot b$, $x_3 = a² \cdot b$, $x_4 = a² \cdot b²$. \\

    $\dfrac{x_{n+1}}{x_n} = \begin{cases}
      b & \text{si n es par}, \\
      a & \text{si no}
    \end{cases}$

    Luego $\not \exists lim_{n \to \infty} \dfrac{x_{n+1}}{x_n}$, pero $\limsup_{n \to \infty} x_n = b$, $\liminf_{n \to \infty} x_n = a$. \\
    Por otro lado $lim_{n \to \infty} \sqrt[n]{x_n} = \sqrt{a \cdot b}$. Si $x_{2k} = a^k \cdot b^k \Rightarrow \sqrt[2k]{a^k \cdot b^k} = \sqrt{a \cdot b}$. \\
    Si $x_{2k-1} = a^k \cdot b^{k-1} \Rightarrow \sqrt[2k-1]{x_{2k-1}} = a^{\frac{k}{2k-1}} \cdot b^{\frac{k}{2k-1}} \to \sqrt{a \cdot b}$ si $k \to \infty$.
  \end{proof}
\end{eg}

\subsection{Criterio de Dirichlet - Abel}

\begin{theorem}
  Sea $\sum_{n \geq 1} a_n$ no necesariamente convergente cuyas sumas parciales forman una sucesión acotada. \\
  Sea $(b_n)_{n \in \N}$ una sucesión decreciente de números positivos con $lim_{n \to \infty} b_n = 0 \Rightarrow \sum_{n \geq 1} a_n \cdot b_n$ es convergente.

  \begin{proof}
    \begin{equation}
      a_1 \cdot b_1 + \cdots a_n \cdot b_n =  
    \end{equation}
    \begin{equation}
      a_1 \cdot (b_1 - b_2) + (a_1 + a_2) \cdot (b_2 - b_3) + \cdots + (a_1 + \cdots + a_{n-1}) \cdot (b_{n-1} - b_n) + (a_1 + \cdots + a_n) \cdot b_n
    \end{equation}
    \begin{equation}
      = S_1 \cdot (b_1 - b_2) + S_2 \cdot (b_2 - b_3) + S_3 \cdot (b_3 - b_2) + \cdots + S_{n-1} (b_{n-1} - b_n) + S_n \cdot b_n 
    \end{equation}
    \begin{equation}
      = \sum_{i = 2}^n S_{i-1} \cdot (b_{i-1} - b_i) + S_n \cdot b_n
    \end{equation}
    Por hipotesis $\exists k > 0 : |S_n| \leq k, \forall n \in \N$ (son acotadas). Además $\sum_{n \geq 2} b_{n-1} \cdot b_n$ es una telescópica de números positivos. \\
    Luego $\sum_{n \geq 2} S_{n-1} \cdot (b_{n-1} - b_n)$ es absolutamente converge y en particular convergente.
    Como $S_n \cdot b_n \to 0 \Rightarrow \exists lim_{n \to \infty} \sum_{i \geq 2}^n S_{i-1} \cdot (b_{i-1} - b_i) + S_n \cdot b_n = lim_{n \to \infty} \sum_{i \geq 1}^n a_n \cdot b_n \therefore$ converge.
  \end{proof}
\end{theorem}

\begin{corollary}[Abel]
  Si $\sum a_n$ es convergente y $(b_n)_{n \in \N}$ es decreciente de términos positivos $\Rightarrow \sum a_n \cdot b_n$ es convergente.
  \begin{proof}
    $b_n \to c$ pues es acotada inferiormente por $0$ y decreciente. \\
    Sea $(b_n - c)_{n \in \N}$ es una sucesión decreciente que tiende a cero, podemos aplicar el teorema anterior y, luego $\sum_{n \geq 1} a_n \cdot (b_n - c) = S \Rightarrow \sum_{n \geq 1} a_n \cdot b_n = S - \sum a_n \cdot c$.
  \end{proof}
\end{corollary}

\begin{corollary}[Criterio de Leibniz]
  Si $(b_n)_{n \in \N}$ es decreciente y $lim_{n \to \infty} b_n = 0 \Rightarrow \sum (-1)^n b_n$ converge.
  \begin{proof}
    Aplicamos el criterio de Dirichlet pues $\sum (-1)^n$ no converge y $S_n = 0, -1$ son acotadas.
  \end{proof}
\end{corollary}

\section{Parte positiva y negativa}

\begin{definition}
  Sea $\sum_{n \geq 1} a_n$, para cada $n \in \N$ definimos $p_n = \begin{cases}
    a_n & \text{si } a_n > 0, \\
    0 & \text{ c.c }
  \end{cases}$ \\
  $q_n = \begin{cases}
    0 & \text{si } a_n \geq 0, \\
    -a_n & \text{ c.c }
  \end{cases} \Rightarrow p_m, q_n \geq 0, \forall n \in N, a_n = p_n - q_n$ y $|a_n| = p_n + q_n = a_n + 2 \cdot q_n = 2 \cdot p_n - a_n$.  
\end{definition}

\begin{note}
  Si $\sum_{n \geq 1} a_n$ converge absolutamente $\Rightarrow \forall k \in \N$ vale que:
  \begin{equation}
    \sum_{n \geq 1} |a_n| \geq \sum_{n = 1}^k |a_n| = \sum_{n = 1}^k p_n + \sum_{n = 1}^k q_n \Rightarrow 
  \end{equation}
  $\sum_{n \geq 1} p_n$ y $\sum_{n \geq 1} q_n$ convergen, pues son crecientes y están acotadas por $\sum_{n \geq 1} |a_n|$. \\

  Además si $\sum_{n \geq 1} p_n$ y $\sum_{n \geq 1} q_n$ convergen $\Rightarrow \sum a_n$ converge absolutamente.
\end{note}

\begin{theorem}
  Si $\sum a_n$ converge condicionalmente $\Rightarrow \sum_{n \geq 1} p_n$ y $\sum_{n \geq 1} q_n$ son divergentes.

  \begin{proof}
    Supongamos que $\sum q_n = c \Rightarrow |a_n| = a_n + 2 \cdot q_n$ \\
    \begin{equation}
      \sum_{n = 1}^k |a_n| = \sum_{n = 1}^k a_n + 2 \cdot \sum_{n = 1}^k q_n
    \end{equation}
    Si $k \to +\infty$ luego $\sum_{n \geq 1} |a_n| = \sum_{n \geq 1} a_n + 2 \cdot c$ Absurdo! pues no converge absolutamente $\therefore \sum_{n \geq 1} q_n$ diverge. 
  \end{proof}
\end{theorem}

\section{Reordenamientos}

\begin{definition}
  Sea $\sum a_n$, cambiar el orden de la suma significa tomar una biyección $\phi: \N \to \N$ y considerar la serie $\sum b_n$ como $b_n = a_{\phi(n)}$.
\end{definition}

\begin{theorem}
  Todas las reordenaciones de una serie absolutamente convergente convergen al mismo valor de la serie original.
  \begin{proof}
    Sea $\sum a_n, a_n > 0 \forall n \in \N$. Sea $\phi: \N \to \N$ una biyección y $b_n = a_{\phi(n)}$ quiero ver que $\sum b_n = \sum a_n$. \\
    Sea $S_n$ las sumas parciales de $a_n$ y $T_n$ las de $b_n$. Para cada $n \in \N$ llamo $m = max(\phi(1), \cdots \phi(n))$. Donde $\{ \phi(1), \cdots, \phi(n) \} \subseteq [[1, m]]$. \\
    Luego $T_n = \sum_{i \geq 1} a_{\phi(i)} \leq \sum_{j \geq 1} a_j = S_m$. Luego $T_n \leq S_m$. \\
    Análogamente con $\phi^{-1} \Rightarrow S_m \leq T_n$ luego $\lim_{n \to \infty} S_n = lim_{n \to \infty} T_n \therefore \sum_{n \geq 1} a_n = \sum_{n \geq 1} b_n$. 
  \end{proof}
\end{theorem}

El caso general, $\sum_{n \geq 1} a_n = \sum_{n \geq 1} p_n - \sum_{n \geq 1} q_n \Rightarrow$ todo reordenamiento $(b_n)_{n \in N}$ de los términos de $(a_n)_{n \in \N}$ produce un reordenamiento de $(p_n)_{n \in \N}$ y $(q_n)_{n \in \N}$ que llamo $(u_n)_{n \in \N}$, $(v_n)_{n \in \N}$. De modo que son la parte positiva y negativa de $(b_n)_{n \in \N}$ y usamos el caso anterior pues son términos de valores positivos.  



