\section{Topología Euclídea}

\begin{definition}
  La norma euclídea es la función \( \| \cdot \| \) que en cada punto \(x = (x_1, \ldots, x_n) \in \R^n\) toma el valor \( \|x\| = \sqrt{\sum_{i = 1}^n x_i^2} \).
  Propiedades: \begin{enumerate}
    \item \(\forall x \in \R^n, \|x\| > 0\) y \( \|x\| = 0 \iff x = 0\).
    \item \(\forall x \in \R^n, \forall \lambda \in \R, \| \lambda x\| = |\lambda| \|x\| \).
    \item \( \|x+y\| \leq \|x\| + \|y\| \).
  \end{enumerate}
\end{definition}

\begin{prop}[Desigualdad de Cauchy-Schwarz]
  Dados \(x\), \(y \in \R^n\) entonces \begin{align*}
    \left| \sum_{i = 1}^n x_i \cdot y_i \right| \leq \|x\| \cdot \|y\|
  \end{align*}
  \begin{proof}
    Tenemos que \begin{align*}
      & \left(\sum_{k = 1}^{n+1} a^2_k\right) \cdot \left(\sum_{k = 1}^{n+1} b^2_k \right) - \sum_{1 \leq k \leq j \leq n+1} (a_k b_j - a_j b_k) = {\left(\sum_{k = 1}^{n+1} a_k b_k \right)}^2 \\
      & \text{Como } \sum_{1 \leq k \leq j \leq n+1} {(a_k b_j + a_j b_k)}^2 \geq 0 \quad \forall n \in \N \\
      & \Rightarrow {\left(\sum_{k = 1}^{n+1} a_k b_k \right)}^2 \leq \left(\sum_{k = 1}^{n+1} a^2_k\right) \cdot \left(\sum_{k = 1}^{n+1} b^2_k \right) 
    \end{align*}
  \end{proof}
\end{prop}

\begin{definition}
  La distancia euclídea es la función \(d: \R^n \times \R^n \to \R : d(x, y) = \|x-y\| \).
  Tenemos que \(\forall x,y,z \in \R^n\).
  \begin{enumerate}
    \item \(d(x, y) > 0\) y \(d(x, y) = 0 \iff x=y\)
    \item \(d(x, y) = d(y, x)\)
    \item \(d(x, z) \leq d(x, y) + d(y, z)\)
  \end{enumerate}
  \begin{proof}
    \begin{enumerate}
      \item \(d(x, y) = \|x-y\| > 0, d(x, y) = 0 \iff \|x-y\| = 0 \iff x = y \).
      \item \(d(x, y) = \|x-y\| = \|(-1)\| \cdot \|y-x\| = \|y-x\| = d(y, x)\).
      \item \(d(x, z) = \|x-z\| = \|x-z+y-y\| \leq \|x-y\| + \|z-y\| = d(x, y) + d(y, z)\).
    \end{enumerate}
  \end{proof}
\end{definition}

\subsection{Conjuntos abiertos}
\begin{definition}
  Si \(x \in \R^n\) y \(r > 0\), la bola abierta centrada en \(x\) de radio \(r\) es el conjunto \(B_r(x) = B(x,r) = \{ y \in \R^n : d(x, y) < r \} \).
\end{definition}

\begin{definition}
  Un subconjunto \(U \subset R^n\) es abierto si \(\forall x \in U, \exists r > 0 : B_r(x) \subset U\).
\end{definition}

\begin{lemma}
  La bola abierta \(B_r(x)\) es un conjunto abierto.
  \begin{proof}
    Sea \(s = r - d(x, y), y \in B_r(x)\). Si \(z \in B_s(y) \Rightarrow \) \\
    \(d(x, z) \leq d(x, y) + d(y, z) < d(x, y) + s = r\) \\
    Luego \(z \in B_r(x) \Rightarrow B_s(y) \subset B_r(x) \therefore \) es abierto.
  \end{proof}
\end{lemma}

\begin{prop}
  Un subconjunto propio \(U \subset \R^n\) es abierto \(\iff \forall x \in U\) se tiene que \( \inf \{d(x, y) : y \in \R^n - U\} > 0\). \\
  El ínfimo existe pues el conjunto es no vacío y acotado inferiormente.
  \begin{proof}
    Para la ida tenemos que si \(U\) es abierto
    \begin{align*}
      \Rightarrow (\forall x \in U) \quad \exists r > 0 : B_r(x) \subset U
    \end{align*}
    Si \(y \in R^n - U \Rightarrow y \notin B_r(x)\) por lo que \(d(x, y) > r > 0\) y \(r\) es una cota inferior mínima positiva del conjunto, pues si \(d(x,y) < r \Rightarrow y \in B_r(x)\) absurdo! \\

    Para la vuelta tomemos \(s = \inf \{ d(x,y) : y \in \R^n - U \} > 0\). Consideremos \(B_{\frac{s}{2}}(x)\). Si \(z\) pertenece a esa bola entonces \(d(x, z) < \frac{s}{2} < s \Rightarrow \) \\
    \(d(x, z) \notin \{ d(x,y) : y \in R^n - U \} \) pues si no sería el ínfimo, se sigue que \(z \in U\) y entonces \(B_{\frac{s}{2}}(x) \subset U \therefore U\) es abierto.
  \end{proof}
\end{prop}

\begin{prop}
  Propiedades de conjuntos abiertos:
  \begin{enumerate}
    \item \(\varnothing \), \(\R^n\) son abiertos.
    \item Intersección de una familia finita y no vacía de \(\R^n\) de subconjuntos abiertos es un conjunto abierto.
    \item La unión de una familia cualquiera de subconjuntos abiertos de \(\R^n\) es abierto.
  \end{enumerate}
  \begin{proof}
    \begin{enumerate}
      \item Por definición
      \item \(U = \bigcap_{i = 1}^k U_i, U_i \subset \R^n\), abierto. Tomando \(B_r(x)\) con \(r = \min(r_1, \ldots, r_k)\) y listo.
      \item Si \(u\) es una familia de abiertos de \(\R^n\) y \(U\) su unión, dado \(x \in U\) tiene que haber algún \(V \in u\) tal que \(x \in V\). Por ser \(V\) abierto \(\exists r > 0 : B_r(x) \subset V \subset U \therefore U\) es abierto.
    \end{enumerate}
  \end{proof}
\end{prop}

Notemos que la hipotesis de que la familia sea finita en la intersección es necesaria pues si definimos \(U_n = \left(1 - \dfrac{1}{n}, 1 + \dfrac{1}{n}\right) \quad \forall n \in \N \Rightarrow \bigcap_{n \geq 1} U_n = \{1\} \) que no es abierto.

\begin{definition}
  La topología Euclídea es el conjunto de todos los abiertos de \(\R^n\).
\end{definition}

\begin{corollary}
  Un subconjunto de \(\R^n\) es abierto \(\iff \) es la unión de una familia de bolas abiertas.
  \begin{proof}
    Para la ida: Sea \(U \subset \R^n\) un conjunto abierto y sea \(u\) la familia de bolas abiertas de \(\R^n\) contenidas en \(U\). Si llamamos \(\tilde{U} \) a la unión de todas basta ver que \(\tilde{U} = U\).\begin{itemize}
      \item[(\(\subset \))] Inmediato.
      \item[(\(\supset \))] Dado \(x \in U, \exists r > 0 : B_r(x) \subset U\) por ser \(U\) abierto y \(B_r(x) \in u \Rightarrow B_r(x) \subset \tilde{U} \Rightarrow U \subset \tilde{U} \) pues vale \(\forall x \in U\).
    \end{itemize}
    La vuelta es la proposición anterior.
  \end{proof}
\end{corollary}

\subsection{Interior de un conjunto}

\begin{definition}[Interior]
  \(X \subset \R^n\) decimos que \(x \in \R^n\) es interior a \(X\) si \(\exists r > 0 : B_r(x) \subset X\). El interior del conjunto X, que notamos \(X^{\circ} \), es el conjunto de puntos interiores de X.
\end{definition}

\begin{note}
  Si \(X^{\circ} \subset X, X\) es abierto \(\iff X^{\circ} = X\).
\end{note}

\begin{prop}
  Sea \(X \subset \R^n\). \(X^{\circ} \) es abierto y coincide con la unión de todos los subconjuntos abiertos de \(X\). En particular \(X^{\circ} \) es el mayor abierto contenido en \(X\).
  \begin{proof}
    Sea \(U\) la unión de todos los subconjuntos abiertos de X, veamos que \(U = X^{\circ} \). \\
    (i) Si \(x \in X^{\circ}, \exists r > 0 : B_r(x) \subset U\), como \(B_r(x)\) es abierto \(B_r(x) \subset U\) por lo que \(X^{\circ} \subset U\). \\
    (ii) Si \(x \in U \Rightarrow \exists V \subset X\) abierto tal que \(x \in X\), como \(V\) es abierto \(\exists r > 0 : B_r(x) \subset V \subset X \Rightarrow X^{\circ} \) por definición de punto interior, luego \(U \subset X^{\circ} \). \\
    \(\therefore X^{\circ} = U\).
  \end{proof}
\end{prop}

\begin{corollary}
  Si \(X \subset Y \Rightarrow X^{\circ} \subset Y^{\circ} \).
  \begin{proof}
    \(X^{\circ} \subset X \subset Y\). Como \(Y^{\circ} \) es el mayor abierto contenido en \(Y\) se sigue que \(X^{\circ} \subset Y^{\circ} \subset Y\).
  \end{proof}
\end{corollary}

\subsection{Propiedades del interior}

\begin{prop}
  \({(X \cap Y)}^{\circ} = X^{\circ} \cap Y^{\circ} \).
  \begin{proof}
    \({(X \cap Y)}^{\circ} \subset X \cap Y \Rightarrow X^{\circ} \cap Y^{\circ} \) es abierto \(\Rightarrow (X^{\circ} \cap Y^{\circ}) \subset {(X \cap Y)}^{\circ} \). \\
    \(X \cap Y \subset X \Rightarrow {(X \cap Y)}^{\circ} \subset X^{\circ} \) y \(X \cap Y \subset Y \Rightarrow {(X \cap Y)}^{\circ} \subset Y^{\circ} \Rightarrow \) \\
    \({(X \cap Y)}^{\circ} \subset X^{\circ} \cap Y^{\circ} \).
  \end{proof}
\end{prop}

\begin{prop}
  \(X^{\circ} \cup Y^{\circ} \subset {(X \cup Y)}^{\circ} \).
  \begin{proof}
    \(X^{\circ} \subset X \subset X \cup Y\) y \(Y^{\circ} \subset Y \subset X \cup Y \Rightarrow \) \\
    \(X^{\circ} \cup Y^{\circ} \subset {(X \cup Y)} \Rightarrow X^{\circ} \cup Y^{\circ} \subset {(X \cup Y)}^{\circ} \), pues \(X^{\circ} \cup Y^{\circ} \) es abierto.
  \end{proof}
\end{prop}

Notemos que la inclusión puede ser estricta pues sea \(X = (-1, 1)\), \(Y = [1, 2] \Rightarrow \) \\
\(1 \notin X^{\circ} \cup Y^{\circ} = (-1, 1) \cup (1, 2)\), pero \(1 \in {(X \cup Y)}^{\circ} = (-1, 2)\).

\section{Entorno}

\begin{definition}[Entorno]
  Sea \(x \in \R^n\), un entorno de \(x\) es un subconjunto \(N \subset R^n : x \in N^{\circ} \).
\end{definition}

\begin{prop}
  Un subconjunto de \(\R^n\) es abierto \(\iff \) es un entorno de cada uno de sus puntos.
  \begin{proof}
    Si \(U\) es entorno de cada uno de sus puntos entonces \(U \subset U^{\circ} \Rightarrow U = U^{\circ} \therefore \) es abierto.
  \end{proof}
\end{prop}

\begin{prop}
  El interior de un subconjunto \(X \subset \R^n\) es el conjunto de puntos de los que \(X\) es entorno.
  \begin{proof}
    Sea \(U\) el conjunto de puntos de \(X\) de los que \(X\) es un entorno. Si \(x \in U \Rightarrow X\) es el entorno de \(x\) y \(x \in X^{\circ} \Rightarrow U \subset X^{\circ} \). Si \(x \in X^{\circ} \Rightarrow X\) es un entorno de \(x \Rightarrow x \in U \Rightarrow X^{\circ} \subset U\).
  \end{proof}
\end{prop}

\section{Conjunto Cerrado}

\begin{definition}[Conjunto cerrado]
  Un subconjunto \(F \subset \R^n\) se dice cerrado si su complemento es abierto (\(R^n - F\) es abierto).
\end{definition}


\begin{prop}
  Propiedades de los conjuntos cerrados:
  \begin{enumerate}
    \item \(\varnothing, \R^n\) son cerrados.
    \item La unión finita de conjuntos cerrados es cerrado.
    \item La intersección de una familia arbitraria de cerrados es cerrada.
  \end{enumerate}
  \begin{proof}
    \begin{enumerate}
      \item \({(\R^n)}^c = \varnothing \) y \(\varnothing^c = \R^n\) abiertos.
      \item Si \(F_1, \ldots, F_k \subset R^n\) son cerrados \(\Rightarrow {(F_1 \cup \cdots \cup F_k)}^c = F_1^c \cap \cdots \cap F_k^c\) y cada \(F_i\) es abierto.
      \item Si \(F\) es una familia de cerrados \(\bigcap_{f \in F} {f}^c = (\bigcup_{f \in F} f^c)\) que es abierto.
    \end{enumerate}
  \end{proof}
\end{prop}

\begin{definition}
  Si \(x \in R^n, r > 0\) llamamos bola cerrada centrada en \(x\) de radio \(r\) al conjunto \(\overline{B_r{x}} = \overline{B(r, x)} = \{ y \in \R^n : d(x, y) \leq r \} \).
\end{definition}

\begin{prop}
  Veamos que la bola cerrada es, efectivamente, cerrada.
  \begin{proof}
    Si \(y \in \R^n - \overline{B_r(x)} \) tenemos que \(d(x, y) > r\). Llamamos \(0 < r_1 = d(x, y) - r\) y queremos que \(B_{r_1}(y) \subset R^n - \overline{B_r(x)} \). \\
    Sea \(z \in B_{r_1}(y)\) por desigualdad triangular: \\
    \(d(x, y) \leq d(x, z) + d(z, y)\) \\
    \(d(x, z) \geq d(x, y) - d(z, y) > d(x, y) - r_1 = r \Rightarrow z \notin \overline{B_r(x)} \therefore B_{r_1}(y) \subset \R^n - \overline{B_r(x)} \).
  \end{proof}
\end{prop}

\subsection{Clausura}

\begin{definition}[Clausura]
  La clausura de un subconjunto \(X \subset R^n\) es la intersección \(\overline{X} \) de todos los subconjuntos cerrados de \(R^n\) que contienen a \(X\).
\end{definition}

\begin{prop}
  Sea \(X \subset R^n \Rightarrow \overline{X} \) es el menor subconjunto cerrado de \(\R^n\) que contiene a \(X\) en el sentido de que es contenido en todo subconjunto cerrado de \(\R^n\) que contiene a \(X\).
\end{prop}

\begin{corollary}
  Un subconjunto \(X \subset \R^n\) es cerrado \(\iff \overline{X} = X\).
  \begin{proof}
    Si \(X\) es cerrado, contiene a \(X\) y luego \(\overline{X} \subset X\), como \(X \subset \overline{X} \Rightarrow X = \overline{X} \). \\
    Si \(X = \overline{X} \Rightarrow X\) es cerrado, pues \(\overline{X} \) es cerrado.
  \end{proof}
\end{corollary}

\begin{prop}
  \(X \subset Y \Rightarrow \overline{X} \subset \overline{Y} \).
  \begin{proof}
    \(X \subset Y \subset \overline{Y} \), pero como \(\overline{X} \) es el menor cerrado que contiene a \(X \Rightarrow \overline{X} \subset \overline{Y} \).
  \end{proof}
\end{prop}

\begin{prop}
  \(\overline{X \cup Y} = \overline{X} \cup \overline{Y} \).
  \begin{proof}
    \(X \subset X \cup Y \Rightarrow \overline{X} \subset \overline{X \cup Y} \) \\
    \(Y \subset X \cup Y \Rightarrow \overline{Y} \subset \overline{X \cup Y} \Rightarrow \) \\
    \(\overline{X} \cup \overline{Y} \subset \overline{X \cup Y} \). \\
    Para la otra contención notemos que \(X \subset \overline{X} \) y \(Y \subset \overline{Y} \Rightarrow X \cup Y \subset \overline{X \cup Y} \Rightarrow \overline{X \cup Y} \subset \overline{X} \cup \overline{Y} \).
  \end{proof}
\end{prop}

\begin{prop}
  \(\overline{X \cap Y} \subset \overline{X} \cap \overline{Y} \).
  \begin{proof}
    \(X \cap Y \subset X \subset \overline{X} \) y \(X \cap Y \subset Y \subset \overline{Y} \Rightarrow X \cap Y \subset \overline{X} \cap \overline{Y} \Rightarrow \overline{X \cap Y} \subset \overline{X} \cap \overline{Y} \). Pues es el menor de los cerrados.
  \end{proof}
\end{prop}

\begin{prop}
  Sean \(X \subset \R^n\) y \(x \in \R^n\). Son equivalentes:
  \begin{enumerate}
    \item \(x \in \overline{X} \).
    \item \(\forall r > 0, B_r(x) \cap X \neq \varnothing \).
    \item \(\forall \) entorno abierto \(N\) de \(x\), \(N \cap X \neq \varnothing \).
    \item \(\forall \) entorno \(N\) de \(x\), \(N \cap X \neq \varnothing \).
  \end{enumerate}

  \begin{proof}
    (4) \(\Rightarrow \) (3) \(\Rightarrow \) (2) son inmediatos. \\
    (2) \(\Rightarrow \) (1) Por contrarrecíproco, supongamos que \(x \notin \overline{X} \) de manera tal que \(\exists \) F cerrado tal que \(x \in F\) y \(X \subset F\) (si no estaría en la clasura) \(\Rightarrow x \in F^c\) que es abierto \(\Rightarrow \exists r > 0, B_r(x) \subset F^c, B_r(X) \cap X \subset B_r(x) \cap F = \varnothing \). \\
    (1) \(\Rightarrow \) (4) Supongamos que existe \(N\) entorno de \(x\) tal que \(N \cap X = \varnothing \). El conjunto \(U = N^{\circ} \) es abierto y \(x \in U\). Además \(U \subset N \subset X^c \Rightarrow X \subset U^c\) que es cerrado y \(x \notin U^c \Rightarrow x \notin \overline{X} \).
  \end{proof}
\end{prop}

\begin{eg}
  Si \(\Q^n\) es el conjunto de puntos de \(\R^n\) que tienen todas sus coordenadas racionales \(\overline{\Q^n} = \R^n\). Decimos que \(\Q^n\) es denso en \(\R^n\).

  \begin{proof}
    (1) \(\overline{\Q^n} \subseteq \R^n : \Q^n \subset \R^n\), que es cerrado \(\Rightarrow \overline{\Q^n} \subseteq \R^n\). \\
    (2) \(\R^n \subseteq \overline{\Q^n}\): Si \(x \in \R^n\) y \(U\) es entorno abierto de \(x, U \cap \Q^n \neq \varnothing \) (1) \(\iff \) (2). Como \(U\) es abierto \(\exists r > 0 : B_r(x) \subset U\). \\
    Si \(x = (x_1, \ldots, x_n)\) para cada \(i \in \{1, \ldots, n\} \) elijo un racional \(q_i\) en el intervalo \((x_i - \dfrac{r}{\sqrt{n}}, x_i + \dfrac{r}{\sqrt{n}}) : |x_i - q_i| < \dfrac{r}{\sqrt{n}} \Rightarrow d(x, q) < r\). Sabemos que existe porque ya probamos que \(\Q \) es denso en \(\R \). \\
    \(\Rightarrow q \in B_r(x) \subset U \Rightarrow U \cap \Q^n \neq \varnothing \) pues al menos \(q\) está allí \(\forall U\) entorno abierto de X.
    \(\therefore \R^n \subset \Q^n\).
  \end{proof}
\end{eg}