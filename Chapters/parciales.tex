\section{Primer parcial - Primera fecha - 14/10}

\begin{enumerate}
  \item Dada $f: X \to Y$. Mostrar que $f$ es inyectiva $\iff f(A-B) = f(A) - f(B)$ para todo par de conjuntos $A, B \subseteq X$.
  \item Considerar $K$ un cuerpo ordenado y $f: K \to K$ definida por $f(x) = (1+x²)^{-1}$. Calcular, en el caso que existan: $inf(f(\Z))$ y $sup(f(\Z))$.
  \item Considerar $B \subseteq A$ conjuntos no vacíos de números reales. Se sabe que $A$ está acotado superior y que $\forall x \in A, \exists y \in B : x \leq y$. Demostrar que $sup A = sup B$.
  \item Demostrar, usando la definición que $\dfrac{4^n - 6}{4^{n+1}+10} \to \dfrac{1}{4}$.
  \item Decidir si las siguientes sucesiones son convergentes. En caso afirmativo, determinar su límite. \begin{enumerate}
    \item $a_1 = -\dfrac{3}{2}, 3 \cdot a_{n+1} = 2 + a_n^3$. Dar un valor de $a_1$ para que la sucesión converja a $-2$.
    \item $b_1 = 1, b_{n+1} = 1 + \dfrac{1}{b_n}$.
  \end{enumerate}
  \item Considerar una sucesión $(a_n)$ de términos positivos tal que $a_n \to L$. Demostrar que \begin{equation} \dfrac{n}{\dfrac{1}{a_1} + \dfrac{1}{a_2} + \cdots + \dfrac{1}{a_n}} \to L \end{equation} 
  \item Considerar $(a_n)$ y $(b_n)$ sucesiones acotadas superiormente mostrar que \begin{equation} \limsup(a_n + b_n) \leq \limsup a_n + \limsup b_n \end{equation} Dar un ejemplo donde la desigualdad sea estricta.
  \item Considerar $(a_n)$ sucesión de términos positivos tal que $\sum a_n$ es convergente. Decidir si la siguiente serie es convergente o divergente: $\sum \dfrac{\sqrt{a_n}}{n}$.
\end{enumerate}

\section{Primer parcial - Segunda fecha - 28/10}

\begin{enumerate}
  \item Considerar $(a_n)$ y $(b_n)$ tales que $a_n \to M$ y $b_n \to L \neq 0$. Demostrar usando la definición que $\dfrac{a_n}{b_n} \to \dfrac{M}{L}$.
  \item Considerar $f: X \to Y$ y $B \subset Y$. Demostrar que $f(f^{-1}(B)) \subseteq B$. Dar un ejemplo donde la desigualdad sea estricta.
  \item Decidir si la sucesión $b_{n+1} = \sqrt{2 + \sqrt{b_n}}$ con $b_1 = 1$ es convergente. En caso afirmativo, determinar su límite.
  \item Considerar $(x_n)$ la sucesión acotada tal que $\liminf x_n = \limsup x_n = L$. Demostrar que $x_n \to L$.
  \item Decidir si $\sum \dfrac{n! \cdot e^n}{n^{n+1}}$ es convergente o divergente.
  \item Considerar $(a_n)$ una sucesión decreciente tal que $\sum a_n$ es convergente. Demostrar que $n a_n \to 0$.
  \item Considerar $a_1 \geq a_2 \geq \cdots \geq 0$. Demostrar que $\sum a_n$ converge $\iff \sum 2^n \cdot a_{2^n}$ converge.
\end{enumerate}

\section{Segundo parcial - Primera fecha - 20/12}

\begin{enumerate}
  \item Sea $f : \R \to \R$ y $f_n(x) = f(nx), \forall n \in \N : (f_n)$ es equicontinua en $0$. Demostrar que $f$ es constante.
  \item Considerar $F$ cerrado y $K$ compacto en $\R^n$ tales que $F \cap K = \varnothing$. Demostrar que $d(F, K) > 0$. Se define $d(F, K) = inf\{ d(f, k) : f \in F, k \in K \}$. \\ Dar un ejemplo donde $F$ y $G$ sean cerrados disjuntos tales que $d(F, G) = 0$.
  \item Considerar $f: X \subseteq \R^n \to R$ una función tal que $\forall \e > 0, \exists g_{\e}: X \to \R$ continua tal que $\forall x \in X$ se cumple que $|f(x) - g_{\e}(x)| < \e$. Mostrar que $f$ es continua.
  \item \begin{enumerate}
    \item Considerar $f_n: X \to Y$ tales que $f_n \rightrightarrows f$ y $g: Y \to Z$ uniformemente continua. Demostrar que $g \circ f_n \rightrightarrows g \circ f$.
    \item Considerar $g_n : Y \to Z$ tales que $g_n \rightrightarrows g$ y $f: X \to Y$. Es cierto que $g_n \circ f \rightrightarrows g \circ f$?
  \end{enumerate}
  \item \begin{enumerate}
    \item Demostrar que $x^n(1-x)$ converge uniformemente a $0$ en $[0, 1]$.
    \item Estudiar la convergencia puntual y uniforme de las siguientes series en el intervalo $[0, 1]$. \begin{enumerate} \item $\sum_{n \geq 1} x^n(1-x)$ \item $\sum_{n \geq 1} (-1)^n x^n(1-x) $ \end{enumerate}
  \end{enumerate}
\end{enumerate}
