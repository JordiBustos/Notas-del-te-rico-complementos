\epigraph{``Considero a cada hombre como un deudor \\ de su profesión, \\ y ya que de ella recibe sustento y provecho, \\ así debe procurar, \\ mediante el estudio, \\ servirle de ayuda y ornato."}{Francis Bacon}

Este libro recoge las notas tomadas durante el curso de Complementos de Análisis Matemático dictado por Irene Drelichman en el segundo cuatrimestre de 2024.

Las clases 1 a 8, junto con la primera parte de la clase 9, abarcan la primera mitad del curso. Las clases 9 a 22 cubren la segunda mitad, que incluye temas de diferenciación, integración y ecuaciones diferenciales, los cuales no fueron evaluados en la práctica. Se tomaron dos exámenes parciales, uno por cada mitad del curso.

Las clases 9 a 11 (hasta la mitad) cubren los conceptos básicos de Topología. Desde la segunda mitad de la clase 11 hasta la 14 se aborda la continuidad. Los teoremas de la función inversa e implícita se tratan en las clases 15 y 16. La integración se desarrolla en las clases 17 a 19. Las clases 20 a 22 (hasta la mitad) se dedican a las sucesiones de funciones, y finalmente, se introducen las ecuaciones diferenciales.

Estas notas se basan principalmente en material de los libros \textit{Principles of Mathematical Analysis} de Walter Rudin, \textit{Curso de análise vol.1} de Elon Lages Lima y \textit{Calculus} de Tom M. Apostol.
